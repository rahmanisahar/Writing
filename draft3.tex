\documentclass[useAMS,usenatbib]{mn2e}
%\usepackage{geometry}[left=1.5in, right=1in, bottom=1in, top=1in]
\usepackage{hyperref}
\usepackage{graphicx}
\usepackage{natbib}
\bibliographystyle{aas}
\usepackage{amsmath}
\usepackage{times}
\usepackage{float}

\newcommand \kpc        {\,{\rm kpc}}
\newcommand \sigmagas    {$\Sigma_{\rm \bld {gas }}$\ }
\newcommand \eqsigmagas    {\Sigma_{\rm \bld {gas }}}
\newcommand \sigmasfr     {$\Sigma_{\rm \bld {SFR }}$\ }
\newcommand \eqsigmasfr     {\Sigma_{\rm \bld {SFR }}}
\newcommand \sigmastar    {$\Sigma_{\rm \bld {star }}$\ }
\newcommand \eqsigmastar    {\Sigma_{\rm \bld {star }}}
\newcommand \halpha    {H$\alpha $\ }
\newcommand \um    {$\mu$m\ }
\newcommand \mice {$\mu$m}
\newcommand \nprime {N$^\prime$}
\newcommand \eqnprime {N^\prime}
\newcommand \Spitzer {$Spitzer$}
\newcommand \Galex {$GALEX$}
\newcommand \Herschel {$Herschel$}
\begin{document}
% TITLE
\title[WHICH STAR FORMATION LAW IN M31]{RELATION BETWEEN STAR FORMATION RATE, GAS MASS AND STELLAR MASS IN THE ANDROMEDA GALAXY}
\author[S. Rahmani $\&$ P. Barmby]{S.~Rahmani, P.~Barmby\\
Department of Physics $\&$ Astronomy, University of Western Ontario, London, ON N6A 3K7, Canada}
\maketitle

%----------------------------------------------------------------------------------------
%----------------------------------------------------------------------------------------
\begin{abstract} % it is just my GESF2014 abstract, just for looks I will change it at the end
 We investigate the relation between star formation rate (SFR), gas surface density (\sigmagas) and stellar surface density (\sigmastar) in the Andromeda galaxy (M31) using the Kennicutt-Schmidt (K-S) law and the extended Schmidt law. Shi et al. argued that the surface density of SFR is related not only to gas mass, as in the empirical K-S law,  but also to stellar mass surface density, in the ``extended Schmidt law''. Using all a combination of \halpha and 24 \um emission, a combination of FUV and 24 \um, and the total infrared emission, we estimate the total SFR in M31 to be between 0.35 M$_{\odot}$yr$^{-1}$ - 0.4 M$_{\odot}$yr$^{-1}$. We use maps of $^{12}$CO and 21cm emission to produce the \sigmagas map.% For regions that are not covered in the $^{12}$CO map, we used the dust map (obtained from $SPIRE$ data) as a tracer of total gas.
 The stellar mass surface density  map is calculated using mid-infrared IRAC imaging. Our preliminary results from testing the K-S law in M31 find a power-law index of  $N = 0.62 \pm 0.01$, at the low end of values compared with previous work. We determine additional K-S law fits using each SFR map independently with molecular hydrogen, atomic hydrogen and total gas maps. We then repeat the analysis considering the stellar mass density in order to test whether the regular or extended Schmidt laws are more appropriate for M31.

\end{abstract}
\begin{keywords}
Galaxies: Andromeda, Star Formation
\end{keywords}
%----------------------------------------------------------------------------------------
%INTRODUCTION
%----------------------------------------------------------------------------------------


\section{Introduction}
\label{sec:intro} %I can add a lot more about calculating star formation and in general why star formation is important but I don't like make my introduction too long.
%P1: Star formation laws and why they are important (introducing K-S law and E-S law)
%PB: agreed with not making it too long, but do start with a sentence about why SF is important.

Empirical star formation laws, relations, has been investigated for more than 50 years, and results never been precise. Formation of stars is a complicated process. Various phenomena affect the collapse of molecular clouds and star formation which include the environment of the star-forming region, chemical compositions, the initial mass distribution of gas, gas accretion and cooling, H$_2$ formation, etc.

The first idea about scaling the star formation rate on the galaxy scale was proposed as a connection between the total star formation rate and the mass of interstellar gas. \cite{Schmidt59} assumed the constant initial mass function (IMF) of stars, $\Psi (M)$, a stellar lifetime function {$T(M$)}, and a star formation rate {$SFR(T)$} and derived the star formation rate (SFR) over the history of the Milky Way. He scaled SFR with a power {\it n} of the gas mass. $M_{gas}$. %Then $SFR(T)\Sigma_{M}(M) = c[M_{gas}(t)]^n$, for a summation over all stellar type $M$.
A decade later, \cite{Kennicutt98a} examined another group of galaxies. He used \halpha, HI and CO observations of 61 normal spiral galaxies and 36 starburst galaxies and showed that the disk-average SFRs and gas densities for this samples are well represented by the Schmidt law with the power of $1.4 \pm 0.15$. Using this power index, the Schmidt law can be rewritten as:

\begin{equation}
\eqsigmasfr \propto \eqsigmagas^{1.4\pm0.15}
\end{equation}
which is often referred to as the Kennicutt-Schmidt (K-S) law, where \sigmasfr is surface density of star formation, \sigmagas is the surface density of gas in a galaxy and, {\it n} is the power index.


In more recent studies, \cite{Kennicutt08} investigated the local star formation law in M51 with 0.5-2 kpc resolution. For the SFR he used pa${\alpha}$ and $24\mu$m + H${\alpha}$ lines and for gas density he used constant conversion factor for CO to H$_2$. He found a correlation, from the radial variation of both SFR and gas surface density, with an index of $1.56 \pm 0.04$.
%He could not find any correlation between SFR and HI only clouds, but the correlation with the molecular cloud was about the same as the total gas.
It should be noted that the KS law can not be applied for the whole range of gas densities. In parallel studies, the star formation threshold was introduced and it was pointed out that calculated SFR is much lower than the value predicted by the KS law at low gas densities $( < 1-10\sim$ M$_{\odot}$ pc$^{-2})$ e.g. in regions far outside the optical disk \citep[e.g.,][]{Martin01, Bigiel08}.

\cite{Hunter98} have shown that, in low surface brightness galaxies the SFR density is only related to the stellar mass density. Also similar relations between stellar masses and SFR densities are seen within galaxies by \cite{Ryder94}, \cite{Hunter04} and recently by \cite{Leroy08} for specific galaxy types or limited density ranges. \cite{Shi11} found a tight relationship between stellar mass surface density, SFR and gas surface density by studying on a large sample of galaxies. They showed this relation as a power law relation and referred to it as the extended Schmidt law:

\begin{equation}
\eqsigmasfr \propto \eqsigmagas^{\eqnprime} \eqsigmastar^{\beta}
\end{equation}
where \sigmasfr is SFR surface density, \sigmagas and \sigmastar are gas mass and stellar surface density. \nprime and $\beta$ are the power law indexes. Testing new relation on 60 galaxies \nprime and $\beta$ were calculated to be $\sim$ 1.4 and $\sim$ 0.6 respectively. They showed that this relation not only predicts SFR as well as the KS law for galaxies and spatially resolved regions ($\sim$1 kpc sizes) where the KS law works, but also is acceptable for low surface brightness galaxies and regions where the KS law fails.

%P2: SFR, measuring, scaling, ...
For testing any of these star formation laws, first thing one should do is calculating the SFR. Many groups are working to find a translation from the light of star forming regions in galaxies to a rate of the formation of new stars. \cite{Kennicutt98b} calibrated luminosity of galaxies as a way of measuring the SFR in specific wavelengths using relations between the SFR per unit mass or per unit luminosity and the integrated colour of the system provided by synthesis models  \citep[e.g.,][]{Bruzual93}. Afterward many other studies tried to find the similar calibration for measuring the SFR of galaxies in other bands \citep[e.g.,][]{Kennicutt12, Calzetti12, Zhu08, Kennicutt09, Boquien10, Boquien11, Hao11}. \cite{Kennicutt09} and \cite{Hao11} introduced new SFR calibration using a combination of \halpha or FUV and far infrared (FIR) emission, respectively. In addition of using suitable calibrations for each region, considering differences between global case and local ($\sim 0.5-1$ kpc) case is important. In section 3, we are talking about the SFR in the Andromeda galaxy and how we did measure it.


%P3: Gas cloud, variety, observing etc
Calculating surface density of gas in galaxies is the other most important ingredients of testing star formation laws. Considering the effect of the interstellar medium (ISM) on studying the star formation law is important due to the fact that stars are born from the gas also release their material into the ISM when they reach the end of their evolution. A map of the total gas in the ISM is produced by direct observations of the gas or using interstellar dust as a tracer. Neutral and molecular hydrogen are the most common elements in the ISM. Therefore, for producing the map of the total gas in galaxies, maps of these two components can be added together and multiplied by a constant factor to account for heavier elements which are mostly He. Another way to do the mapping is assuming that the ratio of total gas and dust is constant across a galaxy, and convert dust observations to the map of total gas. Section 4 is about the calculation of the gas surface density for the Andromeda galaxy.


%PB: shorten or leave out dynamical mass, since you don't discuss it further
%P4: Stellar mass and how to measure it
Additionally, for testing the Extended Schmidt law, measuring the stellar mass density is necessary. Measuring the mass of the stellar population is indirect and subject to significant uncertainties. In principle there are two ways to measure stellar mass in galaxies. First is measuring the dynamical mass of galaxies using kinematics \citep{Cappellari06} or lensing \citep{Auger09}, and then modelling the mass of the dark matter and subtracting it from the measured mass. This method has been successful to predict and subtract the mass of the dark matter, which is the dominant mass in galaxies \citep{Zaritsky94},  and can easily cause uncertainties as high as the order of the measurement.

The second method is based on stellar population models \citep[e.g.][]{ Bruzual93, Kotulla09} and using them to connect stellar mass to an observable, like luminosity in different wavelengths, colours, spectral energy distribution from spectroscopy or multi-band observations. Having two independent methods to measure stellar mass helps to compare the results and check whether there are any systematic differences. Comparing the results from measuring stellar mass in stellar clusters shows there is no huge difference. Comparing the results from measuring stellar mass within galaxies shows that the deferences are on the order of a few x10 percent, so one should be more careful to model subtleties \citep{McLaughlin05}. Calculating stellar mass in the Andromeda galaxy is described in section 5.

%P5(2*): What we did and why Andromeda and
In this project we test and compare both the K-S law and the extended Schmidt law on the Andromeda galaxy (M31). M31 is the closest spiral galaxy to the Milky Way, therefore, we can have higher resolution images of it than we have of any other spiral galaxy. Having high resolution images provides us data from different regions within
the galaxy with different physical situations (e.g. metallicity, surface brightness and etc.). All these parameters make the Andromeda a suitable test bed for studying the scaling law.
%PB: Give a little more detail in the followin sentence
Furthermore, the range of the power index of K-S law calculated for the Andromeda galaxy is between 0.5 - 2 \citep[e.g.,][]{Tabatabaei10,Ford13}. Therefore, testing extended Schmidt law on the Andromeda galaxy would help to solve the problem with these controversial results.


%----------------------------------------------------------------------------------------
%DATA
%----------------------------------------------------------------------------------------
\section{Data}
\label{sec:data}
Since the M31 is the closest spiral galaxy to MW, it was aimed by many probes and telescopes, both space based and ground based ones. This advantage, provides us variety of data from X-ray to radio wavelengths. This wide ranges of data gives us the ability to measure all the required parameters to test the star formation laws in M31. Table~\ref{table:data} lists all the data we used in this paper. Each data comes with different Full Width Half Maximum (FWHM), spacial resolution and grids. In order to make sure that all our data contains the same amount of information, we smooth them using Aniano's Kernels' library and convolution code \citep{Aniano12}, to highest resolution available. Also considering Nyquist relation \citep{Nyquist}, we re-sample and re-grid all our maps to have a proper pixel size.

\begin{table}
\caption{Data used in this study.}
\label{table:data}
\begin{tabular}{@{}lccc}
\hline\hline
Wavelength & FWHM & Telescope
& Ref. \\
\hline
%500 \um & 36\arcsec & $Herschel$ & \cite{Fritz12} \\
%350 \um & 24\arcsec.5 & $Herschel$ & \cite{Fritz12} \\
%250 \um & 18\arcsec.2 & $Herschel$ & \cite{Fritz12} \\
2250 \AA & 6\arcsec & $GALEX$ & \cite{Martin05}\\ %:PB: GALEX, KPNO, IRAM should not be in $$
1550 \AA & 4\arcsec.5 & $GALEX$ & \cite{Martin05}\\ %PB (they are abbreviations)
6570 \AA  & 1\arcsec & $KPNO$& \cite{Massey07}\\
3.6 \um & 1\arcsec.7 & $Spitzer$ & \cite{Barmby06} \\ %PB: write {\it Spitzer}, {\it Herschel}
4.5 \um & 1\arcsec.7 & $Spitzer$ & \cite{Barmby06} \\ %PB because those are both missions and names
8 \um & 1\arcsec.9 & $Spitzer$ & \cite{Barmby06} \\ % PB: of people - italics means the mission, not
24 \um & 6\arcsec & $Spitzer$ & \cite{Gordon06} \\ %PB: the person
70 \um & 18\arcsec & $Spitzer$ & \cite{Gordon06} \\
160 \um & 12\arcsec & $Herschel$ & \cite{Fritz12} \\
2.6 mm & 23\arcsec & $IRAM$ 30-m & \cite{Nieten06}\\
21 cm & 60\arcsec \times \ 90\arcsec & $DRAO$ & \cite{Chemin09}\\
\hline
\end{tabular}
\end{table}

%PB: in general it's good to write number~unit, ie 2750~\AA 
%PB: this means that number and unit don't end up on separate lines
\subsection{Ultra Violet (UV) emission}

One of the calibrations we used to calculate SFR uses FUV emission as a tracer of the star forming regions. FUV emission of M31 was observed by Galaxy Evolution Explorer ($GALEX$; \citep{Martin05}). $GALEX$ telescope data contains information for two bands, FUV:1350-1750 and NUV:1750-2750~\AA, with FWHM of 4\arcsec .5, and 6 \arcsec.

\subsection{Visible light}
Our other method to calculate the SFR was using a combination of \halpha emission and 24\um emission. \cite{Massey06, Massey07} mapped UBVRI and \halpha emission line of M31 as part of the Nearby Galaxies Survey using KPNO telescope. \halpha maps, which were obtained from NOAO science archive are available in 10 overlapping fields. Mosaic image of the maps were created using Montage \citep{Montage}. Assuming that all stars have the same fraction of \halpha light in their spectra, we subtracted the R band emission from the \halpha using averaged scaling factor between R band images and \halpha provided by \cite{Azimlu11}. We also took masked map of stars from \cite{Azimlu11} and masked all point sources (Figure~\ref{fig:halpha}). Details of making the making \halpha map can be find in the Appendix ~\ref{appendix: halpha}.

\subsection{IR and sub mm Data}
In IR and sub mm bands both $Spitzer$ \citep{Wener04} and $Herschel$ \citep{Pilbratt10}  space telescopes observed the M31. The Infrared Array Camera (IRAC; \citep{Fazio04}) on the $Spitzer$ is a camera observing at four channels (3.6, 4.5, 5.8, and 8 \um). IRAC observation of M31 were obtained by \cite{Barmby06}, which covered $3\degr.7 \times 1\degr.6$. IRAC 3.6 and 4.5\um bands  were used to calculate the stellar mass. Multiband Imaging Photometer for Spitzer (MIPS) observed the  M31 at 24, 60, and 160\um and covered region $\sim 1\degr \times 3\degr$ \citep{Gordon06}.

Calculation of the SFR using TIR emission needs maps of the m31 in 8, 24, 70, and 160 \um. IRAC 8, MIPS 24, MIPS 70 bands were used to calculate TIR emission, but due to better quality and resolution of $Herschel$s$'$data, PACS (Photodetector Array Camera and Spectrometer) \citep{Poglitch10} were used. PACS 160\um observation of the Andromeda galaxy alongside of PACS 100 \um band and three (250 \um, 350\um and 500 \um) SPIRE (Spectral and photometric Imaging Receiver) \citep{Griffin10} bands are covered $\sim 5\degr.5 \times 2\degr.5$ regions \citep{Fritz12}.

\subsection{Radio Emission}
Emission from $^{12}$CO(J:$1\rightarrow0$) line which, was observed on-the-fly with the IRAM 30-m telescope\citep{Nieten06} was used to calculate the H$_2$ column density of interstellar medium. This map coveres $2\degr \times 0\degr.5$ that has best resolution available (FWHM = $23\arcsec$) but smallest coverage of halo of the galaxy among our maps. In this project we used data of the 21 cm emission from the atomic gas (HI) in the M31 from \cite{chemin9}. They used Synthesis telescope and the 26 m antenna at the Dominion Radio Astrophysical Observatory to map HI emission in 21cm band of the M31.


\begin{figure*}
 {\vfil  \includegraphics[width=164mm]{halpha_copy.eps}
  \caption{Mosaic created using Montage programme from 6 fields of \halpha emission maps of m31 from \citpe{Massey07}. Result from Montage was continuum subtracted and masked out all point sources. Centre of galaxy was masked out due to saturation of data in R-band image.\label{fig:halpha}}
 \vfil}
\end{figure*}


%----------------------------------------------------------------------------------------
%STAR FORMATION RATE
%----------------------------------------------------------------------------------------
%----------------------------------------------------------------------------------------
\section{Star Formation Rate}
\subsection{FUV plus 24\um Star Formation Rate}

In this project first star formation map is produced using a calibration of a combination of the FUV emission and 24\um emission introduced by \cite{Hao11}. Since peak of the emission of massive and young stars (O - B and A type) are in the UV, this wavelength is one of the most common used star formation rate indicators \citep[e.g.,][]{Kennicutt89}. The UV indicates star forming region with age of $\sim 100$ Myr \citep[e.g.,][]{Kennicutt98, Calzetti05}. However, downside of only using the FUV light as a tracer of star forming region is that this emission is really sensitive to dust extinction. On the other hand 24\um emission is dominated by emission from dust heated by UV photons from young and hot stars. This emission is sensitive to star formation time scale of $\la10$ Myr \citep{Calzetti07}.
The FUV band of \Galex and the 24\um band of \Spitzer data was used to measure the SFR calibrated by \cite{Hao11} as follow:
\begin{equation}
\label{equ: fuvplus24}
SFR =4.46\times10^{-44}[L(FUV)_{obs}+3.89L(24\mu m)]
\end{equation}
where unit of SFR is M$_{\odot}$yr$^{-1}$, and L(24$\mu$m) is in erg/s. $L(FUV)_{obs}$ is the luminosity of galaxy in the FUV emission. The index of obs is shown to indicate that the FUV emission is not corrected for effect of extinction. Nevertheless, this emission is corrected for the foreground stars emission. To do so, we followed a method introduced by \cite{Leroy08}. We assumed for any pixel, in maps with pixel size the same as the pixel size of NUV map, if I$_{NUV}$/I$_{FUV}$ $>$ 15 that pixel value is dominated by foreground emission and masked them out. We also masked the same region in the galaxy for 24\um map.

%----------------------------------------------------------------------------------------
\subsection{\halpha plus 24\um Star Formation Rate}
\label{sec:sfr_halpha}

The SFR can be calculated using combination of \halpha and $Spitzer$ 24\um maps. \halpha emission of galaxies is dominated by light emitted from young and massive stars. Star formation time scaled traced by this emission is $\sim 6-10$ Myr \citep[e.g.,][]{Kennicutt09, Calzetti12}. Therefore \halpha alone could be used to calibrate the SFR \citep[e.g.,][]{Osterbrock06, Kennicutt09}. However, \halpha emission is very sensitive to dust extinction,too. Therefore, combination of the \halpha and 24\um of is considerable choice for calculating the SFR. we used calibration introduced by \cite{Kennicutt09} to measure the SFR:
\begin{equation}
\label{equ: halphaplus24_g}
SFR = 5.5 \times 10^{-42}[L(H{\alpha})_{obs} + 0.020L(24\mu m)]
\end{equation}
where L(H${\alpha}$)$_{obs}$ is the observed H${\alpha}$ luminosity without correction for internal dust attenuation, given in the unit of erg/s. L(24$\mu$ m) is the $24\mu$m IR luminosity in erg/s, and SFR is in M$_{\odot}$yr$^{-1}$. It was indicated that the formulation above can only be used at global situation (i.e. total SFR).

\cite{Calzetti07} introduced a new calibration using the same bands in case of local regions (figure ~\ref{fig:sfr_halpha}):
\begin{equation}
\label{equ: halphaplus24_l}
SFR = 5.5 \times 10^{-42}[L(H{\alpha})_{obs} + 0.033L(24\mu m)]
\end{equation}
units here are the same as equation~\ref{equ: halphaplus24_g}. Equation~\ref{equ: halphaplus24_l} is useful for regions less than $\sim 1$ kpc. According to  \cite{Calzetti07} calibration constant in equations ~\ref{equ: halphaplus24_g} and ~\ref{equ: halphaplus24_l} can be different by changing an IMF. Changes in the IMF mostly affect on calibration based on \halpha luminocity, mostly because significant amount of the ionising photons comes from the stars more massive than $\sim 20$M$\sun$. Changing the calibration constant would change the total SFR by a factor of 1.5. In this paper we used \cite{Kroupa01} IMF with upper mass limit of 100M$\sun$. Total SFR calculated from equation~\ref{equ: halphaplus24_g} is 0.35 M$\sun$yr$^{-1}$, which is in good agreement with previous studies.
%PB: can you put uncertainties on the total SFR numbers?

\subsection{Total Infrared Star Formation Rate}
\label{sec:sfr_fir}
Total infrared emission of a system, can also be used as a tracer of the SFR. Dust absorbs radiation from hot and young stars and re-emitt it in infrared wavelength, however there is no one-to-one mapping between IR and UV emission \citep{Calzeti12}, therefore, integrating over full wavelength range of the IR part and calculating the TIR luminosity is better tracer of the SFR instead of the IR single-band emission. \cite{Calzetti07} assumed that stellar bolometric emission is completely absorbed and re-emitted by dust, i.e., $L_{star}(bol) = L(TIR)$ and calibrated the TIR luminosity of a system to calculate the SFR for a stellar population undergoing constant star formation over $\tau = 100$ Myr:

\begin{equation}
\label{equ:sfr_fir}
SFR(\rm TIR) = 2.8 \times10^{-44}L(\rm TIR)
\end{equation}
where SFR(TIR) and L(TIR), star formation rate calculated from TIR emission and TIR luminosity are in M$_{\odot}$yr$^{-1}$ and erg/s respectively.

%PB: I like this figure! Looks nice.
\begin{figure*}
        \centering
        \begin{subfigure}\\(a){}
                \includegraphics[width=164mm]{sfr_fuv.eps}
                %\caption{A gull}

               \label{fig:sfr_halpha}
        \end{subfigure}
        \begin{subfigure}\\(b){}
                \includegraphics[width=164mm]{sfr_halpha.eps}
                %\caption{A gull}

               \label{fig:sfr_halpha}
        \end{subfigure}
       % \quad
        %add desired spacing between images, e. g. ~, \quad, \qquad etc.
          %(or a blank line to force the subfigure onto a new line)
        \begin{subfigure}\\(c){}
                \includegraphics[width=164mm]{sfr_fir.eps}
                %\caption{A tiger}

                \label{fig:sfr_fir_small}
        \end{subfigure}
         \quad
               \caption{SFR map from a combination of \halpha and 24\um emission (top) and total infrared emission (bottom)}
\label{ISOnIRS}
\end{figure*}


IR luminosity can be measured from integration of IR part of a spectral energy distribution (SED) of a galaxy or combining photometry datas in different IR wavelengths. For second approach \cite{Draine07} modelled $Spitzer$ data and calibrated IR single-band photometry data to calculate the TIR luminosity. \cite{Boquien10} tested and modified this calibration as:
\begin{equation}
 \label{equ: TIR}
L(\rm TIR) = 0.95L(8) + 1.15L(24) + L(70) + L(160)
\end{equation}
where L(8 \um),L(24 \um), L(70 \um) and L(160\um) are luminosities of galaxy in 8 \um, 24 \um, 70 \um and 160 \um in unit of erg/s. We used equation ~\ref{equ:sfr_fir} to calculate second map of SFR of M31 (figure ~\ref{fig:sfr_fir_small}). Using this method we calculated the total SFR 0.40 M$_{\odot}$yr$^{-1}$. Table ~\ref{table:sfr} compares total SFR from literature and current work.

\begin{table*}
\begin{minipage}{100mm}
\caption{Comparison of Total Star Formation Rate of M31}
\label{table:sfr}
\begin{tabular}{@{}lcc}
\hline\hline
Ref.&Method&SFR(M$_{\odot}$yr$^{-1}$) \\
\hline
Current work&FUV and 24\um&0.31\\
Current work&\halpha and 24\um&0.35\\
Current work&TIR luminosity&0.4\\
\cite{Ford13}&FUV and 24\um&0.25\\
\cite{Ford13}&TIR luminosity&0.48-0.52\\
\cite{Azimlu11}& \halpha and 24\um&0.34\\
\cite{Azimlu11}&Extinction corrected \halpha&0.44\\
\cite{Tabatabaei10}&Extinction corrected \halpha&0.27-0.38\\
\cite{Barmby06}&Infrared 8\um luminosity& 0.4\\
\hline
\end{tabular}
\end{minipage}
\end{table*}

%----------------------------------------------------------------------------------------
%ISM
%----------------------------------------------------------------------------------------
\section{Interstellar Medium}
%----------------------------------------------------------------------------------------
%Direct Measurement of the Gas Mass
%----------------------------------------------------------------------------------------
%\subsubsection{Direct Measurement of the Gas Mass}

The surface density of gas in the ISM of galaxies can be measured by direct observation of the neutral and molecular hydrogen. This method can be promising provided that the spatial resolution of observational data is high enough; nonetheless, due to technical limitations, attaining enough spatial resolutions to resolve clouds is almost impossible. This problem shows itself for distant galaxies more than nearby galaxies. As a result, using this method is limited and has a lot of uncertainties.

%\subsubsection*{Molecular Clouds}
%PB: these two paragraphs contain a lot of background information that could either be moved to the
%PB introduction or be omitted; this info doesn't belong here.
 The molecular form of hydrogen in the ISM has no permanent electric dipole moment, which makes it really hard to detect. H$_2$ molecules are located in cool and dense molecular clouds, but fortunately they are not the only component of the molecular gas. The second dominant component, helium, is a mono-atomic gas and has the same problem as hydrogen, but the molecular gas also contains heavier elements such as carbon and oxygen which are combined to form CO \citep{Bolato13}.

CO has a weak permanent electric dipole moment and a ground rotational transition with low excitation energy. Given its low energy and critical density, CO can easily be excited even in cold molecular clouds. Hence CO (usually the $J(1\rightarrow 0)$ rotational transition, observed at 2.6 mm) is used as a tracer of the mass of the molecular cloud dominated by molecular hydrogen \citep[see, for example,][] {Sanders84}. Higher rotational transition of CO can be used as a tracer as well, but they are not as common as the $J(1\rightarrow 0)$. The relation between CO emission and H$_2$ cloud mass is shown as

\begin{equation}
\label{equ:conversion}
\rm N_{H_2}/\rm cm^{-2} = X_{CO} \times I_{CO}/\rm K km s^{-1}
\end{equation}
where X$_{CO}$ is the conversion factor (also known as the X-factor). X-factor could be different in regions within a galaxy due to difference in metallicity. Though assuming constant conversion factor for cases such as M82 and M31 can lead to global molecular gas mass estimates that are very accurate,  but in regions with low metallicity there are uncertainties. Different groups are working on the various galaxies to measure the X-factor for each \citep{Wilson95, Bosselli02, Bolato13}.. In case of M31, different values of X-factor ($1-5.6 \times 10^{20}$) were adopted in the previous studies \citep[e.g.][]{Ford13, Bolato13, Leroy11, Bolato08, Nieten06, Sofue94, Strong88}; however, since any constant differences in X-factor leads to horizontal changes in a plot of log(SFR) versus Log($\Sigma_{\rm {gas}}$), and does not affect our K-S law power index for molecular gas,  we chose $X_{\rm {CO}}= 2 \times 10^{20}$  the same as many other works \citep[e.g.][]{Ford13, Smith12}. This assumption might have some effect on the total gas mass but considering the fact that HI mass dominates H$_2$ mass in M31 this effect is small.
The total gas mass was calculated from:
\begin{equation}
\label{equ:total_gas}
\rm M_{\rm \bld {total \, gas}} = 1.36[M_{HI}+M_{H_2}]
\end{equation}
where  factor of 1.36 is a constant to consider He and the other heavier elements effect on mass, M$_{HI}$ is mass of neutral hydrogen obtained from 21cm observations and M$_{H_2}$ is from equation ~\ref{equ:conversion} in unit of M$\sun$.
%----------------------------------------------------------------------------------------
% Dust Emission as a Tracer of Gas
%----------------------------------------------------------------------------------------

%\subsubsection{Dust Emission as a Tracer of Gas}
%A gas map of a galaxy can be produced using the dust mass and assuming the dust mass to gas mass ratio across the galaxy \citep{Hildebran83}. This method is the most convenient in order to measure gas mass in distant galaxies, which studying the gas cloud in their ISM by using direct methods is almost impossible and not practical. In M31, for regions which are not covered in CO map we used a dust map obtained \cite{Drain14}. They also showed that dust/gas mass ratio from radius 0-25Kpc as follow:

%\begin{equation}
%\label{eq:dust2gas_vs_R}
%\frac{M_d}{M_H} \approx
%\left\{ \begin{array}{l l}
%0.0280 \exp(-R/8.4\kpc)  & R< 8\kpc\\
%0.0165 \exp(-R/19\kpc)   & 8\kpc < R < 18\kpc\\
%0.0605 \exp(-R/8\kpc)    & 18\kpc < R \la 25\kpc ~~~,\\
%\end{array}
%\right.
%\end{equation}

%There are many disadvantages regarding the use of the gas to dust ratio in the ISM. In order to use the dust as a tracer of the gas mass, first a relationship between the optical depth and the mass of the dust and then a relationship between the mass of the dust and the mass of the gas must be derived. \cite{Draine03} pointed out that this approach is uncertain because of the uncertainly in the radiative efficiency of dust grains. Also methods of measuring the gas to dust ratio in galaxies are not accurate, which means that there are problems in both steps \citep{Hildebran83}.
 %Certainly the gas to dust ratio as a tracer of the total gas in galaxies has a number of problems. Unfortunately, at present, given the available data, we cannot measure the gas mass trough whole galaxy directly. Therefore, so far this method is the best available solution for measuring the total gas in regions further out of the disk of galaxy.

%----------------------------------------------------------------------------------------
%STELLAR MASS
%----------------------------------------------------------------------------------------
\section{Stellar Mass}
\label{starmass}
Stellar mass can be measured using a suitable calibration between flux/luminosity of stars in specific wavelengths and stellar mass. The basic idea behind these kind of calibrations comes from the fact that one can use the star formation histories recovered from synthesizing the stellar optical colour magnitudes in different regions in the galaxies. The calculated stellar mass can be used  to calibrate the fluxes from the stars as the easily measurable parameters of stars. Using NIR bands, different groups tried to make a map of stellar mass distributions within galaxies \citep[e.g.,][]{Elmgreen84}.  Using $B-V$ colours \citep{Bell01} and $Spitzer$ 3.6- and 4.5-\um, M/L correction, \cite{Kendall08} produced stellar mass surface densities of M81. %PB: do you mean M81 here? If so, it's not quite clear why you're mentioning a different galaxy.

We calculated the stellar mass work using calibration of the 3.6\um fluxes by \cite{Eskew12}. They used 3.6\um fluxes for calibration of the stellar mass, because this band-pass is almost insensitive to the emission from young stellar population, dust absorption and emission. By using data from the large Magellanic cloud, they find an empirical relation between stellar mass and luminosity of stars as following:

\begin{equation}
\label{equ:eskew}
M _{\star}= 10^{5.97} F_{3.6}(\frac{D}{0.05})^2
\end{equation}

where M$_{\star}$ is the stellar mass in mass of The Sun, $F_{3.6}]$ is the 3.6\um flux in $Jy$ , and D in the distance of the galaxy in Mpc. Figure ~\ref{fig:stellarmass} shows the stellar mass map for M31 using IRAC 3.6\um.
\begin{figure*}
\centering
\includegraphics[width=164mm]{star.eps}
\caption{Stellar Mass surface density. This mass is produced using $IRAC$ 3.6 $\mu$ m data and equation ~\ref{equ:eskew}}
\label{fig:stellarmass}
\end{figure*}

Using this method we calculated the total stellar mass of galaxy is $6.9 \times 10^{10}$ with 6$\%$ uncertainty. %PB: need to compare with other results, as you did with SFR.
%----------------------------------------------------------------------------------------
%SFR LAWS
%----------------------------------------------------------------------------------------
\section{Scaling SFR}
\subsection{Fitting method}
%For investigating and comparing the SFR laws in M31
Given available SFR, gas mass and stellar mass map examination and comparison of K-S law and the extended Schmidt law are possible. We investigated the K-S law and the extended Schmidt law by using  pixel by pixel method in whole galaxy as well as comparing the laws in three different regions of galaxy based on their distance from the centre of the galaxy. Knowing that, we can re-write these two star formation laws in a logarithmic scale we would have two linear equations:

\begin{figure*}
\centering
\includegraphics{ks_tot_compt.png}
\caption{The result from fitting the Kennicutt-Schmidt law on data from whole galaxy using pixel by pixel method. Pixel size on each plots are different. } %PB: somewhere in this section need to mention the
%PB: different pixel sizes and the reason for them.
\label{fig:ks,tot}
\end{figure*}


\begin{subequations}
\begin{align}
\label{eq:sfr_law_log}
\log_{10} \eqsigmasfr = N~\log_{10} \eqsigmagas + A, \\
\log_{10} \eqsigmasfr = \eqnprime~\log_{10} \eqsigmagas + \beta~\log_{10}\eqsigmastar  + A,\\
\end{align}
\end{subequations}
units are the same as the units in the K-S law and the extended Schmidt law.

we find the fitting parameters by applying the hierarchical Bayesian linear regression method as described in \cite{Shetty13} paper. Shetty and his team used a Bayesian linear regression approached to develop a new method to find the K-S law parameters, considering the measurements uncertainties as well as hierarchical data structure. For K-S law fitting parameters, we used a 'R' code provided by Shetty's group. Uncertainties on each observable quantity was measured considering the background noise photometry errors and calibration uncertainties.%PB: are the background noise errors and calib uncertainties discussed elsewhere in this paper (in which case reference the section) or do they need to be discussed here.

For the extended Schmidt law, we expended the code from using simple linear regression which is the case of K-S law to use multiple linear regression. In this case we could examine the effect of the stellar mass in the SFR as shown in equation ~\ref{eq:sfr_law_log}.

We also test each law on three different regions. The regions were chosen the same as regions introduced in \cite{Drain14}. This regions are were the dust/H ratio declines monotonically hence metallicitty changes in the same manner.  %PB: might need a little more explanation here.

\begin{equation}
\label{eq:dust2gas_vs_R}
\frac{M_d}{M_H} \approx
\left\{ \begin{array}{l l}
0.0280 \exp(-R/8.4\kpc)  & R< 8\kpc\\
0.0165 \exp(-R/19\kpc)   & 8\kpc < R < 18\kpc\\
0.0605 \exp(-R/8\kpc)    & 18\kpc < R \la 25\kpc ~~~,\\
\end{array}
\right.
\end{equation}

Testing the SFR in these three regions gives us the tool to test not only effect of the distance from the centre of galaxy on SFR but also effect of the metallicity on that. Tables ~\ref{table:sfr_law_all} through ~\ref{table:sfr_law_25} show fitting parameters and their uncertainties for whole galaxy and each different regions. all the results from \halpha SFR map will be discuss in appendix ~\ref{app:halpha} %PB: Why?

\subsection{Star formation laws}
We had three SFR maps, three gas mass density maps. We applied both laws on whole galaxy on each two combination of two of them, along with applying the laws on three different regions of them. As a result we were be able to test each law 36 times. In total we produced 72 different plots which show the either K-S law or the extended Schmidt law, and helps us to compare both laws in whole galaxy and in different regions. Additionally, these plots provide us a tool to compare the differences in SFR calibrations, ISM gas tracers, and effect of metallicity on SFR.    

Figure ~\ref{fig:ks,tot} shows the pixel by pixel fitting of K-S law from whole galaxy. In upper level, SFR is calculated using TIR emission form section ~\ref{sec:sfr_fir} and it is plotted versus gas surface density traced by only molecular gas (H$_2$),  only atomic gas (HI), and total gas. The only difference between upper level and lower level is SFR tracer in this case is FUV emission.Figure ~\ref{fig:ks,8}, ~\ref{fig:ks,18} and ~\ref{fig:ks,25} are the same as figure ~\ref{fig:ks,tot}, but in these figures, instead of whole galaxy they are showing the regions with R less than 8$\kpc$, $8\kpc < $R $< 18\kpc$, and $18\kpc <$ R $\la 25\kpc$, respectively.  

\begin{figure*}
\centering
\includegraphics{ks_8kpc_all.png}
\caption{same as figure ~\ref{fig:ks,tot}, but for the regions with R less than 8$\kpc$ }
\label{fig:ks,8}
\end{figure*}

\begin{figure*}
\centering
\includegraphics{ks_8kpc_18kpc_all.png}
\caption{same as figure ~\ref{fig:ks,tot}, but for the regions with $8\kpc < $R $< 18\kpc$}
\label{fig:ks,18}
\end{figure*}

\begin{figure*}
\centering
\includegraphics{ks_18kpc_25kpc_all.png}
\caption{same as figure ~\ref{fig:ks,tot}, but for the regions with $18\kpc <$ R $\la 25\kpc$}
\label{fig:ks,25}
\end{figure*}

%PB: the instructions for MNRAS say that 3D plots can be included in papers -- can you please
%PB: look into this and figure out if the format that they want is feasible to generate?
We took the similar approached for the extended Schmidt law. Figures ~\ref{fig:es,tot1} to ~\ref{fig:es,256} shows the result of Bayesian fitting of this law. For the extended Schmidt law, we show our results in 3D to illustrate the relation between SFR, gas mass density and stellar mass density more clearly. In these series of plots, X-axis is the gas mass  surface density, either H$_2$, HI or total gas, Y-axis is the SFR(TIR) or SFR(FUV + 24 $\mu$m), and Z-axis is the stellar mass surface density. Shadow of each surface is also shown. For a comparison, one cane asily conclude that shadow on X-Y surface is a reproduction of K-S law.  Results from fiitings in both K-S law and the Extended Schmidt law are shown in tables ~/ref{table:sf-laws_al} to ~/ref{table:sf-laws_25}. 






%We did the same for the extended Schmidt law including stellar mass map to our other maps (fig. ~\ref{fig:es-law}), results are listed in table ~\ref{table:sf-laws_pix} and table ~\ref{table:sf-laws_cat} .



%\begin{figure}
%\centering
%\includegraphics[width=82mm]{es_tot_fuv_vs_h2.png}
%\caption{Caption}
%\label{fig:es}
%\end{figure}
\begin{figure}
\begin{subfigure}
  \centering
  \includegraphics[width=82mm]{es_tot_fuv_vs_h2.png}
  \caption{a}
  \label{fig:sfig1}
\end{subfigure}%
\begin{subfigure}
  \centering
  \includegraphics[width=82mm]{es_tot_fir_vs_h2.png}
  \caption{b}
  \label{fig:sfig2}
\end{subfigure}
\begin{subfigure}
  \centering
  \includegraphics[width=82mm]{es_tot_fir_vs_hi.png}
  \caption{c}
  \label{fig:sfig2}
\end{subfigure}
\end{figure}
\begin{figure}
 \begin{subfigure}
  \centering
  \includegraphics[width=82mm]{es_tot_fuv_vs_hi.png}
  \caption{d}
  \label{fig:sfig2}
\end{subfigure}
\begin{subfigure}
  \centering
  \includegraphics[width=82mm]{es_tot_fir_vs_tot.png}
  \caption{e}
  \label{fig:sfig2}
\end{subfigure}
\begin{subfigure}
  \centering
  \includegraphics[width=82mm]{es_tot_fuv_vs_tot.png}
  \caption{f}
  \label{fig:sfig2}
\end{subfigure}
\caption{plots of....}
\label{fig:fig}
\end{figure*}

%PB: in these tables, having the numbers all in the same format  (eg with or without the 1e-2
%PB type notation) makes it easier to compare. Negative numbers should be written $-x.xx$.
%THE REST WOULD BE THE comparison of the results:
\begin{table*}
\caption{Fitting parameters of the SF laws from applying the Bayesian fitting on all galaxy regions.}
\label{table:sf-laws_all}
\begin{tabular}{@{}lccccccccccc}
\hline\hline
SFR Tracer& Gas Tracer & N & \sigma$_N$ & $\log$(const$_{KS}$)& \nprime & $\sigma_{\eqnprime}$ & \beta& \sigma$_\beta$&$\log$(const$_{ES}$)&$\sigma_{scatter}$ \\
\hline
FIR&H$_2$only&1.03 &9.47e-3 &-8.92 &0.95 &9.78e-3 &0.34 &4.25 &-9.57 & & &
FIR&HI only& 1.15&9.13e-3 &-9.87 &0.77 &8.84e-3 &0.74 &4.58e-3 &-10.71 & & &
FIR&Total gas&0.81 &6.27e-3 &-9.84 &0.30 &8.73e-3 &0.60 & 1.21e-2&-10.40 & & &
FUV + 24\um&H$_2$only& 1.15&1.16e-2 &-9.28 &1.10 &1.21e-2 &0.25 &4.88e-3 &-9.76 & & &
FUV + 24\um&HI only& 1.21&9.13e-3 &-9.87 &0.80 &7.33e-3 &0.60 &4.02e-3 &-10.70 & & &
FUV + 24\um&Total gas&0.78& 4.67e-3&-10.05 &0.80 &7.35e-3 &0.60 &3.98e-3 &-10.70 & & &
\hline
\end{tabular}
\end{table*}

\begin{table*}
\caption{Fitting parameters of the SF laws from applying the Bayesian fitting on regions with R less than 8$\kpc$ .}
\label{table:sf-laws_8}
\begin{tabular}{@{}lccccccccccc}
\hline\hline
SFR Tracer& Gas Tracer & N & \sigma$_N$ & $\log$(const$_{KS}$)& \nprime & $\sigma_{\eqnprime}$ & \beta& \sigma$_\beta$&$\log$(const$_{ES}$)&$\sigma_{scatter}$ \\
\hline
FIR&H$_2$only&0.96 &1.48e-2 &-8.95 &0.88 &1.48e-2 &0.30 &6.11e-3 &-9.54 & & &
FIR&HI only&0.79 &0.03 &-9.89 &0.47 &0.30 &1.57 &0.28 &-12.11 & & &
FIR&Total gas&1.00 &0.03 &-10.12 &0.63 &0.03 &0.53 &0.08 &-10.62 & & &
FUV + 24\um&H$_2$only&1.04 &1.84e-2 &-9.29 &1.00 &1.85e-2 &0.18 &7.17e-3 &-9.65 & & &
FUV + 24\um&HI only&0.68 &0.02 &-10.03 &0.92 &0.06 &0.42 &0.04 &-10.50 & & &
FUV + 24\um&Total gas&0.87 &0.026 &-10.12 &0.75 &0.03 &0.16 &0.04 &-10.28 & & &
\hline
\end{tabular}
\end{table*}

\begin{table*}
\caption{Fitting parameters of the SF laws from applying the Bayesian fitting on regions with $8\kpc < $R $< 18\kpc$.}
\label{table:sf-laws_18}
\begin{tabular}{@{}lccccccccccc}
\hline\hline
SFR Tracer& Gas Tracer & N & \sigma$_N$ & $\log$(const$_{KS}$)& \nprime & $\sigma_{\eqnprime}$ & \beta& \sigma$_\beta$&$\log$(const$_{ES}$)&$\sigma_{scatter}$ \\
\hline
FIR&H$_2$only&1.08 &1.26e-2 &-8.90 &1.00 &1.29e-2 &0.40 &5.74e-3 &-9.66 & & &
FIR&HI only&1.09 &2.64e-2 &-8.93 &0.77 &0.025 &0.75 &0.01 &-10.71 & & &
FIR&Total gas&0.75 &0.02 &-10.02 &0.47 &0.01 &0.55 &0.01 &-10.53 & & &
FUV + 24\um&H$_2$only&1.21 &1.53e-2 &-9.27 &1.15 &1.58e-2 &0.33 &6.57e-3 &-9.66 & & &
FUV + 24\um&HI only&1.18 &2.89e-2 &-9.81 &0.81 &2.08e-2 &0.61 &1.12e-2 &-10.73 & & &
FUV + 24\um&Total gas&0.71 &1.34e-2 &-10.02 &0.44 &0.01 &0.45 &0.01 &-10.58 & & &
\hline
\end{tabular}
\end{table*}

\begin{table*}
\caption{Fitting parameters of the SF laws from applying the Bayesian fitting on regions with $18\kpc <$ R $\la 25\kpc$.}
\label{table:sf-laws_25}
\begin{tabular}{@{}lccccccccccc}
\hline\hline
SFR Tracer& Gas Tracer & N & \sigma$_N$ & $\log$(const$_{KS}$)&$\sigma_{scatter,KS}$& \nprime & $\sigma_{\eqnprime}$ & \beta& \sigma$_\beta$&$\log$(const$_{ES}$)&$\sigma_{scatter,ES}$ \\
\hline
FIR&H$_2$only&1.57 &0.26 &-8.80 &6.03e+2&0.97 &0.20 &1.19 &0.1 &-10.94 &6.44e+2 & 
FIR&HI only&1.17 &0.05 &-9.65 &3.60e+2&0.76 &0.03 &0.75 &0.01 &-10.72 &3.96e+2 & 
FIR&Total gas&0.84 &0.02 &-9.85 &3.13e+2&0.50 &0.01 &0.48 &0.01 &-10.62 & 3.61e+2& 
FUV + 24\um&H$_2$only&1.87 &0.30 &-9.14 &5.72e+2&1.33 &0.29 &1.38 &0.12 &-11.65 & 5.80e+2& 
FUV + 24\um&HI only&1.22 &0.03 &-9.22 &2.75e+2&0.79 &0.02 &0.62 &0.01 &-10.73 & 3.82e+2&
FUV + 24\um&Total gas&0.81 &1.59 &-10.06 &3.80e+2&0.42 &0.01 &0.52 &0.02 &-10.68 &3.53e+2 & 
\hline
\end{tabular}
\end{table*}




%----------------------------------------------------------------------------------------
%
%----------------------------------------------------------------------------------------
\section{Discussion}
%we should discuss results first
%----------------------------------------------------------------------------------------
%SUMMERY
%----------------------------------------------------------------------------------------
\section{Summery}
%----------------------------------------------------------------------------------------
%BIBLOGRAPHY
%----------------------------------------------------------------------------------------
\bibliography{ref.bib}

\end{document}
