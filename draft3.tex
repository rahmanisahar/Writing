\documentclass[useAMS,usenatbib]{mn2e}
%\usepackage{geometry}[left=1.5in, right=1in, bottom=1in, top=1in]
\usepackage{hyperref}
\usepackage{graphicx}
\usepackage{natbib}
\bibliographystyle{aas}
\usepackage{amsmath}
\usepackage{times}
\usepackage{float}
\usepackage{caption}
%\usepackage[caption=false]{subfig}
\usepackage{subcaption}
\usepackage{multirow}

\newcommand \kpc        {\,{\rm kpc}}
\newcommand \sigmagas    {$\Sigma_{\rm \bld {gas }}$\ }
\newcommand \eqsigmagas    {\Sigma_{\rm \bld {gas }}}
\newcommand \sigmasfr     {$\Sigma_{\rm \bld {SFR }}$\ }
\newcommand \eqsigmasfr     {\Sigma_{\rm \bld {SFR }}}
\newcommand \sigmastar    {$\Sigma_{\rm \bld {star }}$\ }
\newcommand \eqsigmastar    {\Sigma_{\rm \bld {star }}}
\newcommand \halpha    {H$\alpha $\ }
\newcommand \um    {$\mu$m\ }
\newcommand \mice {$\mu$m}
\newcommand \nprime {N$^\prime$}
\newcommand \eqnprime {N^\prime}
\newcommand \Spitzer {{\it Spitzer}}
\newcommand \Galex {GALEX}
\newcommand \Herschel {{\it Herschel}}
\begin{document}
% TITLE
\title[WHICH STAR FORMATION LAW IN M31]{RELATION BETWEEN STAR FORMATION RATE, GAS MASS AND STELLAR MASS IN THE ANDROMEDA GALAXY}
\author[S. Rahmani $\&$ P. Barmby]{S.~Rahmani, P.~Barmby\\
Department of Physics $\&$ Astronomy, University of Western Ontario, London, ON N6A 3K7, Canada}
\maketitle

%----------------------------------------------------------------------------------------
%----------------------------------------------------------------------------------------
\begin{abstract} % it is just my GESF2014 abstract, just for looks I will change it at the end
 We investigate the relation between star formation rate (SFR), gas surface density (\sigmagas) and stellar surface density (\sigmastar) in the Andromeda galaxy (M31) using the Kennicutt-Schmidt (K-S) law and the extended Schmidt law. Shi et al. argued that the surface density of SFR is related not only to gas mass, as in the empirical K-S law,  but also to stellar mass surface density, in the ``extended Schmidt law''. Using all a combination of \halpha and 24 \um emission, a combination of FUV and 24 \um, and the total infrared emission, we estimate the total SFR in M31 to be between 0.35 M$_{\odot}$yr$^{-1}$ - 0.4 M$_{\odot}$yr$^{-1}$. We use maps of $^{12}$CO and 21cm emission to produce the \sigmagas map.% For regions that are not covered in the $^{12}$CO map, we used the dust map (obtained from $SPIRE$ data) as a tracer of total gas.
 The stellar mass surface density  map is calculated using mid-infrared IRAC imaging. Our preliminary results from testing the K-S law in M31 find a power-law index of  $N = 0.62 \pm 0.01$, at the low end of values compared with previous work. We determine additional K-S law fits using each SFR map independently with molecular hydrogen, atomic hydrogen and total gas maps. We then repeat the analysis considering the stellar mass density in order to test whether the regular or extended Schmidt laws are more appropriate for M31.

\end{abstract}
\begin{keywords}
Galaxies: Andromeda, Star Formation
\end{keywords}
%----------------------------------------------------------------------------------------
%INTRODUCTION
%----------------------------------------------------------------------------------------


\section{Introduction}
\label{sec:intro} %I can add a lot more about calculating star formation and in general why star formation is important but I don't like make my introduction too long.
%P1: Star formation laws and why they are important (introducing K-S law and E-S law)
%PB: agreed with not making it too long, but do start with a sentence about why SF is important.

Empirical star formation laws (relations) has been investigated for more than 50 years, and results never been precise. Formation of stars is a complicated process. Various phenomena affect the collapse of molecular clouds and star formation which, include the environment of the star-forming region, chemical compositions, the initial mass distribution of gas, gas accretion and cooling, H$_2$ formation, and etc.

The first idea about scaling the star formation rate (SFR) was proposed as a connection between the total star formation rate and the mass of interstellar gas. \cite{Schmidt59} assumed a constant initial mass function (IMF) of stars, $\Psi (M)$, a stellar lifetime function {$T(M$)}, and a  {$SFR(T)$} and derived the star formation rate over the history of the Milky Way. He scaled SFR with a power {\it N} of the gas mass, $M_{gas}$. %Then $SFR(T)\Sigma_{M}(M) = c[M_{gas}(t)]^n$, for a summation over all stellar type $M$.
A decade later, \cite{Kennicutt98a} examined another group of galaxies. By using H$\alpha$, HI, and CO observations of 61 normal spiral galaxies and 36 starburst galaxies, he showed that the disk-average SFRs and gas densities for these samples are well represented by the Schmidt law with the power of $1.4 \pm 0.15$. Using this power index, the Schmidt law can be rewritten as:

\begin{equation}
\label{equ:ks_org}
\eqsigmasfr \propto \eqsigmagas^{1.4\pm0.15}
\end{equation}
which is often referred to as the Kennicutt-Schmidt (K-S) law, where \sigmasfr is surface density of star formation, and \sigmagas is the surface density of gas in a galaxy.


In more recent studies, \cite{Kennicutt08} investigated the local star formation law in M51 with 0.5-2 kpc resolution. For calculation of the SFR he used pa${\alpha}$ and $24\mu$m + H${\alpha}$ lines and for gas density he used constant conversion factor for CO to H$_2$. He found a correlation, from the radial variation of both SFR and gas surface density, with an index of $1.56 \pm 0.04$.
%He could not find any correlation between SFR and HI only clouds, but the correlation with the molecular cloud was about the same as the total gas.
It should be noted that the KS law can not be applied for the whole range of gas densities. In parallel studies, the star formation threshold was introduced and it was pointed out that calculated SFR is much lower than the value predicted by the KS law at low gas densities $( < 1-10\sim$ M$_{\odot}$ pc$^{-2})$ e.g. in regions far outside the optical disk \citep[e.g.,][]{Martin01, Bigiel08}.

\cite{Hunter98} have shown that, in low surface brightness galaxies the SFR density is only related to the stellar mass density. Also similar relations between stellar masses and SFR densities are seen within galaxies by \cite{Ryder94}, \cite{Hunter04} and recently by \cite{Leroy08} for specific galaxy types or limited density ranges. \cite{Shi11} found a tight relationship between stellar mass surface density, SFR and gas surface density by studying on a large sample of galaxies. They showed this relation as a power law relation and referred to it as the extended Schmidt law:

\begin{equation}
\label{equ:es_org}
\eqsigmasfr \propto \eqsigmagas^{\eqnprime} \eqsigmastar^{\beta}
\end{equation}
where \sigmasfr is SFR surface density, \sigmagas and \sigmastar are the gas mass and the stellar mass surface densities. \nprime and $\beta$ are the power law indexes. Testing new relation on 60 galaxies \nprime and $\beta$ were calculated to be $\sim$ 1.4 and $\sim$ 0.6 respectively. They showed that this relation not only predicts SFR as well as the KS law for galaxies and spatially resolved regions ($\sim$1 kpc sizes) where the KS law works, but also is acceptable for low surface brightness galaxies and regions where the KS law fails.

%P2: SFR, measuring, scaling, ...
For testing any of these star formation laws, first thing one should do is calculating the SFR. Many groups are working to find a translation from the light of star forming regions in galaxies to a rate of the formation of new stars. \cite{Kennicutt98b} calibrated luminosity of galaxies as a way of measuring the SFR in specific wavelengths using relations between the SFR per unit mass or per unit luminosity and the integrated colour of the system provided by synthesis models  \citep[e.g.,][]{Bruzual93}. Afterward many other studies tried to find the similar calibration for measuring the SFR of galaxies in other bands \citep[e.g.,][]{Kennicutt12, Calzetti12, Zhu08, Kennicutt09, Boquien10, Boquien11, Hao11}. \cite{Kennicutt09} and \cite{Hao11} introduced new SFR calibration using a combination of \halpha or FUV and far infrared (FIR) emission, respectively. In addition of using suitable calibrations for each region, considering differences between global case and local ($\sim 0.5-1$~\kpc) case is important. In $\S3$, we are talking about the SFR in the Andromeda galaxy and how we determined it.


%P3: Gas cloud, variety, observing etc
Calculating surface density of gas in galaxies is the other most important ingredients of testing star formation laws. Considering the effect of the interstellar medium (ISM) on studying the star formation law is important due to the fact that stars are born from the gas also release their material into the ISM when they reach the end of their evolution. A map of the total gas in the ISM is produced by direct observations of the gas or using interstellar dust as a tracer. Neutral and molecular hydrogen are the most common elements in the ISM. Therefore, for producing the map of the total gas in galaxies, maps of these two components can be added together and multiplied by a constant factor to account for heavier elements which are mostly He. Another way to do the mapping of total gas mass of the galaxy is to assum that the ratio of total gas mass and dust mass is the same across a galaxy, and convert dust observations to the map of total gas. $\S4$ is about the calculation of the gas surface density for the Andromeda galaxy.


%PB: shorten or leave out dynamical mass, since you don't discuss it further
%P4: Stellar mass and how to measure it
Additionally, for testing the Extended Schmidt law, measuring the stellar mass density is necessary. Measuring the mass of the stellar population is indirect and subject to significant uncertainties.% In principle there are two ways to measure stellar mass in galaxies. First is measuring the dynamical mass of galaxies using kinematics \citep{Cappellari06} or lensing \citep{Auger09}, and then modelling the mass of the dark matter and subtracting it from the measured mass. This method has been successful to predict and subtract the mass of the dark matter, which is the dominant mass in galaxies \citep{Zaritsky94},  and can easily cause uncertainties as high as the order of the measurement.

One of the methods to calculate stellar mass within a galaxy, is based on stellar population models \citep[e.g.][]{ Bruzual93, Kotulla09} and using them to connect stellar mass to an observable, like luminosity in different wavelengths, colours, spectral energy distribution from spectroscopy or multi-band observations. Having two independent methods to measure stellar mass helps to compare the results and check whether there are any systematic differences. Comparing the results from measuring stellar mass in stellar clusters shows there is no huge difference. Comparing the results from measuring stellar mass within galaxies shows that the differences are on the order of a few x10 percent, so one should be more careful to model subtleties \citep{McLaughlin05}. Calculating stellar mass in the Andromeda galaxy is described in  $\S5$ .

%P5(2*): What we did and why Andromeda and
In this paper we present our results from testing and comparing of both the K-S law and the Extended Schmidt law on the Andromeda galaxy (M31). %We used three different gas maps and SFR maps, to identify if any of these laws work better with specific gas mass or SFR tracer. 
Additionally, We applied these laws on three different regions on the M31, to determine whether there is any dependence to a distance from the centre of galaxy or not. Since, M31 is the closest spiral galaxy to the Milky Way, therefore, we can have higher resolution images of it than we have of any other spiral galaxy. Having high resolution images provides us data from different regions within
the galaxy with different physical situations (e.g. metallicity, surface brightness and etc.). All these parameters make the Andromeda a suitable test bed for studying the scaling law.
%PB: Give a little more detail in the followin sentence

Furthermore, the range of the power index of K-S law calculated for the Andromeda galaxy is between 0.5 - 2 \citep[e.g.,][]{Tabatabaei10,Ford13}. %Different groups, using different SFR or gas mass tracers, applied the K-S law on the M31 and have very different power indexes. 
In the most recent work on the K-S law in M31, \cite{Ford13}, test this law on six regions from centre of the galaxy using three different ISM maps (H2 only, total gas calculated from H2 plus HI maps, and total gas calculated from dust emission). The power indexes they measure for each map and each regions vary between 0.6 to 2.03. The origin of these contrary results is still ongoing question. It is not obvious what makes these difference in the power law indexes calculated by different group. Whether it depends to distance from galaxy, metalicity, gas tracers, SFR tracers, fitting methods, or this is because of the K-S law does not work on the M31 at all. If we assume the last case is valid, testing extended Schmidt law on the Andromeda galaxy would help to solve the problem with these controversial results.


%----------------------------------------------------------------------------------------
%DATA
%----------------------------------------------------------------------------------------
\section{Data}
\label{sec:data}
Since the M31 is the closest spiral galaxy to the Milky Way, it was aimed by many probes and telescopes, both space based and ground based ones. Therefore, variety of data from X-ray to radio wavelengths are available. Wide ranges of data provides us the ability to measure all the required parameters to test the star formation laws in M31. Table~\ref{table:data} lists all the data we used in this paper. Each data comes with different Full Width Half Maximum (FWHM), spacial resolution and grids. In order to make sure that all our data contains the same amount of information, we smoothed them using Aniano's Kernels' library and convolution code \citep{Aniano12}, to highest resolution available. Moreover, considering Nyquist relation \citep{Nyquist}, we re-sampled and re-gridded all our maps to have a proper pixel size.

\begin{table}
\caption{Data used in this study.}
\label{table:data}
\begin{tabular}{@{}lccc}
\hline\hline
Wavelength & FWHM & Telescope
& Ref. \\
\hline
%500 \um & 36\arcsec & $Herschel$ & \cite{Fritz12} \\
%350 \um & 24\arcsec.5 & $Herschel$ & \cite{Fritz12} \\
%250 \um & 18\arcsec.2 & $Herschel$ & \cite{Fritz12} \\
2250~\AA & 6\arcsec & GALEX & \cite{Martin05}\\ %:PB: GALEX, KPNO, IRAM should not be in $$
1550~\AA & 4\arcsec.5 & GALEX & \cite{Martin05}\\ %PB (they are abbreviations)
6570~\AA  & 1\arcsec & KPNO& \cite{Massey07}\\
3.6~\um & 1\arcsec.7 & \Spitzer & \cite{Barmby06} \\ %PB: write {\it Spitzer}, {\it Herschel}
4.5~\um & 1\arcsec.7 & \Spitzer & \cite{Barmby06} \\ %PB because those are both missions and names
8~\um & 1\arcsec.9 & \Spitzer & \cite{Barmby06} \\ % PB: of people - italics means the mission, not
24~\um & 6\arcsec & \Spitzer & \cite{Gordon06} \\ %PB: the person
70~\um & 18\arcsec & \Spitzer & \cite{Gordon06} \\
160~\um & 12\arcsec & \Herschel & \cite{Fritz12} \\
2.6~mm & 23\arcsec & IRAM 30-m & \cite{Nieten06}\\
21~m & 60\arcsec \times \ 90\arcsec & DRAO & \cite{Chemin09}\\
\hline
\end{tabular}
\end{table}

%PB: in general it's good to write number~unit, ie 2750~\AA 
%PB: this means that number and unit don't end up on separate lines
\subsection{Ultra Violet (UV) emission}

One of the calibrations, which was used to calculate SFR, uses FUV emission as a tracer of the star forming regions. FUV emission of M31 was observed by Galaxy Evolution Explorer ($GALEX$; \citep{Martin05}). $GALEX$ telescope data contains information for two bands, FUV:1350-1750 and NUV:1750-2750~\AA, with FWHM of 4\arcsec .5, and 6 \arcsec.

\subsection{Visible light}
Our other method to calculate the SFR was using a combination of \halpha emission and 24\um emission. \cite{Massey06, Massey07} mapped UBVRI and \halpha emission line of M31 as part of the Nearby Galaxies Survey using KPNO telescope. \halpha maps, which were obtained from NOAO science archive are available in 10 overlapping fields. We created a Mosaic image of the maps using Montage \citep{Montage}. Assuming that all stars have the same fraction of \halpha light in their spectra, we subtracted the R band emission from the \halpha using averaged scaling factor between R band images and \halpha provided by \cite{Azimlu11}. We also took masked map of stars from \cite{Azimlu11} and masked all point sources (Figure~\ref{fig:halpha}). Details of making the making \halpha map can be find in the Appendix ~\ref{appendix: halpha}.

\subsection{IR and sub mm Data}
In IR and sub mm bands both $Spitzer$ \citep{Wener04} and $Herschel$ \citep{Pilbratt10}  space telescopes observed the M31. The Infrared Array Camera (IRAC; \citep{Fazio04}) on the $Spitzer$ is a camera observing at four channels (3.6, 4.5, 5.8, and 8 \um). IRAC observation of M31 were obtained by \cite{Barmby06}, which covered $3\degr.7 \times 1\degr.6$. IRAC 3.6 and 4.5\um bands  were used to calculate the stellar mass. Multiband Imaging Photometer for Spitzer (MIPS) observed the  M31 at 24, 60, and 160\um and covered region $\sim 1\degr \times 3\degr$ \citep{Gordon06}.

Calculation of the SFR using TIR emission needs maps of the m31 in 8, 24, 70, and 160 \um. IRAC 8, MIPS 24, MIPS 70 bands were used to calculate TIR emission, but due to better quality and resolution of $Herschel$s$'$data, PACS (Photodetector Array Camera and Spectrometer) \citep{Poglitch10} were used. PACS 160\um observation of the Andromeda galaxy alongside of PACS 100 \um band and three (250 \um, 350\um and 500 \um) SPIRE (Spectral and photometric Imaging Receiver) \citep{Griffin10} bands are covered $\sim 5\degr.5 \times 2\degr.5$ regions \citep{Fritz12}.

\subsection{Radio Emission}
Emission from $^{12}$CO(J:$1\rightarrow0$) line which, was observed on-the-fly with the IRAM 30-m telescope\citep{Nieten06} was used to calculate the H$_2$ column density of interstellar medium. This map coveres $2\degr \times 0\degr.5$ that has best resolution available (FWHM = $23\arcsec$) but smallest coverage of halo of the galaxy among our maps. In this project we used data of the 21 cm emission from the atomic gas (HI) in the M31 from \cite{chemin9}. They used Synthesis telescope and the 26 m antenna at the Dominion Radio Astrophysical Observatory to map HI emission in 21cm band of the M31.


\begin{figure*}
 {\vfil  \includegraphics[width=164mm]{halpha_copy.eps}
  \caption{Mosaic created using Montage programme from 6 fields of \halpha emission maps of m31 from \citpe{Massey07}. Result from Montage was continuum subtracted and masked out all point sources. Centre of galaxy was masked out due to saturation of data in R-band image.\label{fig:halpha}}
 \vfil}
\end{figure*}


%----------------------------------------------------------------------------------------
%STAR FORMATION RATE
%----------------------------------------------------------------------------------------
%----------------------------------------------------------------------------------------
\section{Star Formation Rate}
\subsection{FUV plus 24\um Star Formation Rate}

In this project first star formation map was produced using a calibration of a combination of the FUV emission and 24\um emission introduced by \cite{Hao11}. Since peak of the emission of massive and young stars (O - B and A type) are in the UV, this wavelength is one of the most common used star formation rate indicators \citep[e.g.,][]{Kennicutt89}. The UV indicates star forming region with age of $\sim 100$ Myr \citep[e.g.,][]{Kennicutt98, Calzetti05}. However, downside of only using the FUV light as a tracer of star forming region is that this emission is really sensitive to dust extinction. On the other hand 24\um emission is dominated by emission from dust heated by UV photons from young and hot stars. This emission is sensitive to star formation time scale of $\la10$ Myr \citep{Calzetti07}.
The FUV band of \Galex and the 24\um band of \Spitzer ~ data was used to measure the SFR calibrated by \cite{Hao11} as follow:
\begin{equation}
\label{equ: fuvplus24}
SFR =4.46\times10^{-44}[L(FUV)_{obs}+3.89L(24\mu m)]
\end{equation}
where unit of SFR is M$_{\odot}$yr$^{-1}$, and L(24$\mu$m) and L(FUV)$_{obs}$ are in erg/s. L(FUV)$_{obs}$ is the luminosity of galaxy in the FUV emission. The index of obs is shown to indicate that the FUV emission is not corrected for effect of extinction. Nevertheless, this emission is corrected for the foreground stars emission. To do so, we followed a method introduced by \cite{Leroy08}. We assumed for any pixel in FUV$_{obs}$ map, %which has a pixel size equals the pixel size of NUV map,
if I$_{NUV}$/I$_{FUV}$ $>$ 15 that pixel value is dominated by foreground emission and masked them out. We also masked the same region in the galaxy for 24\um map. Using this method, we calculated the total SFR is 0.31\pm 0.04 M$_{\odot}$yr$^{-1}$.

%----------------------------------------------------------------------------------------
\subsection{\halpha plus 24\um Star Formation Rate}
\label{sec:sfr_halpha}

The SFR can be calculated using combination of \halpha and \Spitzer  24\um maps. \halpha emission of galaxies is dominated by light emitted from young and massive stars. Star formation time scaled traced by this emission is $\sim 6-10$ Myr \citep[e.g.,][]{Kennicutt09, Calzetti12}. Therefore \halpha alone could be used to calibrate the SFR \citep[e.g.,][]{Osterbrock06, Kennicutt09}. However, \halpha emission is very sensitive to dust extinction,too. Therefore, combination of the \halpha and 24\um of is considerable choice for calculating the SFR. we used calibration introduced by \cite{Kennicutt09} to measure the SFR:
\begin{equation}
\label{equ: halphaplus24_g}
SFR = 5.5 \times 10^{-42}[L(H{\alpha})_{obs} + 0.020L(24\mu m)]
\end{equation}
where L(H${\alpha}$)$_{obs}$ is the observed H${\alpha}$ luminosity without correction for internal dust attenuation, given in the unit of erg/s. L(24$\mu$m) is the $24\mu$m IR luminosity in erg/s, and SFR is in M$_{\odot}$yr$^{-1}$. It was indicated that the formulation above can only be used at global situation (i.e. total SFR).

\cite{Calzetti07} introduced a new calibration using the same bands in case of local regions (figure ~\ref{fig:sfr_halpha}):
\begin{equation}
\label{equ: halphaplus24_l}
SFR = 5.5 \times 10^{-42}[L(H{\alpha})_{obs} + 0.033L(24\mu m)]
\end{equation}
units here are the same as equation~\ref{equ: halphaplus24_g}. Equation~\ref{equ: halphaplus24_l} is useful for regions less than $\sim 1$ kpc. According to  \cite{Calzetti07} calibration constant in equations ~\ref{equ: halphaplus24_g} and ~\ref{equ: halphaplus24_l} can be different by changing an IMF. Changes in the IMF mostly affect on calibration based on \halpha luminosity, mostly because significant amount of the ionising photons comes from the stars more massive than $\sim 20$M$\sun$. Changing the calibration constant would change the total SFR by a factor of 1.5. In this paper we used \cite{Kroupa01} IMF with upper mass limit of 100M$\sun$. Total SFR calculated from equation~\ref{equ: halphaplus24_g} is 0.35 \pm 0.01 M$\sun$yr$^{-1}$, which is in good agreement with previous studies.
%PB: can you put uncertainties on the total SFR numbers?

\subsection{Total Infrared Star Formation Rate}
\label{sec:sfr_fir}
Total infrared emission of a system, can also be used as a tracer of the SFR. Dust absorbs radiation from hot and young stars and re-emitt it in infrared wavelength, however there is no one-to-one mapping between IR and UV emission \citep{Calzeti12}, therefore, integrating over full wavelength range of the IR part and calculating the TIR luminosity is better tracer of the SFR instead of the IR single-band emission. \cite{Calzetti07} assumed that stellar bolometric emission is completely absorbed and re-emitted by dust, i.e., $L_{star}(bol) = L(TIR)$ and calibrated the TIR luminosity of a system to calculate the SFR for a stellar population undergoing constant star formation over $\tau = 100$ Myr:

\begin{equation}
\label{equ:sfr_fir}
SFR(\rm TIR) = 2.8 \times10^{-44}L(\rm TIR)
\end{equation}
where SFR(TIR) and L(TIR), star formation rate calculated from TIR emission and TIR luminosity are in M$_{\odot}$yr$^{-1}$ and erg/s respectively.


IR luminosity can be measured from integration of IR part of a spectral energy distribution (SED) of a galaxy or combining photometry datas in different IR wavelengths. For second approach \cite{Draine07} modelled \Spitzer data and calibrated IR single-band photometry data to calculate the TIR luminosity. \cite{Boquien10} tested and modified this calibration as:
\begin{equation}
 \label{equ: TIR}
L(\rm TIR) = 0.95L(8) + 1.15L(24) + L(70) + L(160)
\end{equation}
where L(8 \um),L(24 \um), L(70 \um) and L(160\um) are luminosities of galaxy in 8 \um, 24 \um, 70 \um and 160 \um in unit of erg/s. We used equation ~\ref{equ:sfr_fir} to calculate second map of SFR of M31 (figure ~\ref{fig:sfr_fir_small}). Using this method we calculated the total SFR 0.40 \pm 0.04 M$_{\odot}$yr$^{-1}$. Table ~\ref{table:sfr} compares total SFR from literature and current work.

\begin{table*}
\begin{minipage}{100mm}
\caption{Comparison of Total Star Formation Rate of M31}
\label{table:sfr}
\begin{tabular}{@{}lcc}
\hline\hline
Ref.&Method&SFR(M$_{\odot}$yr$^{-1}$) \\
\hline
Current work&FUV and 24\um&0.31\\
Current work&\halpha and 24\um&0.35\\
Current work&TIR luminosity&0.4\\
\cite{Ford13}&FUV and 24\um&0.25\\
\cite{Ford13}&TIR luminosity&0.48-0.52\\
\cite{Azimlu11}& \halpha and 24\um&0.34\\
\cite{Azimlu11}&Extinction corrected \halpha&0.44\\
\cite{Tabatabaei10}&Extinction corrected \halpha&0.27-0.38\\
\cite{Barmby06}&Infrared 8\um luminosity& 0.4\\
\hline
\end{tabular}
\end{minipage}
\end{table*}


%PB: I like this figure! Looks nice.

\begin{figure*}
    \centering
    \begin{subfigure}[b]{1\textwidth}
        \centering
        \includegraphics[width=164mm]{sfr_fuv.eps}
        \caption{SFR,FUV+24\um}
        \label{fig:sfr,fuv}
    \end{subfigure}
    \hfill
    \begin{subfigure}[b]{1\textwidth}
        \centering
        \includegraphics[width=164mm]{sfr_halpha.eps}
        \caption{SFR,H$_{\alpha}$}
        \label{fig:sfr_halpha}
    \end{subfigure}
    \hfill
    \begin{subfigure}[b]{1\textwidth}
        \centering
        \includegraphics[width=164mm]{sfr_fir.eps}
        \caption{SFR,TIR}
        \label{fig:sfr,fir}
    \end{subfigure}
    \caption{SFR map from a combination of FUV + 24\um emission (top), \halpha and 24\um emission (middle), and total infrared emission (bottom)}
    \label{fig:sfrs}
\end{figure*}
%----------------------------------------------------------------------------------------
%ISM
%----------------------------------------------------------------------------------------
\section{Interstellar Medium}
%----------------------------------------------------------------------------------------
%Direct Measurement of the Gas Mass
%----------------------------------------------------------------------------------------
%\subsubsection{Direct Measurement of the Gas Mass}

%The surface density of gas in the ISM of galaxies can be measured by direct observation of the neutral and molecular hydrogen. This method can be promising provided that the spatial resolution of observational data is high enough; nonetheless, due to technical limitations, attaining enough spatial resolutions to resolve clouds is almost impossible. This problem shows itself for distant galaxies more than nearby galaxies. As a result, using this method is limited and has a lot of uncertainties.

%\subsubsection*{Molecular Clouds}
%PB: these two paragraphs contain a lot of background information that could either be moved to the
%PB introduction or be omitted; this info doesn't belong here.
 The molecular form of hydrogen in the ISM has no permanent electric dipole moment, which makes it really hard to detect. H$_2$ molecules are located in cool and dense molecular clouds, but fortunately they are not the only component of the molecular gas. The second dominant component, helium, is a mono-atomic gas and has the same problem as hydrogen, but the molecular gas also contains heavier elements such as carbon and oxygen which are combined to form CO \citep{Bolato13}.

CO has a weak permanent electric dipole moment and a ground rotational transition with low excitation energy. Given its low energy and critical density, CO can easily be excited even in cold molecular clouds. Hence CO (usually the $J(1\rightarrow 0)$ rotational transition, observed at 2.6 mm) is used as a tracer of the mass of the molecular cloud dominated by molecular hydrogen \citep[see, for example,][] {Sanders84}. Higher rotational transition of CO can be used as a tracer as well, but they are not as common as the $J(1\rightarrow 0)$. The relation between CO emission and H$_2$ cloud mass is shown as

\begin{equation}
\label{equ:conversion}
\rm N_{H_2}/\rm cm^{-2} = X_{CO} \times I_{CO}/\rm K km s^{-1}
\end{equation}
where X$_{CO}$ is the conversion factor (also known as the X-factor). X-factor could be different in regions within a galaxy due to difference in metallicity. Though assuming constant conversion factor for cases such as M82 and M31 can lead to global molecular gas mass estimates that are very accurate,  but in regions with low metallicity there are uncertainties. Different groups are working on the various galaxies to measure the X-factor for each \citep{Wilson95, Bosselli02, Bolato13}.. In case of M31, different values of X-factor ($1-5.6 \times 10^{20}$) were adopted in the previous studies \citep[e.g.][]{Ford13, Bolato13, Leroy11, Bolato08, Nieten06, Sofue94, Strong88}; however, since any constant differences in X-factor leads to horizontal changes in a plot of log(SFR) versus Log($\Sigma_{\rm {gas}}$), and does not affect our K-S law power index for molecular gas,  we chose $X_{\rm {CO}}= 2 \times 10^{20}$  the same as many other works \citep[e.g.][]{Ford13, Smith12}. This assumption might have some effect on the total gas mass but considering the fact that HI mass dominates H$_2$ mass in M31 this effect is small.
The total gas mass was calculated from:
\begin{equation}
\label{equ:total_gas}
\rm M_{\rm \bld {total \, gas}} = 1.36[M_{HI}+M_{H_2}]
\end{equation}
where  factor of 1.36 is a constant to consider He and the other heavier elements effect on mass, M$_{HI}$ is mass of neutral hydrogen obtained from 21cm observations and M$_{H_2}$ is from equation ~\ref{equ:conversion} in unit of M$\sun$.

%----------------------------------------------------------------------------------------
%STELLAR MASS
%----------------------------------------------------------------------------------------
\section{Stellar Mass}
\label{starmass}
Stellar mass can be measured using a suitable calibration between flux/luminosity of stars in specific wavelengths and stellar mass. The basic idea behind these kind of calibrations comes from the fact that one can use the star formation histories recovered from synthesizing the stellar optical colour magnitudes in different regions in the galaxies. The calculated stellar mass can be used  to calibrate the fluxes from the stars as the easily measurable parameters of stars. Using NIR bands, different groups tried to make a map of stellar mass distributions within galaxies \citep[e.g.,][]{Elmgreen84}.%  Using $B-V$ colours \citep{Bell01} and $Spitzer$ 3.6- and 4.5-\um, M/L correction, \cite{Kendall08} produced stellar mass surface densities of M81. %PB: do you mean M81 here? If so, it's not quite clear why you're mentioning a different galaxy.

We calculated the stellar mass within the M31 using calibration of the 3.6\um flux by \cite{Eskew12}.  Since 3.6\um flux is almost insensitive to the emission from young stellar population, dust absorption and emission, this band-pass was used for the calibration of the stellar mass. By using data from the large Magellanic cloud, they find an empirical relation between stellar mass and luminosity of stars as following:

\begin{equation}
\label{equ:eskew}
M _{\star}= 10^{5.97} F_{3.6}(\frac{D}{0.05})^2
\end{equation}

where M$_{\star}$ is the stellar mass in mass of The Sun, $F_{3.6}$ is the 3.6\um flux in $Jy$ , and D in the distance of the galaxy in Mpc. Figure ~\ref{fig:stellarmass} shows the stellar mass map for M31 using IRAC 3.6\um. Using this method we calculated the total stellar mass of galaxy is $6.9 \times 10^{10}$M$_{\odot}$ with 6$\%$ uncertainty. This result is in fair agreement with the result from \cite{Tamm2012}. They calculated that stellar mass in the M31 is $(10-15) \times 10^{10}$M$_{\odot}$, 56$\%$ which is in the disk and the rest is in the bulge of the galaxy.   %PB: need to compare with other results, as you did with SFR. 
\begin{figure*}
\centering
\includegraphics[width=164mm]{star.eps}
\caption{Stellar Mass surface density. This mass is produced using $IRAC$ 3.6 $\mu$ m data and its calibration presented in equation ~\ref{equ:eskew}}
\label{fig:stellarmass}
\end{figure*}


%----------------------------------------------------------------------------------------
%SFR LAWS
%----------------------------------------------------------------------------------------
\section{Scaling SFR}
\subsection{Fitting method}
\label{sec:fittinhg}
%For investigating and comparing the SFR laws in M31
%Given available
Producing the SFR, gas mass and stellar mass maps, provides of with enough data to examine and compare the K-S law and the extended Schmidt law.% are possible. 
We investigated the K-S law and the extended Schmidt law by using  pixel by pixel method.  Our final results are divided by two sets of different pixel sizes. That's because the H2 gass mass map has better resolution and lower pixel size of the  HI gas mass map. Consequently, any plots with surface density of H2 as a tracer of the gas mass, have more data points.%in whole galaxy as well as comparing the laws in three different regions of galaxy based on their distance from the centre of the galaxy. 
We can re-write equations ~\ref{equ:ks_org} and ~\ref{equ:es_org} in a logarithmic scale. Hence, we would have two linear equations, as following:

\begin{figure*}
\centering
\includegraphics[width=164mm]{ks_all_1.png}
\caption{The result from fitting the Kennicutt-Schmidt law on data from whole galaxy using pixel by pixel method. Plots have a different pixel size. Each point in the plots with the surface density of H2 as a tracer of gas mass represents regions $~$ Kpc and points in plots with surface density of HI or total gas mass, represeents region in $~$Kpc.} 
\label{fig:ks,all}
\end{figure*}


\begin{subequations}
\begin{align}
\left%\{ \begin{array}{l l}
\label{eq:sfr_law_log}
\log_{10} \eqsigmasfr = N~\log_{10} \eqsigmagas + A, \\
\log_{10} \eqsigmasfr = \eqnprime~\log_{10} \eqsigmagas + \beta~\log_{10}\eqsigmastar  + A^\prime ~~~,
\end{align}
\end{subequations}
where, N, $\eqnprime$, $\beta$, A, and A$^\prime$ are free parameters of fittings. Units in equation above, are the same as the units in the equations ~\ref{equ:ks_org} and ~\ref{equ:es_org}.

We found the free parameters by applying the hierarchical Bayesian linear regression method as described in \cite{Shetty13}. Shetty and his colleagues used a Bayesian linear regression approached to develop a new method to find the K-S law parameters, considering the measurements uncertainties as well as hierarchical data structure. For the K-S law's fitting parameters, we used a 'R' code provided by Shetty's group. Uncertainties on each observable quantity was measured considering the background noise photometry errors and calibration uncertainties.%PB: are the background noise errors and calib uncertainties discussed elsewhere in this paper (in which case reference the section) or do they need to be discussed here. %SR no they didn't, should I discusse more? 
%SR usually nobody write about them.
%PB I think you should discuss the uncertainties somewhere -- presumably they affect your fitting results, so the reader needs to know where the uncertainties come from and their approximate size.
 
For the extended Schmidt law, we expended the code in the way that instead of using simple linear regression, which is the case of the K-S law, it uses multiple linear regression. In this case, we were able to examine the effect of the stellar mass in the SFR as shown in equation ~\ref{eq:sfr_law_log}.

We also test each laws on three different regions. The regions were chosen the same as regions introduced in \cite{Drain14}. They found that across the galaxy, with increasing distance from the centre of galaxy dust/H ratio declines monotonically. Considering that dust/H ratio has a direct relation to metallicity, they showed that the metallicity of the M31 changes in the same manner.
%PB: need to expand this a bit: it's well-known that dust-to-gas and metallicity both go down as distance to galaxy centre increase, but how would you expect this to affect  the SFR/gas/stellar mass relation?

\begin{equation}
\label{eq:dust2gas_vs_R}
\frac{M_d}{M_H} \approx
\left\{ \begin{array}{l l}
0.0280 \exp(-R/8.4\kpc)  & R< 8\kpc\\
0.0165 \exp(-R/19\kpc)   & 8\kpc < R < 18\kpc\\
0.0605 \exp(-R/8\kpc)    & 18\kpc < R \la 25\kpc ~~~,\\
\end{array}
\right.
\end{equation}

Applying the SFR laws in these regions, provides a tool to consider the effect of the distance from the centre of galaxy on SFR laws as well as the effect of the metallicity on those. Tables ~\ref{table:sfr_law_all} through ~\ref{table:sfr_law_25} show fitting parameters and their uncertainties for whole galaxy and each different regions. \halpha data dose not have a smooth background. Also centre of galaxy in this dta is diffused, so we do not have any data from centre of the M31. Since it was the only available \halpha data from the M31, We used this data anyway. However, results from the \halpha SFR map are not shown here and will be discuss in appendix ~\ref{app:halpha}. %can I make it foot note?

\subsection{Star formation laws}
In total, we produced 72 different plots which, show either the K-S law or the extended Schmidt law, both in whole galaxy and in different distances from the centre of the M31. Additionally we can use these results to see if there is any correlation  between the SFR laws and the metallicity, as it was decribed in ~\ref{sec:fittinng}.
  
%We had three SFR maps, three gas mass density maps. We applied both laws on whole galaxy on each two combination of two of them, along with applying the laws on three different regions of them. As a result we were be able to test each law 36 times. In total we produced 72 different plots which show the either K-S law or the extended Schmidt law, and helps us to compare both laws in whole galaxy and in different regions. Additionally, these plots provide us a tool to compare the differences in SFR calibrations, ISM gas tracers, and effect of metallicity on SFR.    
% PB: I think you should keep some of the text commented-out above, so that you can show where the number 72 comes from.


Figure ~\ref{fig:ks,tot} shows the pixel by pixel fitting of the K-S law on the whole galaxy. First row plots are the surface density of the SFR(TIR), which is calculated using  the TIR emission form section ~\ref{sec:sfr_fir}, versus the gas surface density traced by only the molecular gas (H$_2$),  only atomic gas (HI), and total gas from right to left, respectivly. The only difference between the upper and lower plots is that in lower row the SFR indicators is the FUV plus 24\um emission. Figure ~\ref{fig:ks,regs} is the same as figure ~\ref{fig:ks,all}. However, in this figure we separated pixels from different regions in galaxy by their colours. The regions with R less than 8$\kpc$, $8\kpc < $R $< 18\kpc$, and $18\kpc <$ R $\la 25\kpc$ are shown in red, green and blue, respectively. 


\begin{figure*}
\includegraphics[width=164mm]{c3d_all_ks_regs_1.png}
\caption{same as figure ~\ref{fig:ks,all}, but in this figure we separated pixels from different regions in galaxy by their colours. The regions with R less than 8$\kpc$, $8\kpc < $R $< 18\kpc$, and $18\kpc <$ R $\la 25\kpc$ are shown in red, green and blue, respectively.}
\label{fig:ks,regs}
\end{figure*}


We took the similar approach for the extended Schmidt law. Figures ~\ref{fig:es,all} to ~\ref{fig:es,regs,fuv,tot} show the results of the pixel by pixel fitting of this law. For the extended Schmidt law fittings, we plotted our results in 3D to illustrate the relationship between the surface density of the SFR, the gas mass surface density, and stellar mass density more clearly. In these series of plots, X-axis is the gas mass  surface density, either molecular gas (H$_2$), the atomic gas (HI) or the total gas, Y-axis is the SFR(TIR) or SFR(FUV + 24 $\mu$m), and Z-axis is the stellar mass surface density. Shadows of our data on each surface are also plotted, to have a more clear picture of correlations between components . For a comparison, one cane asily conclude that shadows on a X-Y surface is a reproduction of K-S law. Colours used in plots ~\ref{fig:es,regs,fir,h2} to ~\ref{fig:es,regs,fuv,tot} are the same as colours in  ~\ref{fig:ks,regs}.






%We did the same for the extended Schmidt law including stellar mass map to our other maps (fig. ~\ref{fig:es-law}), results are listed in table ~\ref{table:sf-laws_pix} and table ~\ref{table:sf-laws_cat} .

\begin{figure*}
    \centering
    \begin{subfigure}[b]{0.3\textwidth}
        \centering
        \includegraphics[width=\textwidth]{es_tot_fir_vs_h2.png}
        \caption{SFR,TIR vs surface density of H$_2$}
        \label{fig:es,all,fir,h2}
    \end{subfigure}
    \hfill
    \begin{subfigure}[b]{0.3\textwidth}
        \centering
        \includegraphics[width=\textwidth]{es_tot_fuv_vs_h2.png}
        \caption{SFR,FUV+24\um vs surface density of H$_2$}
        \label{fig:es,all,fuv,h2}
    \end{subfigure}
    \hfill
    \begin{subfigure}[b]{0.3\textwidth}
        \includegraphics[width=\textwidth]{es_tot_fir_vs_hi.png}
        \caption{SFR,TIR vs surface density of HI}
        \label{fig:es,all,fir,hi}
    \end{subfigure}
     \centering
    \begin{subfigure}[b]{0.3\textwidth}
        \centering
        \includegraphics[width=\textwidth]{es_tot_fuv_vs_hi.png}
        \caption{SFR,FUV+24\um vs surface density of HI}
        \label{fig:es,all,fuv,hi}
    \end{subfigure}
    \hfill
    \begin{subfigure}[b]{0.3\textwidth}
        \centering
        \includegraphics[width=\textwidth]{es_tot_fir_vs_tot.png}
        \caption{SFR,TIR vs surface density of total gas}
        \label{fig:es,all,fir,tot}
    \end{subfigure}
    \hfill
    \begin{subfigure}[b]{0.3\textwidth}
        \centering
        \includegraphics[width=\textwidth]{es_tot_fuv_vs_tot.png}
        \caption{SFR,FUV+24\um vs surface density of total gas}
        \label{fig:es,all,fuv,tot}
    \end{subfigure}
    \caption{The result from fitting the extended Schmidt law on data from whole galaxy using pixel by pixel method. Plots show SFR vs surface density of gas, and z-axis is surface density of the stellar mass. Each figure shows different combination of the SFR tracer and gas mass tracer results. Plots have a different pixel size. Each point in the plots with the surface density of H2 as a tracer of gas mass represents regions $\sim$30~Kpc and points in plots with surface density of HI or total gas mass, represents region in $\sim$155~Kpc.}
    \label{fig:es,all}
\end{figure*}



% \begin{figure*}
% \centering
% \includegraphics[width=162mm]{c3d_fir_h2.png}
% \caption{Fitting result from the SFR(TIR) vs H$_{2}$.}
% \label{fig:es,regs,fir,h2}
% \end{figure*}


% \begin{figure*}
% \centering
% \includegraphics[width=162mm]{c3d_fuv_h2.png}
% \caption{Fitting result from the SFR(FUV + 24 $\mu$m) vs H$_{2}$.}
% \label{fig:es,regs,fuv,h2}
% \end{figure*}



% \begin{figure*}
% \centering
% \includegraphics[width=162mm]{c3d_fir_hi.png}
% \caption{Fitting result from the SFR(TIR) vs HI.}
% \label{fig:es,regs,fir,hi}
% \end{figure*}


% \begin{figure*}
% \centering
% \includegraphics[width=162mm]{c3d_fuv_hi.png}
% \caption{Fitting result from the SFR(FUV + 24 $\mu$m) vs HI.}
% \label{fig:es,regs,fuv,hi}
% \end{figure*}


\begin{figure*}
\centering
\includegraphics[width=162mm]{c3d_fir_tot.png}
\caption{Fitting result from the SFR(TIR) vs total gas.}
\label{fig:es,regs,fir,tot}
\end{figure*}

\begin{figure*}
\centering
\includegraphics[width=162mm]{c3d_fuv_tot.png}
\caption{Fitting result from the SFR(FUV + 24 $\mu$m) vs total gas.}
\label{fig:es,regs,fuv,tot}
\end{figure*}




\begin{table*}
\caption{Fitting parameters of the SF laws from applying the Bayesian.}
\label{table:res}
\begin{tabular}{||cc||cccrcccccr}
\hline\hline
\multicolumn{1}{||c||}{\multirow{1}{*}{Region}} & SFR Tracer& Gas Tracer & N & $\sigma_N$ & A & \nprime & $\sigma_{\eqnprime}$ & $\beta$ & $\sigma_\beta$ & A$^\prime$& \\
\hline
\multicolumn{1}{||c||}{\multirow{6}{*}{Global case}} & TIR & H$_2$ only & 1.03 & 9.47e-3  & -8.92  & 0.95 & 9.77e-3 & 0.34 & 4.25e-3 & -9.57  \\
 & TIR         & HI only    & 1.15 & 1.47e-2 & $-9.63$  & 0.77 & 8.84e-3 & 0.75 & 4.58e-3 & $-10.71$ \\
 & TIR         & Total gas  & 0.81 & 6.27e-3 & $-9.84$  & 0.50 & 4.49e-3 & 0.55 & 4.57e-3 & $-10.53$\\
 & FUV + 24\um & H$_2$ only & 1.15 & 1.16e-2 & $-9.28$  & 1.09 & 1.21e-2 & 0.25 & 4.88e-3 & $-9.76$  \\
 & FUV + 24\um & HI only    & 1.21 & 9.13e-3 & $-9.87$  & 0.81 & 7.34e-3 & 0.60 & 4.10e-3 & $-10.70$ \\
 & FUV + 24\um & Total gas  & 0.78 & 4.67e-3 & $-10.06$ & 0.48 & 4.61e-3 & 0.45 & 4.69e-3 & $-10.57$ \\
\hline
\multicolumn{1}{||c||}{\multirow{6}{*}{R$< 8\kpc$}}& TIR & H$_2$ only & 0.96 & 1.48e-2 & -8.95  & 0.88 & 1.49e-2 & 0.30 & 6.11e-3 & -9.54  \\
 & TIR         & HI only    & 1.52 & 7.73e-2  & -9.91  & 1.00 & 7.27e-2  & 0.60 & 4.22e-2  & -10.60 \\
 & TIR         & Total gas  & 1.00 & 3.51e-2  & -10.04 & 0.63 & 3.47e-2  & 0.53 & 7.57e-2  & -10.63 \\
 & FUV + 24\um & H$_2$ only & 1.04 & 1.84e-2 & -9.29  & 1.00 & 1.85e-2 & 0.18 & 7.17e-3 & -9.64  \\
 & FUV + 24\um & HI only    & 1.28 & 4.73e-2  & -10.00 & 0.93 & 5.68e-2  & 0.42 & 3.81e-2  & -10.50 \\
 & FUV + 24\um & Total gas  & 0.88 & 2.59e-2  & -10.12 & 0.75 & 3.53e-2  & 0.16 & 3.93e-2  & -10.28 \\
\hline
\multicolumn{1}{||c||}{\multirow{6}{*}{$8\kpc < $R $< 18\kpc$}} & TIR & H$_2$ only & 1.08 & 1.26e-2 & -8.90  & 1.00 & 1.29e-2 & 0.40 & 5.74e-3 & -9.66  \\
 & TIR         & HI only    & 1.09 & 4.50e-2 & -9.54  & 0.77 & 2.52e-2  & 0.75 & 1.26e-2  & -10.71 \\
 & TIR         & Total gas  & 0.75 & 1.76e-2  & -9.81  & 0.47 & 1.18e-2  & 0.55 & 1.24e-2  & -10.53 \\
 & FUV + 24\um & H$_2$ only & 1.21 & 1.53e-2 & -9.27  & 1.15 & 1.57e-2 & 0.33 & 6.57e-3 & -9.89  \\
 & FUV + 24\um & HI only    & 1.18 & 2.89e-2 & -9.81  & 0.81 & 2.08e-2 & 0.61 & 1.12e-2 & -10.73 \\
 & FUV + 24\um & Total gas  & 0.71 & 1.34e-2 & -9.81  & 0.44 & 1.18e-2  & 0.45 & 1.28e-2  & -10.58 \\
\hline
\multicolumn{1}{||c||}{\multirow{6}{*}{$18\kpc <$ R $\la 25\kpc$}}  & TIR & H$_2$ only & 1.57 & 2.63e-1  & -8.80  & 0.97 & 2.00e-1  & 1.19 & 1.01e-1  & -10.94 \\
 & TIR         & HI only    & 1.17 & 4.67e-2 & -9.65  & 0.76 & 2.82e-2 & 0.75 & 1.48e-2 & -10.72 \\
 & TIR         & Total gas  & 0.84 & 2.06e-2  & -9.85  & 0.52 & 1.45e-2  & 0.56 & 1.42e-2  & -10.56 \\
 & FUV + 24\um & H$_2$ only & 1.87 & 3.03e-1  & -9.15  & 1.33 & 2.95e-1  & 1.38 & 1.12e-1  & -11.65 \\
 & FUV + 24\um & HI only    & 1.22 & 2.96e-2 & -9.88  & 0.79 & 2.31e-2 & 0.62 & 1.28e-2 & -10.73 \\
 & FUV + 24\um & Total gas  & 0.81 & 1.59e-2 & -10.06 & 0.42 & 1.14e-2  & 0.52 & 2.06e-1  & -10.69 \\
 \hline
\end{tabular}
\end{table*}



%----------------------------------------------------------------------------------------
%
%----------------------------------------------------------------------------------------
\section{Discussion}
% PB: Here's an idea for how to organize this in paragraphs:
% 1. Global KS law: linear or sublinear?
% 2. KS laws with gas components separated: same or different?
% 3. KS law variation with distance
% 4. EKS law: agrees with Shi? Better or worse than KS?
% 5. EKS with gas components separated: same or different?, compare to KS
% 6. EKS law variation with distance, compare to KS
% 7. Answer the question in the title: which SF law best captures what's going on?

% Not sure where to put this but you also want to answer the question "does it matter which SFR tracer you use" somewhere. All of the factors that go into the "72 plots" should be discussed at some point.

%PB: By the way: better not to start paragraphs with "However" and "On the other hand" as it makes for an awkward transition between topics.

Table ~\ref{table:res} shows results of the fitting parameters for SFR(TIR) and SFR(FUV+24\um) in three different regions. In case of Total gas for the K-S law, there is a sublinear retaliation between SFR and gas mass, in global case and all regions.This result is significatly lower than results from \cite{Ford13}. But it is similar to prediction of the \cite{Shetty13}, which argued that using the hierarchical Bayesian fit leads to power index N lower than other works. 
%PB: more detailed comparison to Ford here: what values did he find, are these significantly different from yours?
%PB: Also compare to Shetty's results: what did he find for other galaxies and are your results compatible with those?

However, if we look at the H$_{2}$ gas or HI gas, we could see power indexes are more similar to the initial sugeestion of the K-S law power index, N$ ~ 1.4\pm 0.15 $. Moreover, these results are in a good agreement with results from \cite{tabatabyi10}. For regions with R less than 8 \kpc, we could see some sublinear values, but it is mostly because H$_2$ is dominant gas in the ISM, and lack of HI gas makes our results very uncertain. Moreover, the K-S law power index increases by the distance from the centre of the galaxy. This could be an affect of metallicity on the SFR or the consequence of lack of H$_2$ gas on outher regions of the galaxy.  
% PB: I would start this paragraph off with a sentence that says what you find by comparing total gas with H_2 only or HI only: different or not? 
% PB: when comparing with Tabatabyi, give values from that paper
% PB: by "lack of HI gas makes our results very uncertain" you mean that the total gas and Hi-only fits are not reliable?
% PB: not sure I see where the last sentence comes from; may need more detail. 

% \begin{figure*}
% \centering
% \includegraphics[width=162mm]{fir_res2.png}
% \caption{Fitting parameters for SFR(TIR) }
% \label{fig:fit,params,fir}
% \end{figure*}

% \begin{figure*}
% \centering
% \includegraphics[width=162mm]{fuv_res.png}
% \caption{Fitting parameters for SFR(FUV + 24\um ) }
% \label{fig:fit,params,fuv}
% \end{figure*}

On the other hand, from fitting the the extended Schmidt law, we have sublinear relation between SFR and gas surface density in global case and all the regions, execpt for gas surface density traced by H$_2$. Besides, for all cases we can see $\beta ~ 0.6$, which is the same as prediction of the \cite{Shi11}. For gas traced by H$_2$ in regions with $18\kpc <$ R $\la 25\kpc$, $\beta$ is way higher than this value. This is could be a result of the lack of H$_2$ gas in these regions. \cite{Kim13} using 3D numirical hydrodynamic simulations showed that in outer disk regions of the galaxy $\eqsigmasfr \propto \eqsigmagas \pho_{sd}^0.5$ where \pho$_{sd}$ is the midplane density of the stellar disk. our results for total gas in regions with $18\kpc <$ R $\la 25\kpc$ shows $\log_{10}(\eqsigmasfr) \propto 0.5 \times \log_{10}(\eqsigmastar)$, which is the same as simulations. However, in this case we found a sublinear relation between \sigmasfr and \sigmagas while \cite{Kim13} found a linear relation. 
%----------------------------------------------------------------------------------------
%SUMMERY
%----------------------------------------------------------------------------------------
\section{Summery}
%----------------------------------------------------------------------------------------
%BIBLOGRAPHY
%----------------------------------------------------------------------------------------
\bibliography{ref.bib}

\end{document}
