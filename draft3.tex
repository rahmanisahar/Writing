\documentclass[useAMS,usenatbib]{mn2e}
\usepackage{hyperref}
\usepackage{graphicx}
\usepackage{natbib}
\bibliographystyle{mn2e}
\usepackage{amsmath}
\usepackage{times}
\usepackage{float}
\usepackage{caption}
\usepackage{subcaption}
\usepackage{multirow}

\newcommand \kpc        {\,{\rm kpc}}
\newcommand \sigmagas    {$\Sigma_{\rm \bld {gas }} $\ }
\newcommand \sigmatotalgas {$\Sigma_{\rm \bld {total gas }} $\ }
\newcommand \eqsigmagas    {\Sigma_{\rm \bld {gas }}}
\newcommand \sigmasfr     {$\Sigma_{\rm \bld {SFR }} $\ }
\newcommand \eqsigmasfr     {\Sigma_{\rm \bld {SFR }}}
\newcommand \sigmastar    {$\Sigma_{\rm \bld {star }} $\ }
\newcommand \eqsigmastar    {\Sigma_{\rm \bld {star }}}
\newcommand \halpha    {H$\alpha $\ }
\newcommand \um    {$\mu$m\ }
\newcommand \mice {$\mu$m}
\newcommand \nprime {N$^\prime$}
\newcommand \eqnprime {N^\prime}
\newcommand \Spitzer {{\it Spitzer }}
\newcommand \Galex {GALEX }
\newcommand \Herschel {{\it Herschel}}
\newcommand \aaj {A\&A}
\newcommand \aarv {A\&ARv}%: Astronomy and Astrophysics Review (the)
\newcommand \aas{A\&AS}%: Astronomy and Astrophysics Supplement Series
\newcommand \afz {Afz}%: Astrofizika
\newcommand \aj {AJ}%: Astronomical Journal (the)
\newcommand \apss {Ap\&SS}%: Astrophysics and Space Science
\newcommand \apj {ApJ}
\newcommand \apjs {ApJS}%: Astrophysical Journal Supplement Series (the)
\newcommand \araa {ARA\&A} %: Annual Review of Astronomy and Astrophysics
\newcommand \asp {ASP Conf. Ser.}%: Astronomy Society of the Pacific Conference Series
\newcommand \azh {Azh}%: Astronomicheskij Zhurnal
\newcommand \baas {BAAS}%: Bulletin of the American Astronomical Society
\newcommand \mem {Mem. RAS}%: Memoirs of the Royal Astronomical Society
\newcommand \mnassa {MNASSA}%: Monthly Notes of the Astronomical Society of Southern Africa
\newcommand \mnras {MNRAS} %: Monthly Notices of the Royal Astronomical Society
%\newcommand {Nature}%(do not abbreviate)
\newcommand \pasj {PASJ}%: Publications of the Astronomical Society of Japan
\newcommand \pasp {PASP}%: Publications of the Astronomical Society of the Pacific
\newcommand \qjras {QJRAS}%: Quarterly Journal of the Royal Astronomical Society
\newcommand \mex {Rev. Mex. Astron. Astrofis.}%: Revista Mexicana de Astronomia y Astrofisica
%\newcommand {Science }%}%(do not abbreviate)
\newcommand \sva {SvA}%: Soviet Astronomy
\newcommand \aap {APP} %:American Academy of Pediatrics
\newcommand \apjl {ApJL} %:The Astrophysical Journal Letters

\begin{document}
% TITLE
\title[STAR FORMATION LAWs IN M31]{Star formation laws in the Andromeda galaxy: effect of the gas mass, stellar mass, and metallicity }
\author[S. Rahmani, et. al.]{S.~Rahmani, S.~Lianou, P.~Barmby\\
Department of Physics $\&$ Astronomy, Western University, London, ON N6A 3K7, Canada}
\maketitle

%----------------------------------------------------------------------------------------
%----------------------------------------------------------------------------------------
\begin{abstract} % it is just my GESF2014 abstract, just for looks I will change it at the end. %PBNew: OK, no comments on this yet.
 We investigate the relation between star formation rate (SFR), gas surface density (\sigmagas) and stellar surface density (\sigmastar) in the Andromeda galaxy (M31) using the Kennicutt-Schmidt (K-S) law and the extended Schmidt law. \cite{Shi11} argued that the surface density of SFR is related not only to gas mass, as in the empirical K-S law,  but also to stellar mass surface density, in the ``extended Schmidt law''. Using all a combination of \halpha and 24 \um emission, a combination of FUV and 24 $\mu$m, and the total infrared emission, we estimate the total SFR in M31 to be between 0.35 M$_{\odot}$yr$^{-1}$ - 0.4 M$_{\odot}$yr$^{-1}$. We use maps of $^{12}$CO and 21cm emission to produce the \sigmagas map.% For regions that are not covered in the $^{12}$CO map, we used the dust map (obtained from $SPIRE$ data) as a tracer of total gas.
 The stellar mass surface density  map is calculated using mid-infrared IRAC imaging. Our preliminary results from testing the K-S law in M31 find a power-law index of  $N = 0.62 \pm 0.01$, at the low end of values compared with previous work. We determine additional K-S law fits using each SFR map independently with molecular hydrogen, atomic hydrogen and total gas maps. We then repeat the analysis considering the stellar mass density in order to test whether the regular or extended Schmidt laws are more appropriate for M31.

\end{abstract}

\begin{keywords} 
galaxies: individual: M31, galaxies: spiral, galaxies: star formation, galaxies: stellar content, galaxies: ISM, stars: formation, ISM: clouds, methods: observational, methods: statistical, techniques: image processing 
\end{keywords}
%----------------------------------------------------------------------------------------
%INTRODUCTION
%----------------------------------------------------------------------------------------


\section{Introduction}
\label{sec:intro}
%P1: Star formation laws and why they are important (introducing K-S law and E-S law)

Understanding the undergoing processes of formation of stars is a necessary step for explaining the formation and evolution of galaxies from the early universe to current epoch, as well as the origin of planetary systems. Stars form by converting the gas inside galaxies; However, no complete answer on how this transformation happens and what physical properties affect it, is available. Various phenomena trigger the collapse of molecular clouds and star formation including the environment of the star-forming region, gas accretion and cooling, H$_2$ formation, the initial mass distribution of gas, chemical compositions of the gas,  dust existence, and other quantities in a galaxy. The physical processes leading to star formation are naturally complex. Therefore, the basis for a theory of star formation requires a strong foundation of empirical data and/or observations.
%me

Finding a relation between the total star formation rate (SFR) and the mass of interstellar gas was proposed as the first attempt at scaling the star formation rate. \cite{Schmidt59} scaled SFR with a power {\it N} of the gas mass, $M_{\rm gas}$. Since then, for more than 50 years empirical star formation laws (relations) have been investigated using both observational data and numerical simulations. Later, \cite{Kennicutt98a} examined H$\alpha$, HI, and CO observations of 61 normal spiral galaxies and 36 starburst galaxies. He showed that the disk-average SFRs and gas densities for these samples are well represented by the Schmidt law with the power-low exponent of $1.4 \pm 0.15$. Using this power index, the Schmidt law can be rewritten as: %me, Schmidt, Kennicutt and every papers on K-S law

\begin{equation}
\label{equ:ks_org}
\eqsigmasfr \propto \eqsigmagas^{1.4\pm0.15},
\end{equation}
which is often referred to as the Kennicutt-Schmidt (K-S) law, where \sigmasfr is the surface density of star formation, and \sigmagas is the surface density of gas in a galaxy. Moreover, \cite{Kennicutt07} investigated the K-S law in spatially resolved regions (0.5 - 2 kpc) in the spiral galaxy M51. To calculate the SFR they used $24 \mu$m plus H${\alpha}$ emission, and for the gas surface density, a constant conversion factor for CO to H$_2$ was applied. They found a correlation, from the radial variation of both the SFR and the gas surface density, with an index of $1.56 \pm 0.04$. %kennicutt07

The K-S law was applied to different types of both local and high redshift galaxies \citep[e.g.,][]{Boissier07,Kennicutt07, Bigiel08, Freundlich13}, but the results were never precise. Low gas density regions, $( < 1-10\sim$ M$_{\odot}$ pc$^{-2})$, such as regions outside the disk of the galaxy, are one of the regions, for which the K-S law fails. The calculated SFR for these regions is much lower than the value predicted by the K-S law \citep[e.g.,][]{Martin01, Bigiel08}. Using different instability models, \cite{Hunter98} showed that in low surface brightness galaxies, the current star formation activity correlates with stellar mass density. Furthermore, a correlation between the star formation rate and the stellar mass density was measured in many studies using both observational data and numerical simulations \citep[e.g.][]{Hunter04,Leroy08,Krumholz09,Shi11,Kim11,Kim13}. \cite{Shi11} found a tight relationship between stellar mass surface density, SFR and gas surface density by studying a large sample of galaxies. They showed this relation as a power law relation and referred to it as the extended Schmidt law: %me + all the ppl I cite

\begin{equation}
\label{equ:es_org}
\eqsigmasfr \propto \eqsigmagas^{\eqnprime} \eqsigmastar^{\beta}
\end{equation}
where \sigmasfr is the SFR surface density while \sigmagas and \sigmastar are the gas mass and stellar mass surface densities, \nprime and $\beta$ are the indices found by \cite{Shi11} to be $1.13 \pm 0.05$ and $0.36\pm0.04$ respectibely in a global case. By testing this relation on sub-kiloparsec resolution regions in 12 spiral galaxies, they showed that the extended Schmidt law predicts SFR as well as the K-S law for spatially resolved regions ($\sim$1 kpc). They demonstrated that the power indices for the local regions are $\eqnprime = 0.8 \pm 0.01$ and $\beta = 0.62\pm0.01$, and concluded that this law is acceptable for low surface brightness galaxies and regions where the K-S law fails.%me

%P2: SFR, measuring, scaling, ...
In order to investigate star formation laws, the first step is to calculate the SFR. Many studies are devoted to determining how flux measurments of the star forming regions can accuratelly be traslated to the rate of the formation of new stars \citep[e.g.,][]{Kennicutt12, Calzetti13, Zhu08, Kennicutt09, Boquien10, Boquien11, Hao11}. \cite{Kennicutt98b} calibrated luminosity of galaxies as a way of measuring the SFR in specific wavelengths using relations between the SFR per unit mass or per unit luminosity and the integrated colour of the system provided by synthesis models \citep[e.g.,][]{Bruzual93}. Subsequent studies tried to find a similar calibration for measuring the SFR of galaxies in other bands. \cite{Kennicutt09} and \cite{Hao11} introduced new SFR calibrations using a combination of \halpha or FUV and far-infrared (FIR) emission, respectively. In addition to using suitable calibrations for each region, considering the differences between the global case and local ($\sim 0.5-1$~\kpc) case is important. In $\S$\ref{sec:sfr}, we will discuss the SFR in the Andromeda galaxy and how it was determined. %me and Keniccutt98b and then me again, with Maryam's corrections

%P3: Gas cloud, variety, observing etc
Calculating the surface density of gas in galaxies is the other most important ingredient of testing star formation laws. Considering the effect of the interstellar medium (ISM) on studying the star formation law is important due to the fact that stars are born from the gas and also release their material into the ISM when they reach the end of their evolution. A map of the total gas in the ISM is produced by direct observations of the gas or by using interstellar dust as a tracer. Neutral and molecular hydrogen are the most common elements in the ISM. Therefore, for producing the map of the total gas in galaxies, maps of these two components can be added together and multiplied by a constant factor to account for heavier elements which are mostly He. Calculations of the gas surface density for the Andromeda galaxy are shown in $\S$\ref{sec:ISM}. %me

%P4: Stellar mass and how to measure it
Additionally, for testing the Extended Schmidt law, measuring the stellar mass density is necessary. Measuring the mass of the stellar population is indirect and subject to significant uncertainties. One method to calculate the stellar mass within a galaxy is based on stellar population models \citep[e.g.][]{ Bruzual93, Kotulla09}. These models are used to connect stellar mass to an observable quantity such as, luminosity in different wavelengths, colours, spectral energy distribution from spectroscopy or multi-band observations. Having different methods to measure stellar mass helps to compare the results and determine whether there are any systematic differences. Comparing the results from measuring the stellar mass via different methods in stellar clusters shows there is no significant difference \citep{Tamm12}. In fact, the differences are found to be on the order of a few x10 percent; so one should be more careful to model subtleties \citep{McLaughlin05}. Calculating stellar mass in the Andromeda galaxy is described in $\S$\ref{sec:starmass}. %me and Tamm and the rest I cited

%P5(2*): What we did and why Andromeda and
In this paper, we present our results from testing and comparing both the K-S law and the Extended Schmidt law on the Andromeda galaxy (M31). Additionally, We applied these two laws on three different regions in M31 to determine if there is any dependence on distance from the centre of the galaxy. Since M31 is the nearest spiral galaxy to the Milky Way, we can have higher resolution images of this spiral galaxy. High resolution images of this spiral galaxy are avaiable that provide data from various regions within the galaxy with different physical properties (e.g. metallicity, surface brightness, gas density). This inside look helps us to test star formation laws in diverse physical conditions. Thus, M31 is a suitable test bed for studying scaling laws. %me


Furthermore, the range of the power index of the K-S law calculated for the Andromeda galaxy is between 0.5 - 2 \citep[e.g.,][]{Tabatabaei10,Ford13}. Various groups used different methods and data, and measured different values for the power index. In the most recent study of the K-S law in M31, \cite{Ford13} tested this law on six regions from the centre of the galaxy using three different ISM maps (H$_2$ only, total gas calculated from H$_2$ plus HI maps, and total gas calculated from dust emission). The measured power indices for each map and each region vary between 0.6 to 2.03. The origin of these variations in the results is still an open
question. It is unclear whether it depends on distance from the centre of the galaxy, metallicity, gas tracers, SFR tracers, fitting methods, or because of the K-S law not working on M31. We also applied a new statistical method instead of the ordinary least square fitting (OLS), which is the most common fitting method for testing K-S law, to test the importance of the fitting method on the results. %me


%----------------------------------------------------------------------------------------
%DATA
%----------------------------------------------------------------------------------------
\section{Data}
\label{sec:data}
Being the closest neighbouring galaxy to our own Milky Way, M31 has extensively been studied using both ground- and space- based telescopes.
%Since M31 is the closest spiral galaxy to the Milky Way, it was aimed by many probes and telescopes, both space based and ground based ones. 
Therefore, large data sets exist on M31 spanning from gamma-ray to radio wavelengths, allowing us to %a variety of data from gamma-ray to radio wavelengths are available. Wide ranges of data provide us the ability to 
measure all the required parameters to test the star formation laws in this galaxy. Table~\ref{table:data} lists the data we used in this paper. Each data comes with a different Full Width Half Maximum (FWHM), spacial resolution and grids. In order to make sure that all our data contains the same amount of information, we smoothed maps with smaller FWHM using Aniano's Kernels' library and convolution code \citep{Aniano12} to have the same resolution as either the $^{12}$CO(J:$1\rightarrow0$) data or atomic gas data. Moreover, considering the Nyquist relation \citep{Nyquist}, we re-sampled and re-gridded all our maps to have a pixel size matching with their new resolution. %I do not like this centence but I am not sure how to fixt it.
%PBnew: what exactly does 'highest resolution available' mean? Same for 'proper pixel size' ? /done!

\begin{table*}
\centering
\caption{Data used in this study.}
\label{table:data}
\begin{tabular}{@{}lcccc}
\hline\hline
Wavelength & FWHM & Coverage area &Telescope
& Ref. \\
\hline
%500 \um & 36\arcsec & $Herschel$ & \cite{Fritz12} \\
%250 \um & 18\arcsec.2 & $Herschel$ & \cite{Fritz12} \\
1550~\AA & 4\arcsec.5 & 5\degr $\times$ 5\degr &GALEX & \cite{Martin05}\\  %:PB: GALEX, KPNO, IRAM should not be in $$
2250~\AA & 6\arcsec & 5\degr $\times$ 5\degr &GALEX & \cite{Martin05}\\%PB (they are abbreviations)
6570~\AA  & 1\arcsec & 2\arcmin.2 $\times$ 0\degr.6 &KPNO& \cite{Massey07}\\
3.6~\um & 1\arcsec.7 & 3\degr.7 $\times$ 1\degr.6 &\Spitzer & \cite{Barmby06} \\ %PB: write {\it Spitzer}, {\it Herschel}
4.5~\um & 1\arcsec.7 & 3\degr.7 $\times$ 1\degr.6 &\Spitzer & \cite{Barmby06} \\ %PB because those are both missions and names
8~\um & 1\arcsec.9 & 1\degr $\times$ 3\degr &\Spitzer & \cite{Barmby06} \\ % PB: of people - italics means the mission, not
24~\um & 6\arcsec & 1\degr $\times$ 3\degr &\Spitzer & \cite{Gordon06} \\ %PB: the person
70~\um & 18\arcsec & 1\degr $\times$ 3\degr &\Spitzer & \cite{Gordon06} \\
160~\um & 12\arcsec & 5\degr.5 $\times$ 2\degr.5 &\Herschel & \cite{Fritz12} \\
350~\um & 24\arcsec.5 & 5\degr.5 $\times$ 2\degr.5 &\Herschel & \cite{Fritz12} \\
2.6~mm & 23\arcsec & 2\degr $\times$ 0\degr.5 &IRAM 30-m & \cite{Nieten06}\\
21~cm & 60\arcsec $\times$ 90\arcsec & 5\degr.2 $\times$ 1\degr.5 &DRAO & \cite{Chemin09}\\
\hline
\end{tabular}
\end{table*}
%PB: in general it's good to write number~unit, ie 2750~\AA /Done
%PB: this means that number and unit don't end up on separate lines


One of the calibrations used to calculate SFR uses FUV emission as a tracer of the star forming regions. FUV emission of M31 was observed by the Galaxy Evolution Explorer (GALEX; \cite{Martin05}). GALEX telescope data contains information for two bands, FUV:1350--1750 and NUV:1750--2750~\AA, with FWHM of 4~\farcs5, and 6~\arcsec.

%\subsection{Visible light}
\label{sec:vislight}
Our other method to calculate the SFR involved using a combination of \halpha and 24 M emission. \cite{Massey06, Massey07} mapped UBVRI and \halpha emission line of M31 as part of the Nearby Galaxies Survey using the KPNO telescope. \halpha maps, which were obtained from the NOAO science archive, are available in 10 overlapping fields and were observed between August 2000 and September 2002. We created a Mosaic image of the maps using the Montage \citep{Berriman08}. Assuming that all stars have the same fraction of \halpha light in their spectra, we subtracted the R band emission from the \halpha image using the averaged scaling factor between R band images and \halpha provided by \cite{Azimlu11}. We also took a masked map from \cite{Azimlu11} and masked all foreground or partially saturated stars (Figure~\ref{fig:halpha}). Details about making the \halpha map can be found in Appendix ~\ref{app:halpha}.

%\subsection{IR and sub mm Data}
Both {\em Spitzer} \citep{Werner04} and {\em Herschel} \citep{Pilbratt10}  space telescopes observed M31 at IR and sub mm bands. Observations with Infrared Array Camera (IRAC; \citep{Fazio04}) are made in four channels (3.6, 4.5, 5.8, and 8 \um). IRAC observations of M31 were obtained by \cite{Barmby06}, covering a region $3\degr.7 \times 1\degr.6$. IRAC 3.6 \um bands were used to calculate the stellar mass. Multiband Imaging Photometer for \Spitzer (MIPS) observed M31 at 24, 60, and 160 \um and covered region $\sim 1\degr \times 3\degr$ \citep{Gordon06}.

Calculation of the SFR using TIR emission requires maps of M31 in 8, 24, 70, and 160 \um. IRAC 8, MIPS 24, MIPS 70 bands were used to calculate TIR emission; however due to better quality and resolution of $Herschel$s$'$data, PACS \citep[Photodetector Array Camera and Spectrometer;][]{Poglitsch10} were used. PACS 160 \um observation of the Andromeda galaxy alongside of PACS 100 \um band and three (250 $\mu$m, 350 $\mu$m, and 500 \um) SPIRE \citep[Spectral and Photometric Imaging Receiver;][]{Griffin10} bands coveres $\sim 5\degr.5 \times 2\degr.5$ regions in M31 \citep{Fritz12}.

%\subsection{Radio Emission}
Emission from $^{12}$CO (J:$1\rightarrow0$) line that, was observed on-the-fly with the IRAM 30-m telescope \citep{Nieten06} was used to calculate the H$_2$ column density of the interstellar medium. This mapa a region $2\degr \times 0\degr.5$ that has the best resolution CO image of M31 (FWHM = $23\arcsec$), but the smallest coverage of the halo of the galaxy among the maps we used in this paper. In this study we also used 21 cm emission map from the atomic gas (HI) in M31 from \cite{Chemin09}. They used Synthesis telescope and the 26 m antenna at the Dominion Radio Astrophysical Observatory to map HI emission in the 21 cm band of M31.

\section{Measuring the component of Star Formation Laws}

%----------------------------------------------------------------------------------------
%STAR FORMATION RATE
%----------------------------------------------------------------------------------------
%----------------------------------------------------------------------------------------
\subsection{Star Formation Rate}
\label{sec:sfr}
\subsubsection{FUV plus 24 M Star Formation Rate}

In this project the first star formation map was produced using a calibration of a combination of FUV emission and 24 \um emission introduced by \cite{Hao11}. Since the peak of the emission of massive and young stars (O-B and A type) is in the UV part of the spectrum, this wavelength is one of the most commonly used star formation rate indicators \citep[e.g.,][]{Kennicutt89}. The UV emission traces recently formed stars with the age of $\sim 100$ Myr \citep[e.g.,][]{Kennicutt98a, Calzetti05}. However, the downside of only using the FUV light as a tracer of the star forming region is that this emission is very sensitive to dust extinction. On the other hand, 24 \um emission is dominated by emission from dust heated by UV photons from young and hot stars. This emission is sensitive to the star formation timescale of $\la10$ Myr \citep{Calzetti07}. The FUV band of \Galex and the 24 \um band of \Spitzer data were used to measure the SFR calibrated by \cite{Hao11} as:
\begin{equation}
\label{equ: fuvplus24}
SFR =4.46\times10^{-44}[L(FUV)_{obs}+3.89L(24\mu m)],
\end{equation}
where SFR is in M$_{\odot}$yr$^{-1}$, and L(24$\mu$m) and L(FUV)$_{obs}$ are in erg~s$^{-1}$. L(FUV)$_{obs}$ is the luminosity of the galaxy in the FUV emission. The subscript "obs" is shown to indicate that the FUV emission is not corrected for the effect of extinction. Nevertheless, this emission was corrected for the foreground star's emission using a method introduced by \cite{Leroy08}. We assumed for any pixel in FUV$_{obs}$ map, if I$_{NUV}$/I$_{FUV}$ $>$ 15 that pixel value is dominated by foreground emission and masked them out. We also masked the same region in the galaxy for the 24 \um map. Using this method, we calculated the total SFR to be $0.31\pm 0.04$~M$_{\odot}$~yr$^{-1}$.

%PB: I like this figure! Looks nice.

\begin{figure*}
    \centering
    \begin{subfigure}[b]{1\textwidth}
        \centering
        \includegraphics[width=164mm]{sfr_fuv.eps}
        \caption{SFR,FUV+24 M}
        \label{fig:sfr,fuv}
    \end{subfigure}
    \hfill
    \begin{subfigure}[b]{1\textwidth}
        \centering
        \includegraphics[width=164mm]{sfr_halpha.eps}
        \caption{SFR,H$_{\alpha}$}
        \label{fig:sfr_halpha}
    \end{subfigure}
    \hfill
    \begin{subfigure}[b]{1\textwidth}
        \centering
        \includegraphics[width=164mm]{sfr_fir.eps}
        \caption{SFR,TIR}
        \label{fig:sfr,fir}
    \end{subfigure}
    \caption{SFR map from a combination of FUV + 24 M emission (top), \halpha and 24 M emission (middle), and total infrared emission (bottom)}
    \label{fig:sfrs}
\end{figure*}

\subsubsection{\halpha plus 24 \um Star Formation Rate}
\label{sec:sfr_halpha}

The SFR can be calculated using combination of \halpha and \Spitzer  24\um maps. \halpha emission of galaxies is dominated by light emitted from young and massive stars. Star formation time scaled traced by this emission is $\sim 6-10$ Myr \citep[e.g.,][]{Kennicutt09, Calzetti13}. Therefore, \halpha alone could be used to calibrate the SFR \citep[e.g.,][]{Osterbrock06, Kennicutt09}. However, \halpha emission is very sensitive to dust extinction, too. Therefore, the combination of the \halpha and 24\um of is considerable choice for calculating the SFR. wW used a calibration introduced by \cite{Kennicutt09} to measure the SFR:
\begin{equation}
\label{equ: halphaplus24_g}
SFR = 5.5 \times 10^{-42}[L(H{\alpha})_{obs} + 0.020L(24\mu m)]
\end{equation}
where L(H${\alpha}$)$_{obs}$ is the observed \halpha luminosity without correction for internal dust attenuation, given in the unit of erg~s$^{-1}$. L(24$\mu$m) is the $24\mu$m IR luminosity in erg~s$^{-1}$, and SFR is in M$_{\odot}$yr$^{-1}$. It was indicated that the formulation above can only be used at the global situation i.e., the total SFR.

\cite{Calzetti07} introduced a new calibration using the same band passes in a case of local regions (Figure~\ref{fig:sfr_halpha}):
\begin{equation}
\label{equ: halphaplus24_l}
SFR = 5.5 \times 10^{-42}[L(H{\alpha})_{obs} + 0.033L(24\mu m)]
\end{equation}
the units here are the same as equation~\ref{equ: halphaplus24_g}. Equation~\ref{equ: halphaplus24_l} is useful for regions less than $\sim 1$ kpc. According to  \cite{Calzetti07} calibration constant in equations ~\ref{equ: halphaplus24_g} and ~\ref{equ: halphaplus24_l} can be different by changing an IMF. Changes in the IMF mostly affect on calibration based on \halpha luminosity, mostly because a significant amount of the ionizing photons comes from the stars more massive than $\sim 20$M$\sun$. Changing the calibration constant would change the total SFR by a factor of 1.5. In this paper, we used \cite{Kroupa01} IMF with an upper mass limit of 100M$\sun$. Total SFR calculated from equation~\ref{equ: halphaplus24_g} is $0.35 \pm 0.01$~M$\sun$~yr$^{-1}$, which is in good agreement with previous studies.

Creating the SFR map using \halpha plus 24 \um was described in Section~\ref{sec:sfr_halpha}. In order to investigate the SFR laws, we used the same method of the fitting as Section~\ref{sec:fitting}. The fitting results using SFR(\halpha $+$ 24 \um ) are totally different from the other two SFR tracers. The main reason for this difference could is mainly because of the extreme amount of the nan values in the map especially lost of the data in the centre of the galaxy.\halpha data does not have a smooth background. Also the centre of galaxy in this data is diffused, so we do not have any data from the centre of the M31. Since it was the only available \halpha data from the M31, We used this data anyway.

\subsubsection{Total Infrared Star Formation Rate}
\label{sec:sfr_fir}
The total infrared emission of a system can also be used as a tracer of the SFR. Dust absorbs radiation from hot and young stars and re-emits it at infrared wavelengths; however, there is no one-to-one mapping between IR and UV emission. Therefore, integrating over the full wavelength range of the IR part and calculating the TIR luminosity is a better tracer of the SFR instead of IR single-band emission \citep{Calzetti13}. \cite{Calzetti13} assumed that stellar bolometric emission is completely absorbed and re-emitted by dust, i.e., $L_{star}(bol) = L(TIR)$, and calibrated the TIR luminosity of a system to calculate the SFR for a stellar population undergoing constant star formation over 100 Myr:

\begin{equation}
\label{equ:sfr_fir}
SFR(\rm TIR) = 2.8 \times10^{-44}L(\rm TIR),
\end{equation}
where SFR(TIR) and L(TIR), the star formation rate calculated from TIR emission and TIR luminosity, are in M$_{\odot}$~yr$^{-1}$ and erg~s$^{-1}$, respectively. Part of the dust emission in galaxies (specially in M31) comes from the dust heated by the cosmic background radiation \citep[e.g.][]{Dole06, Calzetti13, Mattsson14}, hence, using equation~\ref{equ:sfr_fir} overestimates the SFR in M31.  

TIR luminosity can be measured from the integration of the IR part of a spectral energy distribution (SED) of a galaxy or combining photometry data at different IR wavelengths. As a second approach, \cite{Draine07} modelled \Spitzer data and calibrated IR single-band photometry data to calculate the TIR luminosity. \cite{Boquien10} tested and modified this calibration as:
\begin{equation}
 \label{equ: TIR}
L(\rm TIR) = 0.95L(8) + 1.15L(24) + L(70) + L(160),
\end{equation}
where L(8), L(24), L(70), and L(160) are luminosities of the galaxy in 8 $\mu$m, 24 $\mu$m, 70 $\mu$m,and 160 \um in units of erg~s$^{-1}$. We used equation~\ref{equ:sfr_fir} to calculate the second map of the SFR of M31 (Figure~\ref{fig:sfr,fir}). Using this method, we calculated the total SFR to be $0.40 \pm 0.04$~M$_{\odot}$~yr$^{-1}$. Table~\ref{table:sfr} compares the total SFR from literature and the present work. 

\begin{table*}
\begin{minipage}{100mm}
\caption{Comparison of the total star formation rate of M31}
\label{table:sfr}
\begin{tabular}{@{}lcc}
\hline\hline
Ref.&Method&SFR(M$_{\odot}$yr$^{-1}$) \\
\hline
Current work&FUV and 24 M&0.31 $\pm$ 0.04\\
Current work&\halpha and 24 M&0.35 \\
Current work&TIR luminosity&0.4 $\pm$ 0.04\\
\cite{Ford13}&FUV and 24 M&0.25\\
\cite{Ford13}&TIR luminosity&0.48-0.52\\
\cite{Azimlu11}& \halpha and 24 M&0.34\\
\cite{Azimlu11}&Extinction corrected \halpha&0.44\\
\cite{Tabatabaei10}&Extinction corrected \halpha&0.27--0.38\\
\cite{Barmby06}&Infrared 8\um luminosity& 0.4\\
\hline
\end{tabular}
\end{minipage}
\end{table*}


%----------------------------------------------------------------------------------------
%ISM
%----------------------------------------------------------------------------------------
\subsection{Gas Surface Density}
\label{sec:ISM}

 The molecular form of hydrogen is really hard to detect. H$_2$ molecules are located in cool and dense molecular clouds, but fortunately they are not the only component of the molecular gas. CO (usually the $J(1\rightarrow 0)$ rotational transition, observed at 2.6 mm) is used as a tracer of the mass of the molecular cloud dominated by molecular hydrogen \citep[see, for example,][] {Sanders84}. Higher rotational transition of CO can be used as a tracer as well, but they are not as common as $J(1\rightarrow 0)$. The relation between CO emission and H$_2$ cloud mass is shown as:

\begin{equation}
\label{equ:conversion}
\rm N_{H_2}/\rm cm^{-2} = X_{CO} \times I_{CO}/\rm K km s^{-1}
\end{equation}
where X$_{CO}$ is the conversion factor (also known as the X-factor). The X-factor could be different in regions within a galaxy due to the difference in metallicity \citep{Wilson95, Bosselli02, Bolato13}. In case of M31, different values of X-factor ($1-5.6 \times 10^{20}$) were adopted in the previous studies \citep[e.g.][]{Ford13, Bolato13, Leroy11, Bolato08, Nieten06}; however, since any constant differences in X-factor leads to horizontal changes in a plot of log(SFR) versus Log($\Sigma_{\rm {gas}}$), and does not affect our K-S law power index for molecular gas,  we chose $X_{\rm {CO}}= 2 \times 10^{20}$  the same as many other works \citep[e.g.][]{Ford13, Smith12}. Since HI mass dominates H$_2$ mass in M31, the constant X-factor has negligible affect on the total gas mass. The total gas mass was calculated from:
\begin{equation}
\label{equ:total_gas}
\rm M_{\rm \bld {total \, gas}} = 1.36[M_{HI}+M_{H_2}]
\end{equation}
where  factor of 1.36 is a constant to consider He and the other heavier elements affect on the total gas mass, M$_{HI}$ is mass of neutral hydrogen that were obtained from 21cm observations, and M$_{H_2}$ calculated from equation~\ref{equ:conversion} in units of M$_{\sun}$.


%----------------------------------------------------------------------------------------
%STELLAR MASS
%----------------------------------------------------------------------------------------
\subsection{Stellar Mass}
\label{sec:starmass}
\begin{figure*}
\centering
\includegraphics[width=164mm]{star.eps}
\caption{Stellar Mass surface density with pixel size of 4.3\arcsec and FWHM of 1\arcsec.7. The stellar surface density map is produced using IRAC 3.6~$\mu$m data and its calibration presented in equation~\ref{equ:eskew}.}
\label{fig:stellarmass}
\end{figure*}

Near IR band such as: 3.6$\mu$m, 4.5\um and K band emission, are almost insensitive to the emission from the young stellar population, dust absorption and emission, which make them a good tracer of the mass of the older stellar populations\citep[e.g.,][]{Elmgreen84, Eskew12, Zhu10}. We calculated the stellar mass within M31 using calibration of the 3.6\um flux by \cite{Eskew12}. They used data from the Large Magellanic Cloud, and introduced an empirical relation between stellar mass and flux of stars:

\begin{equation}
\label{equ:eskew}
M _{\star}= 10^{5.97} F_{3.6}\left(\frac{D}{0.05}\right)^2
\end{equation}
where M$_{\star}$ is the stellar mass in solar masses, $F_{3.6}$ is 3.6\um flux in Jy, and $D$ is the distance of the galaxy in Mpc. Figure~\ref{fig:stellarmass} shows the stellar mass map for M31 using IRAC 3.6\um. In this figure pixels with dark blue colour show zero M$_{\odot}$, while dark red ones show pixels with mass of more than 6.4e3 M$_{\odot}$. Using this method we calculated the total stellar mass of the galaxy to be $6.9 \times 10^{10}$M$_{\odot}$ with 6$\%$ uncertainty. This result is in fair agreement with the result from \cite{Tamm12}. They calculated that stellar mass in M31 is $(10-15) \times 10^{10}$M$_{\odot}$, 56$\%$ of which is in the disk with the rest in the bulge of the galaxy.


%----------------------------------------------------------------------------------------
%Metallicity
%----------------------------------------------------------------------------------------
\subsection{Metallicity}
%----------------------------------------------------------------------------------------
%SFR LAWS
%----------------------------------------------------------------------------------------
\section{Scaling SFR}
\subsection{Star formation laws}

We made three SFR maps, three gas mass density maps and a stellar mass density map of whole galaxy. We applied both laws on whole galaxy on each two combination of SFR and gas mass density maps, along with applying the laws on three different regions. As a result we were be able to test each law 36 times. In total, we produced 72 different plots, which show either the K-S law or the extended Schmidt law, both in whole galaxy and in different distances from the centre of M31. Additionally, we can use these results to see if there is any correlation between the SFR laws and the metallicity, as it is going to be described in ~\ref{sec:fitting}.
  
\begin{figure*}
\includegraphics[width=164mm]{Results_all_3_regs.pdf}
\caption{Same as Figure~\ref{fig:ks,all}, but in this figure we separated pixels from different regions in the galaxy by their colours. The regions with $R< 8\kpc$, $8\kpc < R < 18\kpc$, and $18\kpc < R \la 25\kpc$ are shown in red, green and blue, respectively.}
\label{fig:ks,regs}
\end{figure*}

%PB: no space, so SFR(TIR)
Figure~\ref{fig:ks,all} shows the pixel by pixel fitting of the K-S law on the whole galaxy. First row plots are the surface density of the SFR(TIR), which is calculated using  the TIR emission form section ~\ref{sec:sfr_fir}, versus the gas surface density traced by only the molecular gas (H$_2$),  only atomic gas (HI), and total gas from right to left, respectively. The only difference between the upper and lower plots is that in the lower row the SFR indicator is the FUV plus 24 M emission. Figure~\ref{fig:ks,regs} shows the same data as Figure~\ref{fig:ks,all}. However, in this figure we separated pixels from different regions in the galaxy by their colours. The regions with R less than 8$\kpc$, $8\kpc < $R $< 18\kpc$, and $18\kpc <$ R $\la 25\kpc$ are shown in red, green and blue, respectively. 

We took the similar approach for the extended Schmidt law. Figures~\ref{fig:es,all}--\ref{fig:es,regs,fuv,tot} show the results of the pixel by pixel fitting of this law. For the extended Schmidt law fitting, we plotted our results in 3D to illustrate the relationship between the surface density of the SFR, the gas mass surface density, and stellar mass density more clearly. In these series of plots, X-axis is the gas mass  surface density, either molecular gas (H$_2$), the atomic gas (HI) or the total gas, Y-axis is the SFR(TIR) or SFR(FUV + 24 $\mu$m), and Z-axis is the stellar mass surface density. Shadows of our data on each surface are also plotted, to have a more clear picture of correlations between components. For a comparison, one can easily conclude that shadows on the X-Y surface is a reproduction of the K-S law. 

\begin{figure*}
    \centering
    \begin{subfigure}[b]{0.3\textwidth}
        \centering
        \includegraphics[width=\textwidth]{es_tot_fir_vs_h2_22.png}
        \caption{SFR,TIR vs surface density of H$_2$}
        \label{fig:es,all,fir,h2}
    \end{subfigure}
    \hfill
    \begin{subfigure}[b]{0.3\textwidth}
        \centering
        \includegraphics[width=\textwidth]{es_tot_fir_vs_hi2.png}
        \caption{SFR,TIR vs surface density of HI}
        \label{fig:es,all,fir,hi}
    \end{subfigure}
    \hfill
   \begin{subfigure}[b]{0.3\textwidth}
        \centering
        \includegraphics[width=\textwidth]{es_tot_fir_vs_tot2.png}
        \caption{SFR,TIR vs surface density of total gas}
        \label{fig:es,all,fir,tot}
    \end{subfigure}
    \hfill
     \begin{subfigure}[b]{0.3\textwidth}
        \centering
        \includegraphics[width=\textwidth]{es_tot_fuv_vs_h22.png}
        \caption{SFR,FUV+24 M vs surface density of H$_2$}
        \label{fig:es,all,fuv,h2}
    \end{subfigure}
     \hfill
   \begin{subfigure}[b]{0.3\textwidth}
        \centering
        \includegraphics[width=\textwidth]{es_tot_fuv_vs_hi2.png}
        \caption{SFR,FUV+24 M vs surface density of HI}
        \label{fig:es,all,fuv,hi}
    \end{subfigure}
    \hfill
    \begin{subfigure}[b]{0.3\textwidth}
        \centering
        \includegraphics[width=\textwidth]{es_tot_fuv_vs_tot2.png}
        \caption{SFR,FUV+24 M vs. surface density of total gas}
        \label{fig:es,all,fuv,tot}
    \end{subfigure}
    \hfill
     \begin{subfigure}[b]{0.3\textwidth}
        \centering
        \includegraphics[width=\textwidth]{es_tot_halpha_vs_h22.png}
        \caption{SFR,H$\alpha$+24 M vs surface density of H$_2$}
        \label{fig:es,all,halpha,h2}
    \end{subfigure}
     \hfill
   \begin{subfigure}[b]{0.3\textwidth}
        \centering
        \includegraphics[width=\textwidth]{es_tot_halpha_vs_hi2.png}
        \caption{SFR,H$\alpha$+24 M vs surface density of HI}
        \label{fig:es,all,halpha,hi}
    \end{subfigure}
    \hfill
    \begin{subfigure}[b]{0.3\textwidth}
        \centering
        \includegraphics[width=\textwidth]{es_tot_halpha_vs_tot2.png}
        \caption{SFR,H$\alpha$+24 M vs. surface density of total gas}
        \label{fig:es,all,halpha,tot}
    \end{subfigure}
    \caption{The results from fitting the extended Schmidt law on data from whole galaxy using pixel by pixel method. The plots show the SFR vs. the surface density of gas, and z-axis is the surface density of the stellar mass. Each figure shows different combinations of the SFR tracer and gas mass tracer results. As in Figure~\ref{fig:ks,all}, the analyses use different pixel sizes. Each point in the plots with the surface density of H$_2$ as a tracer of gas mass represents a region of size $\sim$30~pc and each point in plots with surface density of HI or total gas mass represent a region of size $\sim$155~pc.}
    \label{fig:es,all}
\end{figure*}



\begin{figure*}
\centering
\includegraphics[width=162mm]{tir_vs_tot_3_reg_extendS.png}
\caption{Fitting result from the SFR(TIR) vs total gas. Same as Figure~\ref{fig:es,all,fir,tot}, but in this figure we denote pixels from different regions in the galaxy by their colours. The regions with $R< 8\kpc$, $8\kpc < R < 18\kpc$, and $18\kpc < R \la 25\kpc$ are shown in red, green and blue, respectively.}
\label{fig:es,regs,fir,tot}
\end{figure*}

\begin{figure*}
\centering
\includegraphics[width=162mm]{fuv_vs_tot_3_reg_extendS.png}
\caption{Fitting result from the SFR(FUV + 24 $\mu$m) vs total gas. Same as Figure~\ref{fig:es,all,fuv,tot}, but in this figure we denote pixels from different regions in the galaxy by their colours. The regions with $R< 8\kpc$, $8\kpc < R < 18\kpc$, and $18\kpc < R \la 25\kpc$ are shown in red, green and blue, respectively.}
\label{fig:es,regs,fuv,tot}
\end{figure*}

\begin{figure*}
\centering
\includegraphics[width=162mm]{halpha_vs_tot_3_reg_extendS_97perse.png}
\caption{Fitting result from the SFR(H$\alpha$ + 24 $\mu$m) vs total gas. Same as Figure~\ref{fig:es,all,halpha,tot}, but in this figure we denote pixels from different regions in the galaxy by their colours. The regions with $R< 8\kpc$, $8\kpc < R < 18\kpc$, and $18\kpc < R \la 25\kpc$ are shown in red, green and blue, respectively.}
\label{fig:es,regs,fir,tot}
\end{figure*}


\begin{table*}
\caption{Fitting parameters of the SF laws from applying the Bayesian method}
\label{table:res}
\begin{tabular}{ccccrccrr}
\hline\hline
\multicolumn{1}{c}{\multirow{1}{*}{Region}} & SFR Tracer        & Gas Tracer & N    & A      & \nprime & $\beta$ & A$^\prime$ \\
\hline
\multicolumn{1}{c}{\multirow{9}{*}{Whole Galaxy}} & TIR               & H$_2$ only & 1.05 & -8.98  & 0.97    & 0.35    & -9.53      \\
 & TIR               & HI only    & 1.19 & -9.69  & 0.79    & 0.75    & -10.56     \\
 & TIR               & Total gas  & 0.82 & -9.84  & 0.51    & 0.56    & -10.40     \\
 & FUV + 24 M       & H$_2$ only & 1.16 & -9.36  &         &         &            \\
 & FUV + 24 M       & HI only    & 1.19 & -9.96  & 0.81    & 0.61    & -10.62     \\
 & FUV + 24 M       & Total gas  & 0.78 & -10.09 & 0.49    & 0.45    & -10.50     \\
 & H$\alpha$ + 24 M & H$_2$ only & 1.16 & -9.26  & 1.07    & 0.38    & -9.86      \\
 & H$\alpha$ + 24 M & HI only    & 0.54 & -9.61  & 0.53    & 0.76    & -10.69     \\
 & H$\alpha$ + 24 M & Total gas  & 0.54 & -9.82  & 0.37    & 0.59    & -10.55     \\
\hline
\multicolumn{1}{c}{\multirow{9}{*}{R$< 8\kpc$}} & TIR               & H$_2$ only & 1.01 & -8.95  & 0.92    & 0.36    & -6.66      \\
 & TIR               & HI only    & 1.63 & -9.93  & 1.14    & 0.65    & -10.38     \\
 & TIR               & Total gas  & 1.07 & -9.99  & 0.87    & 0.41    & -10.21     \\
 & FUV + 24 M       & H$_2$ only & 1.07 & -9.30  & 1.03    & 0.18    & -9.62      \\
 & FUV + 24 M       & HI only    & 1.29 & -10.06 & 1.02    & 0.45    & -10.34     \\
 & FUV + 24 M       & Total gas  & 0.89 & -10.12 & 0.81    & 0.21    & -10.18      \\
 & H$\alpha$ + 24 M & H$_2$ only & 1.14 & -9.23  & 1.03    & 0.41    & -6.47      \\
 & H$\alpha$ + 24 M & HI only    & 1.29 & -9.56  & 0.96    & 1.54    & -11.10     \\
 & H$\alpha$ + 24 M & Total gas  & 0.90 & -9.99  & 0.65    & 1.27    & -10.88    \\
\hline
\multicolumn{1}{c}{\multirow{9}{*}{$8\kpc < $R $< 18\kpc$}} & TIR               & H$_2$ only & 1.08 & -8.90  & 1.00    & 0.40    & -9.59      \\
 & TIR               & HI only    & 1.16 & -9.58  & 0.84    & 0.77    & -10.53     \\
 & TIR               & Total gas  & 0.79 & -9.77  & 0.50    & 0.58    & -10.38     \\
 & FUV + 24 M       & H$_2$ only & 1.21 & -9.28  & 1.15    & 0.31    & -9.82      \\
 & FUV + 24 M       & HI only    & 1.19 & -9.88  & 0.85    & 0.63    & -10.61     \\
 & FUV + 24 M       & Total gas  & 0.75 & -10.04 & 0.47    & 0.47    & -10.47     \\
 & H$\alpha$ + 24 M & H$_2$ only & 1.18 & -9.18  & 1.09    & 0.41    & -9.88      \\
 & H$\alpha$ + 24 M & HI only    & 0.56 & -9.49  & 0.72    & 0.81    & -10.67     \\
 & H$\alpha$ + 24 M & Total gas  & 0.52 & -9.71  & 0.40    & 0.61    & -10.47  \\
\hline
\multicolumn{1}{c}{\multirow{9}{*}{$18\kpc <$ R $\la 25\kpc$}} & TIR               & H$_2$ only & 2.08 & -8.76  & 1.81    & 1.32    & -5.82      \\
 & TIR               & HI only    & 1.30 & -9.68  & 0.83    & 0.78    & -10.54     \\
 & TIR               & Total gas  & 0.86 & -9.83  & 0.55    & 0.57    & -10.38     \\
 & FUV + 24 M       & H$_2$ only & 2.87 & -9.18  & 2.89    & 1.49    & -11.33     \\
 & FUV + 24 M       & HI only    & 1.27 & -9.95  & 0.85    & 0.64    & -10.63     \\
 & FUV + 24 M       & Total gas  & 0.80 & -10.08 & 0.52    & 0.48    & -10.49     \\
 & H$\alpha$ + 24 M & H$_2$ only & 2.86 & -9.05  & 3.12    & 1.30    & -10.86     \\
 & H$\alpha$ + 24 M & HI only    & 0.63 & -9.62  & 0.44    & 0.81    & -10.62     \\
 & H$\alpha$ + 24 M & Total gas  & 0.60 & -9.84  & 0.39    & 0.66    & -10.52     \\
 \hline
\end{tabular}
\end{table*}

\subsection{Fitting method}
\label{sec:fitting}

\begin{figure*}
\centering
\includegraphics[width=164mm]{Results_for_all_regions.pdf}
\caption{The result from fitting the Kennicutt-Schmidt law on data from whole galaxy using the pixel by pixel method. The points in the plots represent 
different pixel sizes due to differences in the resolution of the H$_2$ and HI maps. Each point in the plots with the surface density of H$_2$ as a tracer of gas mass represents a region of size $\sim$30~pc and each point in the plots with surface density of HI or total gas mass represents a region of size $\sim$155~pc.
} 
\label{fig:ks,all}
\end{figure*}

Producing the SFR, gas mass and stellar mass maps, provides enough data to examine and compare the K-S law and the extended Schmidt law. We investigated the K-S law and the extended Schmidt law by using pixel by pixel method in whole galaxy as well as comparing the laws in three different regions of galaxy based on their distance from the centre of the galaxy. Our final results have two sets of different pixel sizes. That is because the H$_2$ gas mass map has better resolution and smaller pixel size of the HI gas mass map. Consequently, any plots with surface density of H$_2$ as a tracer of the gas mass will have more data points.


We can re-write equations~\ref{equ:ks_org}~and~\ref{equ:es_org} in a logarithmic scale. Hence, we would have two linear equations, as:

\begin{subequations}
\begin{equation}
\label{eq:sfr_law_ks_log}
\log_{10} \eqsigmasfr = N~\log_{10} \eqsigmagas + A
\end{equation}
\begin{equation}
\label{eq:sfr_law_es_log}
\log_{10} \eqsigmasfr = \eqnprime~\log_{10} \eqsigmagas + \beta~\log_{10}\eqsigmastar  + A^\prime
\end{equation}
\end{subequations}
where, N, $\eqnprime$, $\beta$, A, and A$^\prime$ are free parameters of fittings. Units in the equation above, are the same as the units in the equations ~\ref{equ:ks_org} and ~\ref{equ:es_org}.

We found the free parameters by applying the hierarchical Bayesian linear regression method as described in \cite{Shetty13}. Shetty and his colleagues used a Bayesian linear regression approached to develop a new method to find the K-S law parameters, considering the measurement uncertainties as well as hierarchical data structure. For the K-S law's fitting parameters, we used a 'R' code provided by Shetty's group. For the extended Schmidt law, we expended the code in the way that instead of using simple linear regression, which is the case of the K-S law, it uses multiple linear regression. In this case, we were able to examine the effect of the stellar mass in the SFR as shown in equation ~\ref{eq:sfr_law_es_log}. 

Uncertainties on each observable quantities were measured by following the method described in \cite{Kennicutt07}. We used quadratic sum of variance of the local background of each luminosity map, from the original pixel size images, Poisson noise of images, and calibration uncertainties. Since we used pixel by pixel fitting, we do not consider the photometric uncertainties in our methods.
%PBnew: not sure I see how pixel by pixel fitting means that you don't have to
% worry about photometric uncertainty; maybe I am just not understanding this sentence
%meeting: background uncertainties

%PB20150428: I just realized that you first  discuss the 3 different regions a couple of paragraphs above this. Think this means that this paragraph should go after the first paragraph of section 4.1.
We also test each laws on three different regions. The regions were chosen the same as regions introduced in \cite{Draine14}. They found that across the galaxy, with increasing distance from the centre of galaxy dust/H ratio declines monotonically. Considering that dust/H ratio has a direct relation to metallicity, they conclude that the metallicity of M31 changes in the same manner. As a result they introduced these three different regions as $R< 8\kpc$, $8\kpc < R < 18\kpc$, and $18\kpc < R \la 25\kpc$. 
%PB20150428: not really clear from the text above why the 3 different regions are defined where they are. Need more detail about what Draine et al did. that made them think that 8 and 18 kpc were important, although I still think that (as in my comments below) the precise Md/Mh values are not what's important. Also, from the Draine paper I see that they are using elliptical annuli and not circular ones (which Ford et al did as well). You are doing the same, right? Need to describe this.
%PBnew: don't think you really need the Md/MH values here, just giving the
% region locations is enough. But this makes me think: what are the effects of
% a changing Md/Mh on the derived SFR? Could they be driving (or hiding) differences in SFR laws between regions?

% \begin{equation}
% \label{eq:dust2gas_vs_R}
% \frac{M_d}{M_H} \approx
% \left\{ \begin{array}{l l}
% 0.0280 \exp(-R/8.4\kpc)  & R< 8\kpc\\
% 0.0165 \exp(-R/19\kpc)   & 8\kpc < R < 18\kpc\\
% 0.0605 \exp(-R/8\kpc)    & 18\kpc < R \la 25\kpc ~~~,\\
% \end{array}
% \right.
% \end{equation}

%new metal materials
%PB20150428: this is a good start but needs more detail and grammar work. (Some of this material might go better in the introduction.) List the other things besides metallicity that change with distance from the centre of the galaxy. Explain in more detail how the metallicity affects the components of the SFR law. If there is a correlation between metallicity and SFR (last sentence), what is the sense of the correlation, how big is it?
Applying the SFR laws in these regions, provides a tool to consider the effect of the distance from the centre of the galaxy on SFR laws as well as the effect of the metallicity on them. 
 
The effect of the metallicity on the star formation laws are very complicated.  Metallicity affect on each component of the star formation laws in different way. There are many studies on correlation of the metallicity with stellar mass, SFR, and the effect of the metallicity on the ISM \citep[e.g.,][]{Boissier03, Leroy08, Krumholz09, Mannucci10, Dib11a, Lilly13}. The scope of these studies is that metallicity has a critical role on the formation of the H$_2$ gas. Thus, there should be a correlation between metallicity and the SFR.

\subsection{750 kpc}

\begin{table*}
\caption{Fitting parameters of the SF laws from applying the Bayesian method in regions around 750 Kpc}
\label{table:res750}
\begin{tabular}{ccccrccrr}
\hline\hline
\multicolumn{1}{c}{\multirow{1}{*}{Region}} & SFR Tracer        & Gas Tracer & N    & A      & \nprime & $\beta$ & A$^\prime$ \\
\hline
\multicolumn{1}{c}{\multirow{3}{*}{Whole Galaxy}}
 & FUV + 24 M       & H$_2$ only & 0.65(2.93) & -8.27(3.34)  & 1.61(0.68)  & 0.61(0.38) & 5.29(-10.01)   \\
 & FUV + 24 M       & HI only    & 1.19 & -9.96  & 0.85    & 0.65    & -11.60     \\
 & FUV + 24 M       & Total gas  & 0.78 & -10.09 & 0.52    & 0.49    & -10.48     \\
\hline
\multicolumn{1}{c}{\multirow{3}{*}{R$< 8\kpc$}}
 & FUV + 24 M       & H$_2$ only & 0.65(2.93) & -8.27(3.34) & 0.86(2.25)    & 0.36(0.60)    & -10.57(-0.71)      \\
 & FUV + 24 M       & HI only    & 1.73 & -10.87 & 1.03    & 1.42    & -10.52     \\
 & FUV + 24 M       & Total gas  & 0.88 & -9.94 & 1.45    & 0.83    & -9.11      \\
\hline
\multicolumn{1}{c}{\multirow{3}{*}{$8\kpc < $R $< 18\kpc$}}
 & FUV + 24 M       & H$_2$ only & 2.02 & -8.43  &    1.83     &  0.78       &   -9.16         \\
 & FUV + 24 M       & HI only    & 1.38 & -11.06  & 1.26    & 0.65    & -11.33     \\
 & FUV + 24 M       & Total gas  & 0.80 & -10.00 & 0.72    & 0.53    & -10.17     \\
\hline
\multicolumn{1}{c}{\multirow{3}{*}{$18\kpc <$ R $\la 25\kpc$}} 
 & FUV + 24 M       & H$_2$ only & 1.50 (4.30) & -11.00(-1.02)  &  5.35    & 4.10    & -10.22   \\
 & FUV + 24 M       & HI only    & 1.45 & -10.75  & 1.05    & 0.84    & -11.44     \\
 & FUV + 24 M       & Total gas  & 0.85 & -9.86 & 0.58    & 0.66    & -10.26     \\
 \hline
\end{tabular}
\end{table*}


\subsection{Face On}

\begin{table*}
\caption{Fitting parameters of the SF laws from applying the Bayesian method Face on}
\label{table:res}
\begin{tabular}{ccccrccrr}
\hline\hline
\multicolumn{1}{c}{\multirow{1}{*}{Region}} & SFR Tracer        & Gas Tracer & N    & A      & \nprime & $\beta$ & A$^\prime$ \\
\hline
\multicolumn{1}{c}{\multirow{2}{*}{Whole Galaxy}}
 & TIR               & Total gas & 0.79 & -9.97  &  0.51   & 0.56   & -10.39   \\
 & FUV + 24 M       & Total gas  & 0.44 & -9.99 & 0.35    & 0.41    & -10.32     \\
\hline
\multicolumn{1}{c}{\multirow{2}{*}{R$< 8\kpc$}}
 & TIR               & Total gas    & .73 & -9.98 & 0.53    & 0.62    & -10.50     \\
 & FUV + 24 M       & Total gas  & 0.36 & -9.97 & 0.33    & 0.42    & -10.33      \\
\hline
\multicolumn{1}{c}{\multirow{2}{*}{$8\kpc < $R $< 18\kpc$}}
 & TIR               & Total gas    & 0.79 & -9.96  & 0.52    & 0.58    & -10.41     \\
 & FUV + 24 M       & Total gas  & 0.43 & -9.98 & 0.36    & 0.41    & -10.33     \\
\hline
\multicolumn{1}{c}{\multirow{2}{*}{$18\kpc <$ R $\la 25\kpc$}} 
 & TIR              & Total gas   & 0.83 & -9.96  & 0.52    & 0.54    & -10.34     \\
 & FUV + 24 M       & Total gas  & 0.49 & -9.99 & 0.38    & 0.40    & -10.29     \\
 \hline
\end{tabular}
\end{table*}
%----------------------------------------------------------------------------------------
%
%----------------------------------------------------------------------------------------
\section{Discussion}

Table ~\ref{table:res} shows the results of the hierarchical Bayesian fit and their uncertainties for the whole galaxy and each different regions.. Due to the similarity of our results for both SFR(TIR) and SFR(FUV+24$\mu$m), we only interpret the results from the SFR(FUV+24$\mu$m). Results from the SFR(H$\alpha$) fitting are going to be discussed in the Appendix ~\ref{app:halpha}.
%PBnew: Use a few sentences to discuss similarity between SFR(TIR) and SFR(FUV+24$\mu$m), then go on to say you're only going to discuss one. /done

\subsection{The K-S law in M31}

As mentioned before, the K-S law in M31 was tested by many groups. In one of the most recent works on M31, \cite{Ford13} investigated the K-S law in six annuli and the global case using the ordinary least squares (OLS) fitting method. To make comparisons between our results easier, we averaged the power indices for their annuli which overlap with our three different regions so we could compare results in the same regions.
\cite{Tabatabaei10} also applied the K-S law to M31 using the OLS fitting method, and have different results from \cite{Ford13}.

Our results for the global case of the K-S law show that using total gas gives a sub-linear relation between the \sigmasfr and \sigmatotalgas. While, for using the H$_{2}$ only gas or HI only gas we find the nearly super-linear relation. These results are different from power indices estimated by \cite{Ford13}, $N=2.03\pm0.04$ in case of total gas and $N=0.6\pm0.01$ in case of H$_{2}$ only gas. \cite{Tabatabaei10} estimated the K-S law power index $N=1.30\pm0.05$ using the total gas and $N=0.96\pm0.03$ using the H$_{2}$ only gas. The reason behind these differences is mostly because of the statical method we chose. The OLS fitting using \sigmatotalgas gives us the power index of $N=1.20\pm0.10$ which is in good agreement with \cite{Tabatabaei10} results.

 \cite{Bigiel08} argued that there is no universal relationship between the surface density of the total gas and the SFR. They also showed that the SFR and the molecular hydrogen have a linear relationship. Furthermore, \cite{Shetty13} used the same data as \cite{Bigiel08} and argued that using the hierarchical Bayesian fit leads to significant galaxy-by-galaxy variation between power indices and all of them are lower than \cite{Bigiel08}. 
%PBnew: Need to link material in this paragraph to our results.

The sub-linear power index in the case of the total gas surface density suggests that the star formation efficiency (surface density of SFR per surface density of gas, SFE) is decreasing by increasing the total gas. Consequently, the depletion time increases by increasing the total gas. On the contrary, the super-linear relation suggests that the depletion time is decreasing with increasing the amount of the CO or/and HI gas. The further conclusion is more similar to the initial suggestion of the K-S law. If we assume that stars form in the molecular clouds, the decreasing of the SFE with increasing the total gas is justifiable.  

% PBnew: in following paragraph, switch the order so that you compare with Ford
% (observational) first, then Krumholz (theory). Is there theoretical work that 
% predicts different dependencies? It's important to note the work that disagrees
% with you as well as agrees.
The results of the hierarchical Bayesian fitting on the regions with different distances from the centre of the galaxy give us similar values as the global case. For all three regions, the K-S law using the total gas leads to sub-linear relation between the SFR and the surface density of the total gas. Likewise, for using the H$_{2}$ only gas or HI only gas, this relationship is either linear or nearly super-linear. These results agree with results from \cite{Krumholz09}. They showed that for the regions with \sigmagas $\leq 100$ M$_{\odot}$pc$^{-2}$, \sigmasfr and $\Sigma_{H_2}$ have a nearly linear correlation. They also conclude that, at the intermediate total gas column density, (\sigmagas $\leq 100$ M$_{\odot}$pc$^{-2}$), \sigmasfr and \sigmagas have a linear or slightly sub-linear correlation which is similar to our results. The K-S law power index varies with the distance of the galaxy. Considering molecular gas, the power law index is increasing by going to further distances from the centre of the galaxy, which can be seen in the results from \cite{Ford13}, too. However, this growth does not happen for total gas or atomic gas cases. The increasing of the power index in this case may be an effect of the decreasing metallicity across the galaxy, which leads to have less SFE, or other physical properties of the galaxy such as stellar mass (see section ~\ref{sec:es_res}).
  
\subsection{The extended Schmidt law in M31}
\label{sec:es_res}
The extended Schmidt law has been a recently proposed law \citep{Shi11}, which shows a tight correlation between the SFR and the total gas surface density and the surface density of the stellar mass. The results of the fitting the extended Schmidt law in a global case give us a sub-linear relation between the SFR and the density of stars. Results from the fitting the extended Schmidt law using the molecular gas only are %$\eqnprime = 1.09 \pm 0.01$ and $\beta = 0.25 \pm 0.004$ which
follows the same trend of the extended Schmidt law in the global case,% $\eqnprime = 1.38 \pm 0.05$ and $\beta = 0.36\pm0.02$, 
,which is the nearly super-linear correlation with \sigmagas and sub-linear correlation with \sigmastar. Although, the original results were average results of many type galaxies using the total gas surface density. Our results from the fitting the total gas surface density gives us a lower power index of the gas mass and a higher power index of the stellar mass.This relation suggests that, although increasing the total gas causes a decrease to the SFE but the existing stars will increase the SFE. 

 The fitting results on the regions with different distances from the centre of the galaxy are more similar to the results from \cite{Shi11}, $\eqnprime = 0.8 \pm 0.01$ and $\beta = 0.62\pm0.01$. Taking the average $\beta$  from results of using molecular gas only gives us $\beta = 0.63$ which is in good agreement with \cite{Shi11} but, $\eqnprime$ is more than the original suggestion. In the local case, the same as global case power indices calculated using total gas surface density, are much smaller than the original suggestion of \cite{Shi11}. However, the final indices reported in the extended Schmidt law are an average over 12 different spiral galaxies, and our result is a lower limit for this law. 
%PBnew: why is our result a lower limit? 
 Another reason of differences between indices could be because of fitting method. Using the OLS fitting, one only will consider the SFR errors, nevertheless, the hierarchical Bayesian regression fitting considers uncertainties on all the parameters. Different treating with uncertainties would give us the different result.
 

 Thus, we can conclude that the stars affect on the SFR specially on the regions with lower amount of the gas. \cite{Kim13} using 3D numerical hydrodynamic simulations showed that in outer disk regions of the galaxy  where the total gas is dominated by HI gas,, the SFR have correlation with $\rho_{sd}^{0.5}$ where $\rho_{sd}$ is the mid-plane density of the stellar disk plus dark matter. This is because of the fact that in outer regions of the disk the gravity from stars dominates. This results are in agree with our finding regarding that in outer regions of M31 the stellar surfaces density has more effect on the SFR.

%PBnew: need to discuss metallicity here as well.
%new metal material
%%PB20150428: this paragraph neeeds a topic sentence
\cite{Leroy08} showed that the SFR(FUV+24$\mu$m) does not change with changing metallicity. However, \cite{Boissier03} showed that the metallicity has inverse correlation with CO-H$_2$ conversion factor, which means that, using a constant conversion factor at high metallicity regions causes the overestimation of the molecular gas, and in low metallicity it leads to the underestimation of the H$_2$. Since \cite{Draine14} showed that metallicity in the centre of M31 is more than the outer disk regions, power index from the fitting the K-S law in the case of the molecular gas only or total gas should be steeper than what we measured.
%PB20150428: rather than saying "the true PL index should be steeper" I think it's better to say "our estimate is an underestimate" or "our estimate is a lower limit". What does it mean for the PL index in the centre to be an underestimate? (ie if we corrected for this would it agree better with other regions/other studies, or worse, or what?)

%%PB20150428: this paragraph neeeds a topic sentence
\cite{Mannucci10} and \cite{Lilly13} using both observations and simulation results showed that metallicity depends on both the stellar mass and the SFR. They found that at any fixed stellar mass, metallicity and the SFR have an inverse correlation. Afterward, assuming the K-S law holds for their sample they found the correlation between, metallicity and the stellar mass, the gas mass and the SFR. They found that increasing both the SFR and the stellar mass increases metallicity, but increasing the gas mass decreases metallicity (equation 9 in \citep{Mannucci10}).
 %%PB20150428: what does this mean for our work? Does it agree or disagree?


%----------------------------------------------------------------------------------------
%SUMMERY
%----------------------------------------------------------------------------------------
\section{Summary}

In order to test the effect of stars on the SFR in both global and local scales in M31, we have determined surface density of star formation rate, gas mass and the stellar mass of this galaxy. We produced three SFR maps in M31. The first uses a combination of the FUV and 24 \um emission. For the second SFR map, we used a combination of the H$\alpha$ and 24 \um emission, and the third one was produced using total-infrared luminosity. We noticed that H$_\alpha$ data available for M31 is not suitable for SFR study ~\ref{app:halpha}. We calculated the total SFR from FUV and 24 \um emission is 0.31$\pm$ 0.04 M$_{\odot}$yr$^{-1}$.

We also produced the ISM map, using molecular gas only, atomic gas only and the total gas. %PBnew: following 2 sentences are too much detail for a summary. 
For the molecular gas map, we assumed the CO-H$_2$ conversion factor is $2 \times 10^{20}$ (K km s$^{-1}$)$^{-1}$cm$^{-2}$. 
In order to calculate the total gas, we combined the molecular gas and atomic gas and multiplied this combination by a factor of 1.36 for effect of the heavier elements. We have determined the stellar mass using 3.6 \um emission to create the stellar mass surface density. Then, we used the hierarchical Bayesian fitting to apply both the K-S law and the extended Schmidt law on M31.

We confirmed \cite{Shi11} suggestions that the surface density of stars has impact on the SFR, and in regions with law gas surface brightness this impact is even more important. We also showed that, one of the reasons of adversity of results of power index of K-S law in M31 comes from differences in fitting method and error analysis. %PBnew: do you mean 'diversity' instead of 'adversity' ? 
%PBnew: description of your results should be more extensive: 2 sentences is not enough.


%----------------------------------------------------------------------------------------
%Acknowledgement 
%----------------------------------------------------------------------------------------
%\section*{Acknowledgement}
The authors thank Dr. R. Shetty for giving us his hierarchical Bayesian fitting code, and Dr. D. Kruijssen for his idea about 3D plots.
The authors acknowledge research support from the Natural Sciences and Engineering Research Council of Canada and from the Academic Development Fund of the University of Western Ontario. 
This research made use of Montage, funded by the National Aeronautics and Space Administration's Earth Science Technology Office, Computation Technologies Project, under Cooperative Agreement Number NCC5-626 between NASA and the California Institute of Technology. Montage is maintained by the NASA/IPAC Infra-red Science Archive.
%----------------------------------------------------------------------------------------
%BIBLOGRAPHY
%----------------------------------------------------------------------------------------
%PBnew: something is wrong with the references here: should have 'et al.' instead of 'e.a.'
\bibliography{ref.bib}

%----------------------------------------------------------------------------------------
%Appendix
%----------------------------------------------------------------------------------------

\appendix
\section{SFR from H$\alpha$ plus 24 \um produces and result}
\label{app:halpha}


\begin{figure*}
\centering
\includegraphics[width=164mm]{halpha.eps}
\caption{Mosaic created using the Montage programme from six fields of H$\alpha$ emission maps of M31 from \cite{Massey07}. The result image from Montage was continuum subtracted and masked out for all point sources. Centre of the galaxy was masked out due to saturation of data in an R-band image.}
\label{fig:halpha}
\end{figure*}

As mentioned in Section~\ref{sec:vislight}, we used the H$\alpha$ data from the Nearby Galaxies Survey \citep{Massey07} to create the SFR map. These data are publicly available in H$\alpha$ and R band in 10 overlapping fields across the disk of M31. Making a spatially resolved SFR map of M31 using H$\alpha$ emission requires removal of background and stellar emission in each field, masking out foreground stars and saturated regions, creating a mosaic of H$\alpha$ continuum-subtracted map, and finally correcting for the flux contribution to the \halpha filter from the [N II] 6583~\AA line.

 R-band images were used to remove the stellar emission from H$\alpha$ using scaling factor between fluxes in these two band passes determined by \cite{Azimlu11}. They estimated continuum emission in each H$\alpha$ image, using photometry results in both \halpha and R-band images, and determined the scaling factor. These images were observed from a ground base telescope over two years time. Consequently, there is a non-negligible sky contribution in the background which should be estimated and removed from each image in both bands. On the other hand, there is no image of the further area in the sky to measure the sky contribution from there, which makes the removal of background emission even harder. The only choice were left here is to subtract the local background for each region% which gave us an image with non-smooth background.
. However, subtracting a local background of images will result in a non-smooth background in final mosaic image. 

To create the final mosaic, first we removed the background from each region in both \halpha and R-band. Our second step was to subtract the continuum from the \halpha images. Since both \halpha and R-band images were aligned on the same coordinate grid, for each field, \halpha image was subtracted by R-band image multiplied by the scaling factor. At the end, we masked out all the foreground and saturated regions which include a $10\arcmin \times 10\arcmin$ region in the centre of the galaxy. In total, nearly 50$\%$ of the pixels in the \halpha map were masked out. To account for the flux contribution of the [N II] emission we used the flux ratio of the [NII]$/$\halpha $= 0.54$ from \cite{Kennicutt08}.


\subsubsection{\halpha plus 24 M Star Formation Rate}
\label{sec:sfr_halpha}

The SFR can be calculated using combination of \halpha and \Spitzer  24 M maps. \halpha emission of galaxies is dominated by light emitted from young and massive stars. Star formation time scaled traced by this emission is $\sim 6-10$ Myr \citep[e.g.,][]{Kennicutt09, Calzetti13}. Therefore, \halpha alone could be used to calibrate the SFR \citep[e.g.,][]{Osterbrock06, Kennicutt09}. However, \halpha emission is very sensitive to dust extinction, too. Therefore, the combination of the \halpha and 24 M of is considerable choice for calculating the SFR. wW used a calibration introduced by \cite{Kennicutt09} to measure the SFR:
\begin{equation}
\label{equ: halphaplus24_g}
SFR = 5.5 \times 10^{-42}[L(H{\alpha})_{obs} + 0.020L(24\mu m)]
\end{equation}
where L(H${\alpha}$)$_{obs}$ is the observed \halpha luminosity without correction for internal dust attenuation, given in the unit of erg~s$^{-1}$. L(24$\mu$m) is the $24\mu$m IR luminosity in erg~s$^{-1}$, and SFR is in M$_{\odot}$yr$^{-1}$. It was indicated that the formulation above can only be used at the global situation i.e., the total SFR.

\cite{Calzetti07} introduced a new calibration using the same band passes in a case of local regions (Figure~\ref{fig:sfr_halpha}):
\begin{equation}
\label{equ: halphaplus24_l}
SFR = 5.5 \times 10^{-42}[L(H{\alpha})_{obs} + 0.033L(24\mu m)]
\end{equation}
the units here are the same as equation~\ref{equ: halphaplus24_g}. Equation~\ref{equ: halphaplus24_l} is useful for regions less than $\sim 1$ kpc. According to  \cite{Calzetti07} calibration constant in equations ~\ref{equ: halphaplus24_g} and ~\ref{equ: halphaplus24_l} can be different by changing an IMF. Changes in the IMF mostly affect on calibration based on \halpha luminosity, mostly because a significant amount of the ionizing photons comes from the stars more massive than $\sim 20$M$\sun$. Changing the calibration constant would change the total SFR by a factor of 1.5. In this paper, we used \cite{Kroupa01} IMF with an upper mass limit of 100M$\sun$. Total SFR calculated from equation~\ref{equ: halphaplus24_g} is $0.35 \pm 0.01$~M$\sun$~yr$^{-1}$, which is in good agreement with previous studies.

Creating the SFR map using \halpha plus 24 \um was described in Section~\ref{sec:sfr_halpha}. In order to investigate the SFR laws, we used the same method of the fitting as Section~\ref{sec:fitting}. The fitting results using SFR(\halpha $+$ 24 \um ) are totally different from the other two SFR tracers. The main reason for this difference could is mainly because of the extreme amount of the nan values in the map especially lost of the data in the centre of the galaxy.\halpha data does not have a smooth background. Also the centre of galaxy in this data is diffused, so we do not have any data from the centre of M31. Since it was the only available \halpha data from M31, We used this data anyway.





\end{document}
