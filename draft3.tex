\documentclass[useAMS,usenatbib]{mn2e}
\usepackage{hyperref}
\usepackage{graphicx}
\usepackage{natbib}
\bibliographystyle{mn2e}
\usepackage{amsmath}
\usepackage{times}
\usepackage{float}
\usepackage{caption}
\usepackage{subcaption}
\usepackage{multirow}

\newcommand \kpc        {\,{\rm kpc}}
\newcommand \sigmagas    {$\Sigma_{\rm \bld {gas }} $\ }
\newcommand \sigmatotalgas {$\Sigma_{\rm \bld {total gas }} $\ }
\newcommand \eqsigmagas    {\Sigma_{\rm \bld {gas }}}
\newcommand \sigmasfr     {$\Sigma_{\rm \bld {SFR }} $\ }
\newcommand \eqsigmasfr     {\Sigma_{\rm \bld {SFR }}}
\newcommand \sigmastar    {$\Sigma_{\rm \bld {star }} $\ }
\newcommand \eqsigmastar    {\Sigma_{\rm \bld {star }}}
\newcommand \halpha    {H$\alpha $\ }
\newcommand \um    {$\mu$m\ }
\newcommand \mice {$\mu$m}
\newcommand \nprime {N$^\prime$}
\newcommand \eqnprime {N^\prime}
\newcommand \Spitzer {{\it Spitzer }}
\newcommand \Galex {GALEX }
\newcommand \Herschel {{\it Herschel}}
\newcommand \aaj {A\&A}
\newcommand \aarv {A\&ARv}%: Astronomy and Astrophysics Review (the)
\newcommand \aas{A\&AS}%: Astronomy and Astrophysics Supplement Series
\newcommand \afz {Afz}%: Astrofizika
\newcommand \aj {AJ}%: Astronomical Journal (the)
\newcommand \apss {Ap\&SS}%: Astrophysics and Space Science
\newcommand \apj {ApJ}
\newcommand \apjs {ApJS}%: Astrophysical Journal Supplement Series (the)
\newcommand \araa {ARA\&A} %: Annual Review of Astronomy and Astrophysics
\newcommand \asp {ASP Conf. Ser.}%: Astronomy Society of the Pacific Conference Series
\newcommand \azh {Azh}%: Astronomicheskij Zhurnal
\newcommand \baas {BAAS}%: Bulletin of the American Astronomical Society
\newcommand \mem {Mem. RAS}%: Memoirs of the Royal Astronomical Society
\newcommand \mnassa {MNASSA}%: Monthly Notes of the Astronomical Society of Southern Africa
\newcommand \mnras {MNRAS} %: Monthly Notices of the Royal Astronomical Society
%\newcommand {Nature}%(do not abbreviate)
\newcommand \pasj {PASJ}%: Publications of the Astronomical Society of Japan
\newcommand \pasp {PASP}%: Publications of the Astronomical Society of the Pacific
\newcommand \qjras {QJRAS}%: Quarterly Journal of the Royal Astronomical Society
\newcommand \mex {Rev. Mex. Astron. Astrofis.}%: Revista Mexicana de Astronomia y Astrofisica
%\newcommand {Science }%}%(do not abbreviate)
\newcommand \sva {SvA}%: Soviet Astronomy
\newcommand \aap {APP} %:American Academy of Pediatrics
\newcommand \apjl {ApJL} %:The Astrophysical Journal Letters

\begin{document}
% TITLE
\title[STAR FORMATION LAWS IN M31]{Star formation laws in the Andromeda galaxy: gas, stars, metals and the surface density of star formation}
%PB150629: suggestion (not sure it's better): Star formation laws in the Andromeda galaxy: gas, stars, metals and the surface density of star formation
%older: Star formation laws in the Andromeda galaxy: gas mass, stellar mass, and metallicity
\author[S. Rahmani, et. al.]{S.~Rahmani, S.~Lianou, P.~Barmby\\
Department of Physics $\&$ Astronomy, Western University, London, ON N6A 3K7, Canada}
\maketitle

%----------------------------------------------------------------------------------------
%----------------------------------------------------------------------------------------
\begin{abstract} 
 We investigate star formation laws in the the Andromeda galaxy (M31). We studied the effects of the gas surface density (\sigmagas), stellar surface density (\sigmastar) and metallicity on the Star formation rate (SFR) both in local ($\sim$ 30 pc, $\sim$ 155 pc, and $\sim$ 750 pc) and global scales. Using a combination of \halpha and 24 \um emission, a combination of FUV and 24 $\mu$m, and the total infrared emission, we estimate the total SFR in M31 to be between 0.35 M$_{\odot}$yr$^{-1}\pm 0.04$~and~0.4 M$_{\odot}$yr$^{-1} \pm 0.04$. We use maps of $^{12}$CO and 21 cm emission to produce the \sigmagas map. We produce the stellar mass surface density using IRAC 3.6 $\mu$m and measured it to be $6.9 \times 10^{10}$ M $_{\odot}$. The hierarchical Bayesian regression analysis was applied to fit the star formation power laws. For the Kennicutt Schmidt (K-S) law in M31, we find the power-law index to be between $N = 0.52$ and $N = 2.87$. We show that the power index varies by gas tracer more than by changing the SFR tracer. We also show that the power index changes by distance from the centre of the galaxy. We that show the fitting method is important in the study of the star formation laws and that using different fitting methods leads to different power indices. We then repeat the analysis considering the stellar mass density in order to compare the K-S law and the extended Schmidt law and determine which is more appropriate for M31. We find a correlation between the surface density of SFR and the stellar mass surface density, which confirms that the K-S law needs to be extended to consider the other physical properties of galaxies. We could not find any correlation between metallicity, the SFR and the stellar mass surface density.

\end{abstract}

\begin{keywords} 
galaxies: individual: M31, galaxies: spiral, galaxies: star formation, galaxies: stellar content, galaxies: ISM, stars: formation, ISM: clouds, methods: observational, methods: statistical, techniques: image processing 
\end{keywords}
%----------------------------------------------------------------------------------------
%INTRODUCTION
%----------------------------------------------------------------------------------------


\section{Introduction}
\label{sec:intro}
%P1: Star formation laws and why they are important (introducing K-S law and E-S law)

Understanding the undergoing processes of formation of stars is a necessary step for explaining the formation and evolution of galaxies from the early universe to the current epoch, as well as the origin of planetary systems. Stars form when molecular cloud cores collapse; However, a complete picture explaining the physical processes involved in the formation of stars is yet to emerge. Various phenomena trigger the collapse of molecular clouds and star formation including the environment of the star-forming region, gas accretion and cooling, H$_2$ formation, the initial mass distribution and chemical compositions of the gas,  existence of dust, and other quantities in a galaxy. The physical processes leading to the formation of stars are naturally complex. Therefore, the basis for a theory of star formation requires a strong foundation of empirical data and/or observations.
%me

The first attempt at scaling the star formation rate was to find a relation between the total star formation rate (SFR) and the mass of interstellar gas. \cite{Schmidt59} proposed a power law relation between SFR and the gas mass ($M_{\rm gas}$) with {\it N} as the power law exponent. Since then, for more than 50 years empirical star formation laws (relations) have been investigated using both observational data and numerical simulations. Later, \cite{Kennicutt98a} examined H$\alpha$, HI, and CO observations of 61 normal spiral galaxies and 36 starburst galaxies. He showed that the disk-average SFRs and gas densities for these samples are well represented by the Schmidt law with the power-law exponent of $1.4 \pm 0.15$. Using this power index, the Schmidt law can be rewritten as: %me, Schmidt, Kennicutt and every papers on K-S law
%Should I write laws (relations) or laws/relations?

\begin{equation}
\label{equ:ks_org}
\eqsigmasfr \propto \eqsigmagas^{1.4\pm0.15},
\end{equation}
\noindent which is often referred to as the Kennicutt-Schmidt (K-S) law, where \sigmasfr is the surface density of star formation, and \sigmagas is the surface density of gas in a galaxy. Moreover, \cite{Kennicutt07} investigated the K-S law in spatially resolved regions (0.5 - 2 kpc across) in the spiral galaxy M51. To calculate the SFR, they used a combination of 24 $\mu$m and H${\alpha}$ emission, and for the gas surface density, a constant conversion factor for CO to H$_2$ was applied. They showed that the K-S law also holds locally in galaxies, and found {\it N} to be $1.56 \pm 0.04$. %kennicutt07

The K-S law was applied to different types of both local and high-redshift galaxies \citep[e.g.][]{Boissier07,Kennicutt07, Bigiel08, Freundlich13}, but the results were never precise. Low gas density regions $( < 1-10\sim$ M$_{\odot}$ pc$^{-2})$, such as regions outside disks of galaxies, are an example of the regions for which the K-S law fails. The calculated SFR for these regions is much lower than the value predicted by the K-S law \citep[e.g.][]{Martin01, Bigiel08}. Using different instability models, \cite{Hunter98} showed that in low surface brightness galaxies, the current star formation activity correlates with stellar mass density. Furthermore, a correlation between the star formation rate and the stellar mass density was measured in many studies using both observational data and numerical simulations \citep[e.g.][]{Hunter04,Leroy08,Krumholz09,Shi11,Kim11,Kim13}. \cite{Shi11} found a close relationship between stellar mass surface density, SFR and gas surface density by studying a large sample of galaxies. They showed this relation as a power law and referred to it as the extended Schmidt law \footnote{\cite{Shi11} introduced the extended Schmidt law as a power law relation between the star formation efficiency (surface density of SFR per surface density of gas, SFE) and \sigmastar, but we are using the equation~\ref{equ:es_org}, which is the equation 5 in their paper, to be consistent with equation~\ref{equ:ks_org}.}: %me + all the ppl I cite

\begin{equation}
\label{equ:es_org}
\eqsigmasfr \propto \eqsigmagas^{\eqnprime} \eqsigmastar^{\beta},
\end{equation}
\noindent where \sigmasfr is the SFR surface density while \sigmagas and \sigmastar are the gas mass and stellar mass surface densities, \nprime and $\beta$ are the indices found by \cite{Shi11} to be $1.13 \pm 0.05$ and $0.36\pm0.04$, respectively, in a global case. By testing this relation in sub-kiloparsec resolution regions in 12 spiral galaxies, they showed that the extended Schmidt law works just as well in spatially resolved regions ($\sim$1 kpc) as the K-S law. Furthermore, they found the power indices for the local regions to be $\eqnprime = 0.80 \pm 0.01$ and $\beta = 0.63\pm0.01$, and concluded that this law is acceptable for low surface brightness galaxies and regions where the K-S law fails.%me

%P5: How to measure it
A galaxy's metallicity is another physical quantity that likely plays a role in the rate of star formation. However, it affects each component of star formation laws in a different way, making it complicated to fully understand its implications. There are many studies on correlation of metallicity with stellar mass, SFR, as well as the effect of metallicity on the ISM \citep[e.g.][]{Boissier03, Leroy08, Krumholz09, Mannucci10, Dib11a, Lilly13}. These studies suggest that metallicity has a critical role on the formation of H$_2$ gas. Thus there should be a correlation between metallicity and the SFR. \cite{Krumholz09} introduced a theoretical SFR law which suggests that the SFR and metallicity of galaxies have a power-law correlation, but the power index changes with the amount of total gas. %should I say more?
 Using results obtained from both observations and analytical models, \cite{Mannucci10} and \cite{Lilly13} showed that metallicity depends on both stellar mass and the SFR, which they referred to as a 'fundamental metallicity relation'. They found that at any fixed stellar mass, metallicity and the SFR have an inverse correlation. It should be noted that the above two correlations are only valid in the global case. However, \cite{Leroy08} showed that the SFR(FUV+24$\mu$m) does not change with changing metallicity. \cite{Roychowdhury15} argued that in nearby spiral galaxies effect of metallicity on the SFR calculation is smaller than the calibration errors and can be ignored. On the other side, \cite{Boissier03} showed that metallicity has inverse correlation with CO-H$_2$ conversion factor. Therefore, using a constant conversion factor at high metallicity regions causes the overestimation of the molecular gas, and in low metallicity it leads to the underestimation of the H$_2$. 


%P2: SFR, measuring, scaling, ...
In order to investigate star formation laws, the first step is to calculate the SFR. Many studies are devoted to determining how flux measurements of star-forming regions can be accurately translated into the rate of formation of new stars \citep[e.g.][]{Kennicutt12, Calzetti13, Zhu08, Kennicutt09, Boquien10, Boquien11, Hao11}. \cite{Kennicutt98b} calibrated luminosity of galaxies as a way of measuring the SFR in specific wavelengths using relations between the SFR per unit mass or per unit luminosity and the integrated colour of the system provided by synthesis models \citep[e.g.][]{Bruzual93}. Subsequent studies tried to find a similar calibration for measuring the SFR of galaxies in other bands. \cite{Kennicutt09} and \cite{Hao11} introduced new SFR calibrations using a combination of \halpha or FUV and far-infrared (FIR) emission. In addition to using suitable calibrations for each region, considering the differences between the global and the local ($\sim 0.5-1$~\kpc) cases is important. In $\S$~\ref{sec:sfr}, we will discuss the SFR in the Andromeda galaxy and how it was determined. %me and Keniccutt98b and then me again, with Maryam's corrections

%P3: Gas cloud, variety, observing etc
Calculating the surface density of gas in galaxies is another important parameter in testing star formation laws. Considering the effect of the interstellar medium (ISM) is essential in understanding star formation laws since stars are born from gas and also release their material into the ISM when they reach the end of their evolution. A map of the total gas in the ISM is produced by direct observations of the gas or by using interstellar dust as a tracer. Neutral and molecular hydrogen are the most common elements in the ISM. Therefore, for producing the map of the total gas in galaxies, maps of these two components can be added together and multiplied by a constant factor to account for heavier elements, mostly He. Calculations of the gas surface density for the Andromeda galaxy are shown in $\S$~\ref{sec:ISM}. %me

%P4: Stellar mass and how to measure it
Additionally, for testing the Extended Schmidt law, measuring the stellar mass density is necessary. Measuring the mass of stellar populations is indirect and subject to significant uncertainties. One method to calculate the stellar mass within a galaxy is based on stellar population models \citep[e.g.][]{ Bruzual93, Kotulla09}. These models are used to connect stellar mass to an observable quantity such as luminosity in different wavelengths, colours, spectral energy distribution from spectroscopy or multi-band observations. Having various methods to measure stellar mass helps to compare the results and determine whether there are any systematic differences. In fact, the differences are found to be on the order of a few x10 percent; so one should be more careful to model subtleties \citep{McLaughlin05}. However, Comparing the results from measuring the stellar mass via different methods in stellar clusters showed no significant difference \citep{Tamm12}. Section $\S$~\ref{sec:starmass} describes our calculations of stellar mass in the Andromeda galaxy. %me and Tamm and the rest I cited



%P6(2*): What we did and why Andromeda
In this paper, we present our results from testing and comparing both the K-S law and the Extended Schmidt law on the Andromeda galaxy (M31). Additionally, We applied these two laws on three different regions in M31 to determine if there is any dependence on distance from the centre of the galaxy. Since M31 is the nearest spiral galaxy to our own, high resolution images of this galaxy are available, providing data from various regions with different physical properties (e.g. metallicity, surface brightness, gas density). This inside look helps us test star formation laws in diverse physical conditions. Thus M31 is a suitable testbed for studying scaling laws. %me


Furthermore, the range of power indices calculated for the Andromeda galaxy using the K-S law is between 0.5 and 2 \citep[e.g.][]{Tabatabaei10,Ford13}. The use of different methods and data results in measuring different values for the power index. In a recent study of the K-S law in M31, \cite{Ford13} tested this law on six regions extending radially outward from the centre of the galaxy using three different ISM maps (H$_2$ only, total gas calculated from H$_2$ plus HI maps, and total gas calculated from dust emission). The measured power indices for each map and each region vary between 0.6 to 2.03. The origin of these variations in the results is still an open question. It is unclear whether it depends on distance from the centre of the galaxy, metallicity, gas tracers, SFR tracers, fitting methods, or it is because of the K-S law not working in M31. In order to examine the dependence of the results on the fitting method, we also applied a new statistical method instead of Ordinary Least Square (OLS) fitting, which is more commonly used in the literature to test the K-S law ($\S$~\ref{sec:fitting}). %me




%----------------------------------------------------------------------------------------
%DATA
%----------------------------------------------------------------------------------------
\section{Data}
\label{sec:data}
Being the closest neighbouring galaxy to the Milky Way (MW), M31 has extensively been studied using both ground- and space- based telescopes. Therefore, large datasets exist on M31 spanning from gamma-ray to radio wavelengths, allowing us to measure all the required parameters to test star formation laws in this galaxy. Table~\ref{table:data} lists the data we used in this paper. Each data comes with a different angular resolution and pixel size. In order to match the images at different wavelengths, we smoothed the maps to the same FWHM using the kernel library and convolution code of \citet{Aniano11}. Our final results have two different angular resolutions depending on which gas tracer we use in studying the SFR laws. For studying the correlations between the SFR and molecular gas, our maps have the same resolution and pixel size as the $^{12}$CO(J:$1\rightarrow0$) data; while for investigating the relationships between the SFR and atomic gas (or total gas), we smoothed and re-gridded our maps to the resolution and pixel size of the one from the atomic gas. 


\begin{table*}
\centering
\caption{Data used in this study.}
\label{table:data}
\begin{tabular}{@{}lcccc}
\hline\hline
Wavelength & FWHM & Coverage area &Telescope
& Ref. \\
\hline
%500 \um & 36\arcsec & $Herschel$ & \cite{Fritz12} \\
%250 \um & 18\arcsec.2 & $Herschel$ & \cite{Fritz12} \\
1550~\AA & 4\arcsec.5 & 5\degr $\times$ 5\degr &GALEX & \cite{Martin05}\\  %:PB: GALEX, KPNO, IRAM should not be in $$
2250~\AA & 6\arcsec & 5\degr $\times$ 5\degr &GALEX & \cite{Martin05}\\%PB (they are abbreviations)
6570~\AA  & 1\arcsec & 2\arcmin.2 $\times$ 0\degr.6 &KPNO& \cite{Massey07}\\
3.6~\um & 1\arcsec.7 & 3\degr.7 $\times$ 1\degr.6 &\Spitzer & \cite{Barmby06} \\ %PB: write {\it Spitzer}, {\it Herschel} %PB because those are both missions and names
8~\um & 1\arcsec.9 & 1\degr $\times$ 3\degr &\Spitzer & \cite{Barmby06} \\ % PB: of people - italics means the mission, not
24~\um & 6\arcsec & 1\degr $\times$ 3\degr &\Spitzer & \cite{Gordon06} \\ %PB: the person
70~\um & 18\arcsec & 1\degr $\times$ 3\degr &\Spitzer & \cite{Gordon06} \\
160~\um & 12\arcsec & 5\degr.5 $\times$ 2\degr.5 &\Herschel & \cite{Fritz12} \\
350~\um & 24\arcsec.9 & 5\degr.5 $\times$ 2\degr.5 &\Herschel & \cite{Fritz12} \\
2.6~mm & 23\arcsec & 2\degr $\times$ 0\degr.5 &IRAM 30-m & \cite{Nieten06}\\
21~cm & 60\arcsec $\times$ 90\arcsec & 5\degr.2 $\times$ 1\degr.5 &DRAO & \cite{Chemin09}\\
\hline
\end{tabular}
\end{table*}
%PB: in general it's good to write number~unit, ie 2750~\AA /Done
%PB: this means that number and unit don't end up on separate lines


Several different observational datasets are used to compute properties of M31 for testing star formation laws.
One method for calculating the SFR uses FUV emission as a tracer of the star-forming regions: for this we
use the FUV image of M31 as observed by the Galaxy Evolution Explorer (GALEX; \cite{Martin05}). 

Our other method to calculate the SFR involved using a combination of \halpha and 24 $\mu$m emission. \citet{Massey06, Massey07} mapped M31 in broad and narrow bands, including \halpha, as part of the Nearby Galaxies Survey using the KPNO 4-m telescope. A detailed description of how our \halpha map is made can be found in Appendix~\ref{app:halpha}.

%\subsection{IR and sub mm Data}
Calculations of the SFR using the total infrared emission requires observations at 8, 24, 70, and 160~\um, made
with the {\em Spitzer} \citep{Werner04} and {\em Herschel} \citep{Pilbratt10}  space telescopes. 
Observations with the Infrared Array Camera (IRAC; \citep{Fazio04}) are made in four channels (3.6, 4.5, 5.8, and 8~\um); IRAC observations of M31 were obtained by \cite{Barmby06}, covering a region $3\degr.7 \times 1\degr.6$. 
The Multiband Imaging Photometer for \Spitzer (MIPS) observed M31 at 24, 60, and 160 \um and covered a region $\sim 1\degr \times 3\degr$ \citep{Gordon06}. {\em Herschel} observations of M31 were made with with PACS \citep[Photodetector Array Camera and Spectrometer;][]{Poglitsch10}  at 100 and 160~\um and SPIRE  \citep[Spectral and Photometric Imaging Receiver;][]{Griffin10} 
at 250, 350, and 500~\um, covering a region  $\sim 5\degr.5 \times 2\degr.5$ \citep{Fritz12}.
The IRAC 8, MIPS 24, MIPS 70, and PACS 160 images were used to calculate M31's total infrared emission. 
The IRAC 3.6~\um band image was used to calculate the stellar mass. 

%\subsection{Radio Emission}
Gas densities in galaxies are derived from observations at millimetre and centimetre radio wavelengths.
Emission from the $^{12}$CO (J:$1\rightarrow0$) line observed on-the-fly with the IRAM 30-m telescope \citep{Nieten06} was used to calculate the H$_2$ column density of M31's interstellar medium. This map covers a $2\degr \times 0\degr.5$ region with an angular resolution of $23\arcsec$ and has the smallest coverage of the galaxy among the maps we used in this paper. In this study, we also used a 21-cm emission map of the atomic gas (HI) in M31 from \cite{Chemin09}. This map was made using the synthesis telescope at the Dominion Radio Astrophysical Observatory and covers a $5\degr.1 \times 1\degr.5$ region with an angular resolution of $60\arcsec \times 90\arcsec$.


\section{Measuring the components of Star Formation Laws}
Before studying the star formation laws, we need to calculate the components of these laws. We used different data and calibrations to produce the SFR maps, the stellar mass surface density and the gas mass surface density. To use this calibrations, we used M31 distance to be 0.78~Mpc~\cite{McConnachie05}, and all the total values recorded in this paper are up to~23~kpc from centre of the galaxy. In this section we are showing how we measured every ingredients using available data, and calibrations. 

\subsection{Stellar Initial Mass Function}%I know it is too basic. But I couldn't decide which part should I delete.
\label{sec: imf}

% edited version, 20150715
In order to determine star formation rate and stellar mass from calibrated photometric data, it is necessary to know the IMF. 
Stellar population synthesis models make different assumptions about the IMF, often either  \cite{Salpeter55},
$ dN / d \log M \propto M^{-3.5 }$ over the full range of masses, or \cite{Kroupa01} $ dN / d \log M \propto M^{-\Gamma }$ 
with $\Gamma=-2.3$ at low masses and  $\Gamma=-3.3$.

%%% Sahar_ans: dN/d log M = mdN/dM so your number should be correct.
For the same total mass, the Salpeter and Kroupa IMFs produce almost the same amount of high-mass stars, but the Salpeter IMF produces more low-mass stars.  \cite{Calzetti12} showed the impact of different IMF assumptions on SFR calibrations. Adopting a modified \cite{Kroupa01} IMF, with its maximum stellar mass set to 30 M$_{\odot}$ instead of 100 M$_{\odot}$, changes the SFR calibration constant by factors 1.4 to 5.6 for different SFR indicators. Changes in the IMF mostly affect the calibration based on \halpha luminosity, because a significant amount of the ionizing photons comes from stars more massive than $\sim 20$~M$\sun$. 

In this project we used values based on the \cite{Kroupa01} IMF as is standard in the field, although discussion continues on whether the IMF is universal \citep{Bastin10}. Where necessary we converted from SFR calibration values based on the Salpeter IMF by multiplying by 1.61  \citep{Madau14}.

%One of the most basic and difficult questions that a complete theory of the star formation must answer is the initial mass function (IMF). In order to determine SFR and stellar mass, from calibrated photometry data, it is necessary to know the IMF. Most problem in modern astrophysical studies regarding origin and evolution of stars and galaxies can be solved by varying an IMF \citep{Bastin10}. \cite{Salpeter55} empirically derived the IMF from the observed stellar luminosity function from the solar neighbourhood \citep{Shu87}. He introduced a power law IMF of the form:

%\begin{equation}
%\label{equ: salp}
%\Phi (\log M) = dN / d \log M \propto M^{-\Gamma }
%\end{equation} 
%where M is the mass of a star and N is the number of stars in some logarithmic mass range $\log M + d\log M$. 

%It was recognised that the using the single power low IMF over all stellar masses is not a correct assumption. Since then many groups have studied the IMF and find the different power laws for it \citep{Bastin10}. Most SFR calibrations have the same assumption that the IMF is constant across all environments and adopted from the standard IMF of \cite{Kroupa01}:

%\begin{align}
%\chi (M) = dN/dM = A M^{-1.3}    \quad    (0.1 \le M(M{\odot}) \le 0.5)\\                  
 %          = 0.5 A M^{-2.3}    \quad    (0.5 \le M(M{\odot}) \le 100)
%\end{align}
%where $\chi(M)$ is the number of stars between mass $M$ and $M+dM$. Although the Salpeter IMF produces more low-mass stars then the Krupa IMF, but they produce almost the same amount of the high mass stars. therefore, the SFR calibrations, which mostly trace massive stars in galaxies, based on the Kroupa IMF and Salpeter IMF can be converted together by multiplying the calibration$^\prime$s constant by $ \sim 1.6$ \citep{Calzetti12}. For co \cite{Bastin10} reviewed  the IMF  by using many observational results showed that the IMF is constant and universal, although there are uncertainly in different mass bins especially at the high mass end.

 %Carefully measuring the IMF is a key to understanding the SFR. In distant galaxies, only massive stars can be detected, therefore, IMF must be measured to include all stellar mass ranges to SFR calibrations. To show the impact of the different IMF assumption on the SFR calibrations, \cite{Calzetti12} used two different IMFs and derived the SFR calibrations with these new assumptions. Adopting a modified \cite{Kroupa01} IMF, which its maximum stellar mass set to 30 M$_{\odot}$ instead of 100 M$_{\odot}$, changes the SFR calibration constant by factors 1.4 to 5.6 for different SFR indicators. Changes in the IMF mostly affect the calibration based on \halpha luminosity, because a significant amount of the ionizing photons comes from the stars more massive than $\sim 20$~M$\sun$. In this project we used \cite{Kroupa01} IMF.

%----------------------------------------------------------------------------------------
%STAR FORMATION RATE
%----------------------------------------------------------------------------------------
%----------------------------------------------------------------------------------------
\subsection{Star Formation Rate}
\label{sec:sfr}
\subsubsection{FUV plus 24 $\mu$m Star Formation Rate}

In this project the first star formation map was produced using a calibration of a combination of FUV emission and 24 \um emission introduced by \cite{Hao11}. Since the peak of the emission of massive and young stars (O-B and A type) is in the UV part of the spectrum, this wavelength is one of the most commonly used star formation rate indicators \citep[e.g.][]{Kennicutt89}. The UV emission traces recently formed stars ($\sim 100$ Myr) \citep[e.g.][]{Kennicutt98a, Calzetti05}. However, the downside of only using the FUV light as a tracer of the star-forming region is that this emission is very sensitive to dust extinction. On the other hand, 24 \um emission is dominated by emission from dust heated by UV photons from young and hot stars. This emission is sensitive to the star formation timescale of $\la10$ Myr \citep{Calzetti07}. The FUV band of \Galex and the 24 \um band of \Spitzer data were used to measure the SFR calibrated by \cite{Hao11} using nearby galaxies and the \cite{Kroupa01} IMF as:
\begin{equation}
\label{equ: fuvplus24}
SFR =4.46\times10^{-44}[L(FUV)_{obs}+3.89L(24\mu m)],
\end{equation}
\noindent where SFR is in M$_{\odot}$yr$^{-1}$, and L(24$\mu$m) and L(FUV)$_{obs}$ are in erg~s$^{-1}$. L(FUV)$_{obs}$ is the luminosity of the galaxy in the FUV emission. The subscript "obs" indicates that the FUV emission is not corrected for extinction. Nevertheless, this emission was corrected for the foreground stars' emission using a method introduced by \cite{Leroy08}. We masked out pixel values in the FUV$_{obs}$ map for which I$_{NUV}$/I$_{FUV}$ $>$ 15, assuming that those pixels are dominated by foreground emission. Then we smoothed our map to have the same angular resolution and pixel size as the MIPS 24 $\mu$m, and masked the same regions in the galaxy for the 24 \um map. Figure \ref{fig:sfr,fuv} shows the SFR(FUV+24 $\mu$m) map. We added all the pixels in the map and calculated the total SFR to be $0.31\pm 0.04$~M$_{\odot}$~yr$^{-1}$.

%PB: I like this figure! Looks nice.

\begin{figure*}
    \centering
    \begin{subfigure}[b]{1\textwidth}
        \centering
        \includegraphics[width=164mm]{sfr_fuv.eps}
        \caption{SFR(FUV+24 $\mu$m)}
        \label{fig:sfr,fuv}
    \end{subfigure}
    \hfill
    \begin{subfigure}[b]{1\textwidth}
        \centering
        \includegraphics[width=164mm]{sfr_halpha.eps}
        \caption{SFR(H$_{\alpha}$)}
        \label{fig:sfr_halpha}
    \end{subfigure}
    \hfill
    \begin{subfigure}[b]{1\textwidth}
        \centering
        \includegraphics[width=164mm]{sfr_fir.eps}
        \caption{SFR(TIR)}
        \label{fig:sfr,fir}
    \end{subfigure}
    \caption{SFR map from a combination of FUV + 24 $\mu$m emission (top), \halpha and 24 $\mu$m emission (middle), and total infrared emission (bottom). All the SFR maps here, have the angular resolution of the 18\arcsec and pixel size of 9.85\arcsec, same as MIPS 70 $\mu$m map; highest angular resolution among the components of the SFR}
    \label{fig:sfrs}
\end{figure*}

\subsubsection{\halpha plus 24 \um Star Formation Rate}
\label{sec:sfr_halpha}

The SFR can also be calculated using a combination of \halpha and \Spitzer  24 \um maps. \halpha emission of galaxies is dominated by light emitted from young and massive stars. Star formation timescales traced by this emission is $\sim 6-10$ Myr \citep[e.g.][]{Kennicutt09, Calzetti13}. Therefore, \halpha alone could be used to calibrate the SFR \citep[e.g.][]{Osterbrock06, Kennicutt09}. However, \halpha emission is very sensitive to dust extinction. Therefore, the combination of the \halpha and 24\um of is considerable choice for calculating the SFR. We used a calibration introduced by \cite{Kennicutt09} to measure the SFR:
\begin{equation}
\label{equ: halphaplus24_g}
SFR = 5.5 \times 10^{-42}[L(H{\alpha})_{obs} + 0.020L(24\mu m)],
\end{equation}
\noindent where L(H${\alpha}$)$_{obs}$ is the observed \halpha luminosity without correction for internal dust attenuation, given in the unit of erg~s$^{-1}$. L(24$\mu$m) is the $24\mu$m IR luminosity in erg~s$^{-1}$, and SFR is in M$_{\odot}$yr$^{-1}$. It was indicated that the formulation above can only be used at the global situation i.e., the total SFR.

\cite{Calzetti07} introduced a new calibration using the same band passes in a case of local regions (Figure~\ref{fig:sfr_halpha}):

\begin{equation}
\label{equ: halphaplus24_l}
SFR = 5.5 \times 10^{-42}[L(H{\alpha})_{obs} + 0.033L(24\mu m)],
\end{equation}

\noindent the units here are the same as equation~\ref{equ: halphaplus24_g}. Equation~\ref{equ: halphaplus24_l} is useful for regions less than $\sim 1$ kpc. For testing the SFR laws in local regions we used equation~\ref{equ: halphaplus24_l}. Total SFR calculated from equation~\ref{equ: halphaplus24_g} is $0.35 \pm 0.01$~M$\sun$~yr$^{-1}$, which is in good agreement with previous studies.

\subsubsection{Total Infrared Star Formation Rate}
\label{sec:sfr_fir}
The total infrared emission of a system can also be used as a tracer of the SFR. Dust absorbs radiation from hot and young stars and re-emits it at infrared wavelengths; however, there is no one-to-one mapping between IR and UV emission. Therefore, integrating over the full wavelength range of the IR part and calculating the TIR luminosity is a better tracer of the SFR instead of IR single-band emission \citep{Calzetti13}. \cite{Calzetti13} assumed that stellar bolometric emission is completely absorbed and re-emitted by dust, i.e., $L_{star}(bol) = L(TIR)$, and calibrated the TIR luminosity of a system to calculate the SFR for a stellar population undergoing constant star formation over 100 Myr:

\begin{equation}
\label{equ:sfr_fir}
SFR(\rm TIR) = 2.8 \times10^{-44}L(\rm TIR),
\end{equation}
\noindent where SFR(TIR) and L(TIR), the star formation rate calculated from TIR emission and TIR luminosity, are in M$_{\odot}$~yr$^{-1}$ and erg~s$^{-1}$, respectively. Part of the dust emission in galaxies (specially in M31) comes from the dust heated by the cosmic background radiation \citep[e.g.][]{Dole06, Calzetti13, Mattsson14}, hence, using equation~\ref{equ:sfr_fir} overestimates the SFR in M31.  

TIR luminosity can be measured from the integration of the IR part of a spectral energy distribution (SED) of a galaxy or combining photometry data at different IR wavelengths. As a second approach, \cite{Draine07} modelled \Spitzer data and calibrated IR single-band photometry data to calculate the TIR luminosity. \cite{Boquien10} tested and modified this calibration as:
\begin{equation}
 \label{equ: TIR}
L(\rm TIR) = 0.95L(8) + 1.15L(24) + L(70) + L(160),
\end{equation}
\noindent where L(8), L(24), L(70), and L(160) are luminosities of the galaxy in 8 $\mu$m, 24 $\mu$m, 70 $\mu$m,and 160 \um in units of erg~s$^{-1}$. We used results from equation~\ref{equ: TIR} in the equation~\ref{equ:sfr_fir} to calculate our last map of the SFR of M31 (Figure~\ref{fig:sfr,fir}). Using this method, we calculated the total SFR by adding all the pixels to be $0.40 \pm 0.04$~M$_{\odot}$~yr$^{-1}$. Table~\ref{table:sfr} compares the total SFR from literature and the present work. 


\begin{table*}
\begin{minipage}{100mm}
\caption{Comparison of the total star formation rate of M31}
\label{table:sfr}
\begin{tabular}{@{}lcc}
\hline\hline
Ref.&Method&SFR(M$_{\odot}$yr$^{-1}$) \\
\hline
%Current work&FUV and 24 $\mu$m&0.31 $\pm$ 0.04\\
%Current work&\halpha and 24 $\mu$m&0.35 $\pm$ 0.01\\
%Current work&TIR luminosity&0.4 $\pm$ 0.04\\
Current work&FUV and 24 $\mu$m&0.31 \\
Current work&\halpha and 24 $\mu$m&0.35 \\
Current work&TIR luminosity&0.4\\
\cite{Ford13}&FUV and 24 $\mu$m&0.25\\
\cite{Ford13}&TIR luminosity&0.48-0.52\\
\cite{Azimlu11}& \halpha and 24 $\mu$m&0.34\\
\cite{Azimlu11}&Extinction corrected \halpha&0.44\\
\cite{Tabatabaei10}&Extinction corrected \halpha&0.27--0.38\\
\cite{Barmby06}&Infrared 8\um luminosity& 0.4\\
\hline
\end{tabular}
\end{minipage}
\end{table*}


%----------------------------------------------------------------------------------------
%ISM
%----------------------------------------------------------------------------------------


\subsection{Gas Surface Density}
\label{sec:ISM}

 The molecular form of hydrogen is very hard to detect. Therefore, CO (usually the $J(1\rightarrow 0)$ rotational transition, observed at 2.6 mm) is used as a tracer of the mass of the molecular cloud dominated by molecular hydrogen \citep[see, for example,][] {Sanders84}. Higher rotational transitions of CO can be used as tracers as well; however, they are not as common as $J(1\rightarrow 0)$. Equation \ref{equ:conversion} shows the relation between CO emission and H$_2$ column density:

\begin{equation}
\label{equ:conversion}
\rm N_{H_2}/\rm cm^{-2} = X_{CO} \times I_{CO}/\rm K km s^{-1} ~,
\end{equation}

\noindent where X$_{CO}$ is the conversion factor (also known as the X-factor), I$_{CO}$ is CO intensity in  K km s$^{-1}$ and N$_{H_2}$ is molecular hydrogen column density. The X-factor could be different in regions within a galaxy due to the different metallicities \citep{Wilson95, Bosselli02, Bolato13}. In the case of M31, different values of X-factor in the range of ($1-5.6 \times 10^{20}$) were adopted in previous studies \citep[e.g.][]{Ford13, Bolato13, Leroy11, Bolato08, Nieten06}; however, since any constant difference in X-factor leads to horizontal changes in a plot of log(SFR) versus log($\Sigma_{\rm {gas}}$) and does not affect our power indices for molecular gas, we chose $X_{\rm {CO}}= 2 \times 10^{20}$ which is the same value that is adopted in many other works \citep[e.g.][]{Ford13, Smith12}. Since HI mass dominates the H$_2$ mass in M31, the constant X-factor has a negligible effect on the total gas mass. We assumed each hydrogen molecule has a mass of 2.34 mass of H, and calculate the H$_2$ mass density.

The total gas mass was calculated from:
\begin{equation}
\label{equ:total_gas}
\rm M_{\rm \bld {total \, gas}} = 1.36[M_{\rm HI}+M_{H_2}],
\end{equation}
\noindent where the factor of 1.36 is a constant that takes into account the contribution of He and other heavier elements to the total gas mass, M$_{\rm HI}$ is the mass of neutral hydrogen that was obtained from 21 cm observations (the map, which we got from \cite{Chemin09} has a unit of number density which we used it to calculate mass density of HI), and M$_{H_2}$ was calculated from equation~\ref{equ:conversion} in units of M$_{\sun}$/pc$^2$.

%----------------------------------------------------------------------------------------
%STELLAR MASS
%----------------------------------------------------------------------------------------
\subsection{Stellar Mass}
\label{sec:starmass}
\begin{figure*}
\centering
\includegraphics[width=164mm]{star.eps}
\caption{The stellar surface density map is produced using IRAC 3.6~$\mu$m data and its calibration is presented in equation~\ref{equ:eskew}.}
\label{fig:stellarmass}
\end{figure*}

Near IR bands, such as: 3.6$\mu$m, 4.5\um and K band emission, are almost insensitive to the emission from the young stellar population, dust absorption and emission, which make them good tracers of the mass of the older stellar populations \citep[e.g.][]{Elmgreen84, Eskew12, Zhu10}. We calculated the stellar mass within M31 using calibration of the 3.6 \um flux by \cite{Eskew12}. They used data from the Large Magellanic Cloud and introduced an empirical relation between stellar mass and flux:

\begin{equation}
\label{equ:eskew}
M _{\star}= 10^{5.97} F_{3.6}\left(\frac{D}{0.05}\right)^2
\end{equation}

\noindent where M$_{\star}$ is the stellar mass in solar masses, $F_{3.6}$ is 3.6 \um flux in Jy, and $D$ is the distance to the galaxy in Mpc. This calibration is based on the Salpeter \cite{Salpeter55} IMF. Since the Kroupa IMF is adopted for the SFR maps, we corrected the calibration for the Kroupa IMF (see section~\ref{sec: imf}). We should note that \cite{Eskew12} have two calibrations for calculating the stellar mass in their paper, one is equation~\ref{equ:eskew} which only uses 3.6 \um emission and the other is using both 3.6 \um and 4.5 \um emission. We noticed that the other calibration can only be used in the case of total stellar mass due to the fact that the correlation shows a non-linear dependency of stellar mass to the $F_{3.6 \mu m}$ and $F_{4.5 \mu m}$, though this was not mentioned in aforementioned paper. 
Figure~\ref{fig:stellarmass} shows the stellar mass map for M31 using IRAC 3.6 \um in original angular resolution and pixel size of the IRAC 3.6 \um. We had a coverage map of the IRAC 3.6 \um band; and we used only regions with coverage of more than 2, and replaced the rest with zero M$_{\odot}$ which are shown as dark blue pixels in this figure. 
Using this method we calculated the total stellar mass of the galaxy by adding all pixel values and found it to be $6.9 \times 10^{10}$M$_{\odot}$ $\pm$ 6$\%$. This result is in fairly good agreement with the results of \cite{Tamm12}. They calculated that stellar mass in M31 is $(10-15) \times 10^{10}$M$_{\odot}$, 56$\%$ of which is in the disk with the rest in the bulge of the galaxy.



%----------------------------------------------------------------------------------------
%Metallicity
%----------------------------------------------------------------------------------------
\subsection{Metallicity}
\label{sec:metal}
 
Metallicity of galaxies can be estimated by measuring abundances of heavier elements of the ISM in galaxies. Assuming gas phase oxygen abundance, $[$O$/$H$]$, is proportional to abundances of heavier elements and [O/H] would be a gauge to determine metallicity \citep[e.g.][]{McGaugh91, Zaritsky94}. Results from modelling dust in the MW and nearby galaxies showed that the dust to gas mass ratio depends on $[$O$/$H$]$ and, consequently, on metallicity \cite{Draine07}. \cite{Draine14} assumed that depletion of gas elements from the gas to diffuse ISM in M31 is similar to the MW, and proposed that: 

 \begin{equation*}
 \label{equ: metal}
\frac{M_d}{M_H}=0.0091 \frac{Z}{Z_{\odot}},
 \end{equation*} 
 \noindent where M$_d$ and M$_H$ are the masses of dust and hydrogen atom, Z is metallicity of M31 and Z$_{\odot}$ is solar metallicity. \cite{Draine14} produced two dust maps of M31 using MIPS 160 resolution (angular resolution = 39\arcsec; 18\arcsec $\times$ 18\arcsec pixels) and SPIRE 350 resolution (angular resolution = 24\arcsec.9; 10\arcsec $\times$ 10\arcsec pixels), which are publicly available. We used dust map with SPIRE 350 resolution, convolved and re-gridded it to have the same resolution and pixel size for the HI map, then we divided this dust mass by the mass of hydrogen in M31. However, in a private correspondence with the author, Draine pointed out that based on the Plank observation of the MW dust \cite{Tauber10}, the dust mass surface density map estimated for M31 by \cite{Draine14} may be systematically high by a factor of 2. Therefore, we divided the dust map by 2 to account for this correction. We then generated the metallicity map of M31 by using the dust map in combination with equation~\ref{equ: metal}. \cite{Draine14} showed that with increasing distance from the centre of M31, dust to H mass ratio declines monotonically and concluded that metallicity also behaves in the same way. Their fitting results showed that M31 can be divided into three regions ($R< 8\kpc$, $8\kpc < R < 18\kpc$, and $18\kpc < R \la 25\kpc$) where dust to H mass ratio (or metallicity) has different correlations with distance from the centre of the galaxy.
  
%----------------------------------------------------------------------------------------
%SFR LAWS
%----------------------------------------------------------------------------------------
\section{Scaling SFR}

\subsection{Fitting method}
\label{sec:fitting}


\begin{figure*}
\centering
\includegraphics[width=164mm]{Results_for_all_regions.pdf}
\caption{Results from fitting the Kennicutt-Schmidt law to data from the entire galaxy using the pixel by pixel method. The points in the plots represent different pixel sizes due to differences in the resolution of the H$_2$ and HI maps. Each point on the plots where the surface density of H$_2$ is used as a tracer of gas mass represents a region of size $\sim$30~pc while each point for which the surface density of HI or total gas mass is a tracer represents a region of size $\sim$155~pc. Solid lines shows the best fit, using the max value of the ranges in table~\ref{table:res}
} 
\label{fig:ks,all}
\end{figure*}

Producing the SFR, gas mass, stellar mass, and metallicity maps provides enough data to examine and compare the K-S law and the extended Schmidt law. To create the surface density map, we divided the value of each pixel by its area in pc$^2$. We investigated the K-S law and the extended Schmidt law by using pixel by pixel method in the entire galaxy as well as comparing the laws in three elliptical regions, assuming the position angle 38$\degr$, inclination of 77$\degr$ for M31 with distance from the centre of the galaxy the same as the ones described in section~\ref{sec: metal}. As mentioned in section~\ref{sec:data}, since the H$_2$ gas mass map has higher angular resolution and smaller pixel size than the HI gas mass map, our final results have two sets of different pixel sizes. Consequently, any plots with surface density of H$_2$ as a tracer of the gas mass will have more data points. 


Equations~\ref{equ:ks_org}~and~\ref{equ:es_org} can be re-written in logarithmic scale to obtain two linear equations:

\begin{subequations}
\begin{equation}
\label{eq:sfr_law_ks_log}
\log_{10} \eqsigmasfr = N~\log_{10} \eqsigmagas + A,
\end{equation}
\begin{equation}
\label{eq:sfr_law_es_log}
\log_{10} \eqsigmasfr = \eqnprime~\log_{10} \eqsigmagas + \beta~\log_{10}\eqsigmastar  + A^\prime,
\end{equation}
\end{subequations}
\noindent where, {\it N}, $\eqnprime$, $\beta$, A, and A$^\prime$ are free fitting parameters. The units in the above equations are the same as those in equations ~\ref{equ:ks_org} and ~\ref{equ:es_org}.

We found the free parameters by applying the hierarchical Bayesian linear regression method as described in \cite{Shetty13}. Shetty and his colleagues used a Bayesian linear regression approach to develop a new method to find the K-S law parameters, considering the measurement uncertainties as well as hierarchical data structure. We used the same technique as \cite{Shetty13} to estimate the KS law parameters. For the extended Schmidt law, we extended the code such that instead of using  simple hierarchical Bayesian linear regression, which is used for the K-S law, it uses multiple hierarchical Bayesian linear regression. In this case, we were able to examine the effect of the stellar mass in the SFR as shown in equation ~\ref{eq:sfr_law_es_log}. % It doesn't sound good!

Uncertainties in each observable quantities were measured by following the method described in \cite{Kennicutt07}. We used the quadratic sum of the variance of the local background for each luminosity map from the original pixel size images, Poisson noise of images, and calibration uncertainties. 

We also tested each law on three different regions which were chosen in the same manner as explained in~\ref{sec:metal}. Applying the SFR laws in these regions provides a tool to consider the effect of the distance from the centre of the galaxy on SFR laws such as metallicity, gas mass, stellar mass gradients and changing the ratio of H$_2$ to HI. 


\subsection{Star formation laws}
\label{sec: sfl}

We made three SFR maps, three gas mass density maps and a stellar mass density map of whole galaxy. We applied both laws on the whole galaxy using each two combination of SFR and gas mass density maps, along with applying the laws on three different regions.\cite{Kennicutt12} reviewed star formation in the MW and the nearby galaxies and described that there might be a breakdown on SFR laws for small size regions. Therefore, we re-gridded all the maps to have a pixel size equivalent to 750~pc, and repeated all of our processes on these new maps (table~\ref{table:res750}). Additionally, we used these results to see if there is any correlation between the SFR laws and metallicity, as it is going to be described in ~\ref{sec:fitting}.
 
\begin{figure*}
\includegraphics[width=164mm]{Results_all_3_regs.pdf}
\caption{Same as Figure~\ref{fig:ks,all}, but in this figure we separated pixels from different regions in the galaxy by their colours. The regions with $R< 8\kpc$, $8\kpc < R < 18\kpc$, and $18\kpc < R \la 25\kpc$ are shown in red, green and blue, respectively.}
\label{fig:ks,regs}
\end{figure*}

%PB: no space, so SFR(TIR)
Figure~\ref{fig:ks,all} shows the pixel by pixel fitting of the K-S law on the whole galaxy. First row plots are the surface density of the SFR(TIR), which is calculated using  the TIR emission form section ~\ref{sec:sfr_fir}, versus the gas surface density traced by only the molecular gas (H$_2$),  only atomic gas (HI), and total gas from right to left, respectively. The only difference between the upper and the other rows is that in the lower rows the SFR indicator is the FUV plus 24 $\mu$m (second row), and \halpha plus 24 $\mu$m emission (third row). Figure~\ref{fig:ks,regs} shows the same data as figure~\ref{fig:ks,all}. However, in this figure we separated pixels from different regions in the galaxy by their colours. The regions with R less than 8$\kpc$, $8\kpc < $R $< 18\kpc$, and $18\kpc <$ R $\la 25\kpc$ are shown in red, green and blue, respectively. There is a deviation in middle panel plots of figure~\ref{fig:ks,all} which shows region with high star formation rate and low HI gas. It is clear that these deviation belongs to regions further than 8~kpc from centre (figure~\ref{fig:ks,regs}). HI surface density profile of M31 \citep[See figure 16 in.][]{Chemin09} shows several local minimum from centre of M31 to 30~kpc distance which could be results of the HI wrap. Some of these minimums are around the rings area of M31, which has higher SFR regarding to the other regions in the galaxy and causes these deviations. Since the total gas mass is calculated by adding HI and H$_2$ gas mass, we can see these deviation on right panel of figure~\ref{fig:ks,all}, too. However, since we have higher H$_2$ emission in those regions compare to other regions of M31 these deviations are less than the ones in the middle panel.  

We took the similar approach for the extended Schmidt law. Figures~\ref{fig:es,all}-~\ref{fig:es,regs,halpha,tot} show the results of the pixel by pixel fitting of the extended Schmidt law. For the extended Schmidt law fitting, we plotted our results in 3D to illustrate the relationship between the surface density of the SFR, the gas mass surface density, and stellar mass density more clearly. In these series of plots, X-axis is the gas mass surface density, either molecular gas (H$_2$), the atomic gas (HI), or the total gas, Y-axis is the SFR(TIR) or SFR(FUV + 24 $\mu$m), and Z-axis is the stellar mass surface density. Shadows of our data on each surface are also plotted, to have a more clear picture of correlations between components. Shadows on the X-Y surface is the same as figure~\ref{fig:ks,all}. 

\begin{figure*}
    \centering
    \begin{subfigure}[b]{0.3\textwidth}
        \centering
        \includegraphics[width=\textwidth]{es_tot_fir_vs_h2_22.png}
        \caption{SFR,TIR vs surface density of H$_2$}
        \label{fig:es,all,fir,h2}
    \end{subfigure}
    \hfill
    \begin{subfigure}[b]{0.3\textwidth}
        \centering
        \includegraphics[width=\textwidth]{es_tot_fir_vs_hi2.png}
        \caption{SFR,TIR vs surface density of HI}
        \label{fig:es,all,fir,hi}
    \end{subfigure}
    \hfill
   \begin{subfigure}[b]{0.3\textwidth}
        \centering
        \includegraphics[width=\textwidth]{es_tot_fir_vs_tot2.png}
        \caption{SFR,TIR vs surface density of total gas}
        \label{fig:es,all,fir,tot}
    \end{subfigure}
    \hfill
     \begin{subfigure}[b]{0.3\textwidth}
        \centering
        \includegraphics[width=\textwidth]{es_tot_fuv_vs_h22.png}
        \caption{SFR,FUV+24 $\mu$m vs surface density of H$_2$}
        \label{fig:es,all,fuv,h2}
    \end{subfigure}
     \hfill
   \begin{subfigure}[b]{0.3\textwidth}
        \centering
        \includegraphics[width=\textwidth]{es_tot_fuv_vs_hi2.png}
        \caption{SFR,FUV+24 $\mu$m vs surface density of HI}
        \label{fig:es,all,fuv,hi}
    \end{subfigure}
    \hfill
    \begin{subfigure}[b]{0.3\textwidth}
        \centering
        \includegraphics[width=\textwidth]{es_tot_fuv_vs_tot2.png}
        \caption{SFR,FUV+24 $\mu$m vs. surface density of total gas}
        \label{fig:es,all,fuv,tot}
    \end{subfigure}
    \hfill
     \begin{subfigure}[b]{0.3\textwidth}
        \centering
        \includegraphics[width=\textwidth]{es_tot_halpha_vs_h22.png}
        \caption{SFR,H$\alpha$+24 $\mu$m vs surface density of H$_2$}
        \label{fig:es,all,halpha,h2}
    \end{subfigure}
     \hfill
   \begin{subfigure}[b]{0.3\textwidth}
        \centering
        \includegraphics[width=\textwidth]{es_tot_halpha_vs_hi2.png}
        \caption{SFR,H$\alpha$+24 $\mu$m vs surface density of HI}
        \label{fig:es,all,halpha,hi}
    \end{subfigure}
    \hfill
    \begin{subfigure}[b]{0.3\textwidth}
        \centering
        \includegraphics[width=\textwidth]{es_tot_halpha_vs_tot2.png}
        \caption{SFR,H$\alpha$+24 $\mu$m vs. surface density of total gas}
        \label{fig:es,all,halpha,tot}
    \end{subfigure}
    \caption{The results from fitting the extended Schmidt law on data from whole galaxy using pixel by pixel method. The plots show the SFR vs. the surface density of gas, and z-axis is the surface density of the stellar mass. Each figure shows different combinations of the SFR tracer and gas mass tracer results. As in Figure~\ref{fig:ks,all}, the analyses use different pixel sizes. Each point in the plots with the surface density of H$_2$ as a tracer of gas mass represents a region of size $\sim$30~pc and each point in plots with surface density of HI or total gas mass represent a region of size $\sim$155~pc. Solid lines shows the best fit, using the max value of the ranges in table~\ref{table:res}}
    \label{fig:es,all}
\end{figure*}



\begin{figure*}
\centering
\includegraphics[width=162mm]{tir_vs_tot_3_reg_extendS.png}
\caption{Fitting result from the SFR(TIR) vs total gas. Same as Figure~\ref{fig:es,all,fir,tot}, but in this figure we denote pixels from different regions in the galaxy by their colours. The regions with $R< 8\kpc$, $8\kpc < R < 18\kpc$, and $18\kpc < R \la 25\kpc$ are shown in red, green and blue, respectively.}
\label{fig:es,regs,fir,tot}
\end{figure*}

\begin{figure*}
\centering
\includegraphics[width=162mm]{fuv_vs_tot_3_reg_extendS.png}
\caption{Fitting result from the SFR(FUV + 24 $\mu$m) vs total gas. Same as Figure~\ref{fig:es,all,fuv,tot}, but in this figure we denote pixels from different regions in the galaxy by their colours. The regions with $R< 8\kpc$, $8\kpc < R < 18\kpc$, and $18\kpc < R \la 25\kpc$ are shown in red, green and blue, respectively.}
\label{fig:es,regs,fuv,tot}
\end{figure*}

%\begin{figure*}
%\centering
%\includegraphics[width=162mm]{halpha_vs_tot_3_reg_extendS_97perse.png}
%\caption{Fitting result from the SFR(H$\alpha$ + 24 $\mu$m) vs total gas. Same as Figure~\ref{fig:es,all,halpha,tot}, but in this figure we denote pixels from different regions in the galaxy by their colours. The regions with $R< 8\kpc$, $8\kpc < R < 18\kpc$, and $18\kpc < R \la 25\kpc$ are shown in red, green and blue, respectively.}
%\label{fig:es,regs,halpha,tot}
%\end{figure*}


\begin{table*}
\caption{Fitting parameters of the SF laws from applying the Bayesian method, {\it N} is the power index of the K-S law; A is the intercept of the K-S law; \nprime is the gas power index in the extended Schmidt law; $\beta$ is the power index of the stellar component in the extended Schmidt law; and A$^\prime$ is the intercept of the extended Schmidt law}
\label{table:res}
\begin{tabular}{ccccrccrr}
\hline\hline
\multicolumn{1}{c}{\multirow{1}{*}{Region}} & SFR Tracer        & Gas Tracer & {\it N}    & A      & \nprime & $\beta$ & A$^\prime$ \\
\hline
\multicolumn{1}{c}{\multirow{9}{*}{Whole Galaxy}} & TIR               & H$_2$ only & 1.05 & -8.98  & 0.97    & 0.35    & -9.53      \\
 & TIR               & HI only    & 1.19 & -9.69  & 0.79    & 0.75    & -10.56     \\
 & TIR               & Total gas  & 0.82 & -9.84  & 0.51    & 0.56    & -10.40     \\
 & FUV + 24 $\mu$m       & H$_2$ only & 1.16 & -9.36  & 1.15    & 0.25    & -9.74      \\
 & FUV + 24 $\mu$m       & HI only    & 1.19 & -9.96  & 0.81    & 0.61    & -10.62     \\
 & FUV + 24 $\mu$m       & Total gas  & 0.78 & -10.09 & 0.49    & 0.45    & -10.50     \\
 & H$\alpha$ + 24 $\mu$m & H$_2$ only & 1.16 & -9.26  & 1.07    & 0.38    & -9.86      \\
 & H$\alpha$ + 24 $\mu$m & HI only    & 0.54 & -9.61  & 0.53    & 0.76    & -10.69     \\
 & H$\alpha$ + 24 $\mu$m & Total gas  & 0.54 & -9.82  & 0.37    & 0.59    & -10.55     \\
\hline
\multicolumn{1}{c}{\multirow{9}{*}{R$< 8\kpc$}} & TIR               & H$_2$ only & 1.01 & -8.95  & 0.92    & 0.36    & -6.66      \\
 & TIR               & HI only    & 1.63 & -9.93  & 1.14    & 0.65    & -10.38     \\
 & TIR               & Total gas  & 1.07 & -9.99  & 0.87    & 0.41    & -10.21     \\
 & FUV + 24 $\mu$m       & H$_2$ only & 1.07 & -9.30  & 1.03    & 0.18    & -9.62      \\
 & FUV + 24 $\mu$m       & HI only    & 1.29 & -10.06 & 1.02    & 0.45    & -10.34     \\
 & FUV + 24 $\mu$m       & Total gas  & 0.89 & -10.12 & 0.81    & 0.21    & -10.18      \\
 & H$\alpha$ + 24 $\mu$m & H$_2$ only & 1.14 & -9.23  & 1.03    & 0.41    & -6.47      \\
 & H$\alpha$ + 24 $\mu$m & HI only    & 1.29 & -9.56  & 0.96    & 1.54    & -11.10     \\
 & H$\alpha$ + 24 $\mu$m & Total gas  & 0.90 & -9.99  & 0.65    & 1.27    & -10.88    \\
\hline
\multicolumn{1}{c}{\multirow{9}{*}{$8\kpc < $R $< 18\kpc$}} & TIR               & H$_2$ only & 1.08 & -8.90  & 1.00    & 0.40    & -9.59      \\
 & TIR               & HI only    & 1.16 & -9.58  & 0.84    & 0.77    & -10.53     \\
 & TIR               & Total gas  & 0.79 & -9.77  & 0.50    & 0.58    & -10.38     \\
 & FUV + 24 $\mu$m       & H$_2$ only & 1.21 & -9.28  & 1.15    & 0.31    & -9.82      \\
 & FUV + 24 $\mu$m       & HI only    & 1.19 & -9.88  & 0.85    & 0.63    & -10.61     \\
 & FUV + 24 $\mu$m       & Total gas  & 0.75 & -10.04 & 0.47    & 0.47    & -10.47     \\
 & H$\alpha$ + 24 $\mu$m & H$_2$ only & 1.18 & -9.18  & 1.09    & 0.41    & -9.88      \\
 & H$\alpha$ + 24 $\mu$m & HI only    & 0.56 & -9.49  & 0.72    & 0.81    & -10.67     \\
 & H$\alpha$ + 24 $\mu$m & Total gas  & 0.52 & -9.71  & 0.40    & 0.61    & -10.47  \\
\hline
\multicolumn{1}{c}{\multirow{9}{*}{$18\kpc <$ R $\la 25\kpc$}} & TIR               & H$_2$ only & 2.08 & -8.76  & 1.81    & 1.32    & -5.82      \\
 & TIR               & HI only    & 1.30 & -9.68  & 0.83    & 0.78    & -10.54     \\
 & TIR               & Total gas  & 0.86 & -9.83  & 0.55    & 0.57    & -10.38     \\
 & FUV + 24 $\mu$m       & H$_2$ only & 2.87 & -9.18  & 2.89    & 1.49    & -11.33     \\
 & FUV + 24 $\mu$m       & HI only    & 1.27 & -9.95  & 0.85    & 0.64    & -10.63     \\
 & FUV + 24 $\mu$m       & Total gas  & 0.80 & -10.08 & 0.52    & 0.48    & -10.49     \\
 & H$\alpha$ + 24 $\mu$m & H$_2$ only & 2.86 & -9.05  & 3.12    & 1.30    & -10.86     \\
 & H$\alpha$ + 24 $\mu$m & HI only    & 0.63 & -9.62  & 0.44    & 0.81    & -10.62     \\
 & H$\alpha$ + 24 $\mu$m & Total gas  & 0.60 & -9.84  & 0.39    & 0.66    & -10.52     \\
 \hline
\end{tabular}
\end{table*}



\begin{table*}
\caption{Similar to table~\ref{table:res} but here the fitting performed on the regions with size of 750~pc, and we only showed the results from SFR(FUV+24$\mu$m).}
\label{table:res750}
\begin{tabular}{cccrccrr}
\hline\hline
\multicolumn{1}{c}{\multirow{1}{*}{Region}}  & Gas Tracer & {\it N} & A  & \nprime & $\beta$ & A$^\prime$ \\
\hline
\multicolumn{1}{c}{\multirow{3}{*}{Whole Galaxy}}
 & H$_2$ only & 0.32 & -9.91   & 0.27  & 0.41    &  -10.31  \\
 & HI only    & 1.24 & -11.36  & 0.85  & 0.65    & -11.60     \\
 & Total gas  & 0.89 & -9.94 & 0.52    & 0.49    & -10.48     \\
\hline
\multicolumn{1}{c}{\multirow{3}{*}{R$< 8\kpc$}}
 & H$_2$ only & 0.91 & -9.03  & 0.41   & 0.71   & -9.88      \\
 & HI only    & 1.73 & -10.88 & 1.03    & 1.42    & -10.52     \\
 & Total gas  & 0.88 & -9.94  & 1.45    & 0.83    & -9.11      \\
\hline
\multicolumn{1}{c}{\multirow{3}{*}{$8\kpc < $R $< 18\kpc$}}
 & H$_2$ only & 0.36 & -9.66  & 0.28    &  0.53   & -10.15       \\
 & HI only    & 1.38 & -11.06 & 1.26    & 0.65    & -11.33     \\
 & Total gas  & 0.80 & -10.00 & 0.72    & 0.53    & -10.17     \\
\hline
\multicolumn{1}{c}{\multirow{3}{*}{$18\kpc <$ R $\la 25\kpc$}} 
 & H$_2$ only & 0.34 & -9.47  &  0.34    & 0.75    & -9.72   \\
 & HI only    & 1.46 & -10.73  & 1.05    & 0.84    & -11.44     \\
 & Total gas  & 0.85 & -9.86 & 0.58    & 0.66    & -10.26     \\
 \hline
\end{tabular}
\end{table*}



%----------------------------------------------------------------------------------------
%
%----------------------------------------------------------------------------------------
\section{Discussion}

Table~\ref{table:res} shows the results of the hierarchical Bayesian fitting for the whole galaxy and all three regions. Considering the data and uncertainties for each parameters, we are 95$\%$ confident that our fitting parameters are in given ranges in table~\ref{table:res}. Our results suggest changing SFR tracers doesn't have a significant impact to the determination of the power law index, while changing gas tracer is more important. Therefore, we only discuss the results from the SFR(FUV+24$\mu$m), and in~\ref{table:res750} we only showed results from this SFR map.

\subsection{The K-S law in M31}

As mentioned before, the K-S law in M31 was tested by many groups. In one of the most recent works on M31, \cite{Ford13} investigated the K-S law in six annuli and the global case using the ordinary least squares (OLS) fitting method. To make comparisons between our results easier, we averaged the power indices for their annuli which overlap with our three different regions so we could compare results in the same regions.
\cite{Tabatabaei10} also applied the K-S law to M31 using the OLS fitting method, and have some differences with results from \cite{Ford13}. \cite{Ford13} argued that the main reason for these differences is the cut-off they used to fit the data. 
 
our results for the whole-galaxy fit of the K-S law show that using total gas gives a sub-linear relation between \sigmasfr and \sigmatotalgas. While, for using the H$_{2}$ only gas or HI only gas we find the nearly super-linear relation. These results are different from power indices estimated by \cite{Ford13}, $N=2.03\pm0.04$ in case of total gas and $N=0.6\pm0.01$ in case of H$_{2}$ only gas. \cite{Tabatabaei10} estimated the K-S law power index $N=1.30\pm0.05$ using the total gas and $N=0.96\pm0.03$ using the H$_{2}$ only gas. The reason behind these differences is mostly because of the statical method we chose. The OLS fitting using H$_{2}$ only gas gives us the power index of $N=0.57\pm1e-4$ which is close to \cite{Ford13} results but still different from the results from \cite{Tabatabaei10} which, could be the same reason \cite{Ford13} suggested for discrepancy between their results and results from \cite{Tabatabaei10}. However, results from the OLS fitting with \sigmatotalgas gives the same results as the fitting with the Bayesian method.% Another differences between our calculation with \cite{Ford13} and \cite{Tabatabaei10} is that both of them used the de-projected values for their calculation which, were calculated by multiplying all of their map by factor of $cos(i)$, inclination angle. We didn't include the calculation of effect of the inclination angle in our results. I checked and my reults doesn't changed

 \cite{Bigiel08} argued that there is no universal relationship between the surface density of the total gas and the SFR. They also showed that the SFR and the molecular hydrogen have a linear relationship. \cite{Shetty13} used the same data as \cite{Bigiel08} and argued that using the hierarchical Bayesian fit leads to significant galaxy-by-galaxy variation between power indices and all of them are lower than \cite{Bigiel08}. Although some of our results in table`\ref{table: res} showed the nearly linear correlation between the SFR and the molecular hydrogen, we couldn't find any specific trend between power indices in the different regions. 

The sub-linear power index in the case of the total gas surface density suggests that the star formation efficiency is decreasing by increasing the total gas. Consequently, the depletion time increases by increasing the total gas. On the contrary, the super-linear relation suggests that the depletion time is decreasing with increasing the amount of the CO or/and HI gas. The further conclusion is more similar to the initial suggestion of the K-S law. If we assume that stars form in the molecular clouds, the decreasing of the SFE with increasing the total gas is justifiable.  

The results of the hierarchical Bayesian fitting on the regions with different distances from the centre of the galaxy give us similar values as the fitting to the whole galaxy data. For all three regions, the K-S law using the total gas leads to sub-linear relation between the SFR and the surface density of the total gas. Likewise, for using the H$_{2}$ only gas or HI only gas, this relationship is either nearly linear or super-linear. These results agree with those from the models of \cite{Krumholz09}. They showed that for the regions with \sigmagas $\leq 100$ M$_{\odot}$pc$^{-2}$, \sigmasfr and $\Sigma_{H_2}$ have a nearly linear correlation. They also conclude that, at the intermediate total gas column density, (\sigmagas $\leq 100$ M$_{\odot}$pc$^{-2}$), \sigmasfr and \sigmagas have a linear or slightly sub-linear correlation which is similar to our results. However, they also showed the SFR has dependency on metallicity of the galaxy which we could not see any correlation between metallicity and the SFR in M31, (~\ref{fig:metal}). 

The K-S law power index varies with the distance of the centre of the galaxy. Similar to results from \cite{Ford13}, the power index increases by going to further distances from the centre of the galaxy considering molecular gas only. The power index using total gas or atomic gas does not show any particular correlation with distance. The increasing of the power index while using molecular cloud only may be an effect of the decreasing metallicity across the galaxy or other physical properties of the galaxy such as stellar mass (see section ~\ref{sec:es_res}).

  
\subsection{The extended Schmidt law in M31}
\label{sec:es_res}
The extended Schmidt law has been a recently proposed law \citep{Shi11}, which shows a tight correlation between the SFR and the total gas surface density and the surface density of the stellar mass. Our results from fitting whole galaxy shows a sub-linear relation between \sigmasfr and \sigmastar.

Using only atomic gas $\beta$ is in good agreement with the original suggestion of the extended Schmidt law $\eqnprime = 0.8 \pm 0.01$ and $\beta = 0.62\pm0.01$. However, $\beta$ is way smaller in other two gas tracers. Using total gas mass, we find $\eqnprime$ to be very close to $\beta$. This relation suggests that, although increasing the total gas causes a decrease to the SFE but the existing stars will increase the SFE. 

The fitting results on the three regions (see table ~\ref{table:res}) shows the same results as the fitting whole galaxy together. However, taking the average $\beta$ from results of using molecular gas only gives us $\beta = 0.66$ which is in good agreement with \cite{Shi11}, but $\eqnprime$ is almost twice of the original suggestion. However, the final indices reported in the extended Schmidt law are an average over 12 different spiral galaxies, and some of their results (e.g. NGC 4736, NGC 4726) have similar results to M31. 

Similar to the K-S law, the extended Schmidt law is sensitive to fitting method. Using the OLS fitting, one only will consider the SFR and stellar mass surface density errors, nevertheless, the hierarchical Bayesian regression fitting considers uncertainties on all the parameters.
 

Thus we can conclude that the stars affect on the SFR in most cases this effect is more on the regions with lower amount of the gas. \cite{Kim13} using 3D numerical hydrodynamic simulations showed that in outer disk regions of the galaxy  where the total gas is dominated by HI gas, the SFR have correlation with $\rho_{sd}^{0.5}$ where $\rho_{sd}$ is the mid-plane density of the stellar disk plus dark matter. This is because of the fact that in outer regions of the disk the gravity from stars dominates. This results are in agree with our finding regarding that in outer regions of M31 the stellar surfaces density has more effect on the SFR.% we cannot say that about Halpha map, but these regions are the ones, halpha map has more uncertaiintly on it!

\section{Metallicity}

It is well established that metallicity in the centre of M31 is more than the outer disk regions \citep[e.g.][]{Draine14}. Inverse correlation between metallicity and x-factor, which was described in section~\ref{sec:intro} gives the underestimation of the power indices from fitting the K-S law in the case of the molecular gas only and total gas. Thus power index of the K-S law is more closer to the original suggestion for the KS law ($\lt N \sim 1.56$). Moreover, results of \cite{Krumholz09} showed that a correlation between SFR and metallicity is strongly depends on amount of total gas in galaxies. We could not see any correlation between the SFR and metallicity in sub-kiloparces regions~\ref{fig:metal}. We estimated the SFR using \cite{Krumholz09} star formation rate for regions with size of $\sim$750~pc. We found that the SFR calculated using Krumholz law in average 45$\%$ different from the measured SFR. We used an average value of metallicity 1.65, clumping factor of 1.2 to get this results. A through calculation is needed to solve the issue of this controversy which is beyond scope of this paper.  

 \cite{Mannucci10} and \cite{Lilly13} assumed the K-S law holds for their sample of study and obtained a correlation between, metallicity and the stellar mass, the gas mass and the SFR. They found that increasing both the SFR and the stellar mass increases metallicity, but increasing the gas mass decreases metallicity (equation 9 in \citep{Mannucci10}). However, by considering results from \cite{wong13} regarding the dependency of rate of HI-H$_2$ conversion to metallicity, \cite{Roychowdhury15} showed that the KS law is independent of metallicity in galaxies with HI as a dominant gas. As demonstrated in ~\ref{fig:metal}, we plot metallicity versus the SFR and the stellar mass, and could not find any correlation.

\begin{figure*}
    \centering
    \begin{subfigure}[b]{053\textwidth}
        \centering
        \includegraphics[width=\textwidth]{metal_vs_sfr.png}
        \caption{SFR(FUV + 24 $\mu$m) vs matalicity surface density}
        \label{fig:mtal_sfr}
    \end{subfigure}
    \hfill
    \begin{subfigure}[b]{0.5\textwidth}
        \centering
        \includegraphics[width=\textwidth]{metal_vs_star.png}
        \caption{Stellar mass vs metalicity surface density}
        \label{fig:metal_star}}
    \end{subfigure}
    \caption{Right: the SFR(FUV + 24 $\mu$m) surface density versus metalicity surface density and left: Stellar mass vs metalicity surface density  shows the stellar mass. Each point shows a region with size of $\sim$750~pc. The regions with $R< 8\kpc$, $8\kpc < R < 18\kpc$, and $18\kpc < R \la 25\kpc$ are shown in red, green and blue, respectively.}
    \label{fig:metal}
\end{figure*}

%----------------------------------------------------------------------------------------
%SUMMERY
%----------------------------------------------------------------------------------------
\section{Summary}
 We investigated the K-S law, the extended Schmidt law and the Krumholz law in both global and local scales in M31. We have determined surface density of star formation rate, gas mass and the stellar mass of this galaxy. We produced three SFR maps in M31. The first uses a combination of the FUV and 24 \um emission. For the second SFR map, we used a combination of the H$\alpha$ and 24 \um emission, and the third one was produced using total-infrared luminosity. We noticed that H$_\alpha$ data available for M31 is not suitable for SFR study ~\ref{app:halpha}. We calculated the total SFR from FUV and 24 \um emission is 0.31$\pm$ 0.04 M$_{\odot}$yr$^{-1}$. We also produced the ISM map, using molecular gas only, atomic gas only and the total gas.

1) We found the power index of the K-S law is mostly depended on which gas we use as a tracer of the gas mass in galaxy. And it is mostly independent of the SFR tracer.

2) We showed that, using different fitting methods gives us different results

3) Our results showed although there is a correlation between \sigmasfr and \sigmagas, it is not the same in all regions in M31. 

4) We confirmed \cite{Shi11} suggestions that the surface density of stars has impact on the SFR, and in regions with law gas surface brightness this impact is even more important. 

5) We could not find any relation between SFR and metallicity.


%----------------------------------------------------------------------------------------
%Acknowledgement 
%----------------------------------------------------------------------------------------
%\section*{Acknowledgement}
The authors thank R. Shetty for giving us his hierarchical Bayesian fitting code, D. Kruijssen for his idea about 3D plots, L. Chemin for giving us HI map, K. Gordon for his latest version of the Mips 24 \um and 70 \um maps, and M. Smith for \Herschel data. We also want to thank B.T Draine, P. Massey, K. Sandstorm, M. Azimlu, and G. Ford for their useful suggestions.
The authors acknowledge research support from the Natural Sciences and Engineering Research Council of Canada and from the Academic Development Fund of the University of Western Ontario. 
This research made use of Montage, funded by the National Aeronautics and Space Administration's Earth Science Technology Office, Computation Technologies Project, under Cooperative Agreement Number NCC5-626 between NASA and the California Institute of Technology. Montage is maintained by the NASA/IPAC Infra-red Science Archive.
%----------------------------------------------------------------------------------------
%BIBLOGRAPHY
%----------------------------------------------------------------------------------------

\bibliography{ref.bib}

%----------------------------------------------------------------------------------------
%Appendix
%----------------------------------------------------------------------------------------

\appendix
\section{SFR from H$\alpha$ plus 24 \um produces and result}
\label{app:halpha}


\begin{figure*}
\centering
\includegraphics[width=164mm]{halpha.eps}
\caption{Mosaic created using the Montage programme from six fields of H$\alpha$ emission maps of M31 from \cite{Massey07}. The result image from Montage was continuum subtracted and masked out for all point sources. Centre of the galaxy was masked out due to saturation of data in an R-band image.}
\label{fig:halpha}
\end{figure*}

As mentioned in Section~\ref{sec:data}, we used the H$\alpha$ data from the Nearby Galaxies Survey \citep{Massey07} to create one of our SFR maps. These data cover 10 overlapping fields across the disk of M31 in broad and narrow bands in 10 overlapping fields and were observed between August 2000 and September 2002; we obtained them  from the NOAO science archive. Making a spatially resolved SFR map of M31 using H$\alpha$ emission requires removal of background and stellar emission in each field, masking out foreground stars and saturated regions, creating a mosaic of H$\alpha$ continuum-subtracted map, and finally correcting for the flux contribution to the \halpha filter from the [N II] 6583~\AA line.

 R-band images were used to remove the stellar emission from H$\alpha$ using scaling factor between fluxes in these two band passes determined by \cite{Azimlu11}. They estimated continuum emission in each H$\alpha$ image, using photometry results in both \halpha and R-band images, and determined the scaling factor. These images were observed from a ground base telescope over two years time. Consequently, there is a non-negligible sky contribution in the background which should be estimated and removed from each image in both bands. On the other hand, there is no image of the further area in the sky to measure the sky contribution from there, which makes the removal of background emission even harder. The only choice were left here is to subtract the local background for each region. However, subtracting a local background of images will result in a non-smooth background in final mosaic image. 

To create the final mosaic, first we removed the background from each region in both \halpha and R-band. Our second step was to subtract the continuum from the \halpha images. Since both \halpha and R-band images were aligned on the same coordinate grid, for each field, \halpha image was subtracted by R-band image multiplied by the scaling factor. At the end, we masked out all the foreground and saturated regions which include a $10\arcmin \times 10\arcmin$ region in the centre of the galaxy. To account for the flux contribution of the [N II] emission we used the flux ratio of the [NII]$/$\halpha $= 0.54$ from \cite{Kennicutt08}, and subtracted it from \halpha map. We created the final mosaic image~\ref{fig:halpha} using the Montage program \citep{Berriman08}.

Creating the SFR map using \halpha plus 24 \um was described in Section~\ref{sec:sfr_halpha}. In order to investigate the SFR laws, we used the same method of the fitting as Section~\ref{sec:fitting}. The fitting results using SFR(\halpha $+$ 24 \um ) are more or less the same as the other two SFR tracers. The main reason for this difference could is mainly because of lost data in the centre of the galaxy.\halpha data does not have a smooth background. Thus we did not include the SFR(\halpha $+$ 24 \um ) in our final analysis.






\end{document}
